\documentclass[12pt]{article}
\usepackage{graphicx}
\usepackage[margin=.5in]{geometry} 
\usepackage{amsmath,amsthm,amssymb}
%\usepackage{gensymb}
  \usepackage{hyperref}
  \usepackage{pdfpages} 
  %\usepackage[table]{xcolor}
  
 %\usepackage{xcolor} % Required for specifying colors by name
\definecolor{ocre}{RGB}{52,177,201} % Define the orange color used for highlighting throughout the book

% Font Settings
\usepackage{avant} % Use the Avantgarde font for headings
%\usepackage{times} % Use the Times font for headings
\usepackage{mathptmx} % Use the Adobe Times Roman as the default text font together with math symbols from the Sym­bol, Chancery and Com­puter Modern fonts

%\usepackage{microtype} % Slightly tweak font spacing for aesthetics
%\usepackage[utf8]{inputenc} % Required for including letters with accents
\usepackage[T1]{fontenc} % Use 8-bit encoding that has 256 glyphs
%\usepackage{soul}
\usepackage{undertilde}
%\usepackage{accents}
\newcommand{\StandardLM}{\by=\bX \bbeta +{\epsilonbf}}

%\usepackage{xcolor}
\usepackage{xcolor}
\usepackage{xparse}
\definecolor{lightGray}{gray}{0.95}
\definecolor{lightGrayOne}{gray}{0.9}
\definecolor{lightBlueOne}{RGB}{204, 255, 255}
\definecolor{lightBlueTwo}{RGB}{204, 238, 255}
\definecolor{lightBlueThree}{RGB}{204, 204, 255}
\definecolor{AltBlue}{RGB}{119,14,161}


\definecolor{BGBlue}{RGB}{220,221,252}
\definecolor{BGBlueOne}{RGB}{204,229,255}



\definecolor{BGGreen}{RGB}{240,243,245}
\definecolor{lightGreenOne}{RGB}{179, 255, 179}
\definecolor{lightGreenTwo}{RGB}{198, 255, 179}
\definecolor{lightGreenThree}{RGB}{243, 255, 230}
\definecolor{AltGreen}{RGB}{193, 240, 240}

\definecolor{BOGreen}{RGB}{180,0,0}
\definecolor{BGGreenOne}{RGB}{220,250,220}

\definecolor{lightBrownOne}{RGB}{255, 221, 204}
\definecolor{lightBrownTwo}{RGB}{255, 229, 204}
\definecolor{lightBrownThree}{RGB}{242, 217, 230}


\definecolor{HLTGreen}{RGB}{230,244,215}
\definecolor{ExcBrown}{RGB}{153,0,0}
\definecolor{ExcBGBrown}{RGB}{255,204,204}
\definecolor{BGYellowOne}{RGB}{255,235,208}
\definecolor{BGPink}{RGB}{255,215,240}



\NewDocumentCommand{\HLT}{ O{HLTGreen} m }{\colorbox{#1}{#2}}
\NewDocumentCommand{\HLTEQ}{ O{HLTGreen} m }{\colorbox{#1}{$#2$}}

%\newcommand{\HLT}[1]{\colorbox{HLTGreen}{#1}}
\newcommand{\DEHLT}[1]{\colorbox{lightGrayOne}{\color{white} #1}}

\newcommand{\TextInBoxOne}[2]{  {\fcolorbox{lightGrayOne}{white}{\begin{minipage}{#1}  #2 \end{minipage}}}}

\newcommand{\TextInBoxOneQ}[2]{  {\fcolorbox{white}{lightGrayOne}{\begin{minipage}{#1}  #2 \end{minipage}}}}

\newcommand{\TextInBoxOneEQ}[2]{  {\fcolorbox{white}{lightBlueTwo}{\begin{minipage}{#1}  #2 \end{minipage}}}}

\newcommand{\QuizQuestion}[3]{  {\fcolorbox{black}{white}{\begin{minipage}{5.6 in}
\TextInBoxOneEQ{5.5in}{ #1 }\\
{\large \HLTEQ{\hspace{4.61in}\frac{\text{Score: \;\;\;\;}}{\text{#3}}}}\\
\vspace{.01in}#2 \end{minipage}}}}

\newcommand{\QuizQ}[3]{  {\fcolorbox{black}{lightGrayOne}{\begin{minipage}{5.6 in}
\TextInBoxOne{5.5in}{ #1 }\\
\vspace{.01in}#2 \end{minipage}}}}



\newcommand{\ExamQuestion}[3]{  {\fcolorbox{lightBlueTwo}{lightBlueTwo}{\begin{minipage}{5.85 in}
\TextInBoxOne{5.8in}{ #1 }\\
{\large \HLTEQ[lightBlueTwo]{\hspace{5.01in}\frac{\text{Score: \;\;\;\;}}{\text{#3}}}}\\
\end{minipage} }
#2 }}


\NewDocumentCommand{\MCOption}{O{1.75 in}m}{
\TextInBoxTwo[BGPink]{ #1 } {\TextInBoxTwo[white]{.1 in }{ \quad}\HLT{#2}}
}




\NewDocumentCommand{\MCOptionSelected}{m}{
\TextInBoxTwo[BGPink]{ 1.75 in } {\TextInBoxTwo[white]{.1 in }{{\huge $\bullet$}}\HLT{#1}}
}


%
%\NewDocumentCommand{\MCOption}{m}{
%\TextInBoxTwo[white]{.1 in }{ \quad}\HLT{#1}}







\NewDocumentCommand{\TextInBoxTwo}{ O{lightGrayOne} m m } {{\fcolorbox{white}{#1}{\begin{minipage}{#2} { #3} \end{minipage}}}}


\newcommand{\TextInBox}[2]{  {\fcolorbox{BGGreen}{BGGreen}{\begin{minipage}{#1}  #2 \end{minipage}}}}
\newcommand{\TextInBoxCol}[2]{
\fcolorbox{BGBlue}{BGBlue}{%
\begin{minipage}{#1}
 {\color{AltBlue} #2}
\end{minipage}}%
}




\newcommand{\BandInTopBox}[2]{
\fcolorbox{AltBlue}{AltBlue}{%
\begin{minipage}{#1}{ {\color{white}  #2 \hspace{.1in}} }
\end{minipage}}%
}


\newcommand{\TextInBoxThm}[2]{
\fcolorbox{AltBlue}{lightGray}{%
\begin{minipage}{#1}
 {\color{black} #2}
\end{minipage}}%
}

\newcommand{\TextInBoxThmOne}[2]{
\fcolorbox{BGBlue}{BGBlueOne}{%
\begin{minipage}{#1}
 {\color{AltBlue} #2}
\end{minipage}}%
}

\newcommand{\TextInBoxLem}[2]{
\fcolorbox{BGBlue}{lightGray}{%
\begin{minipage}{#1}
 {\color{black} #2}
\end{minipage}}%
}



\newcommand{\TextInBoxLemOne}[2]{
\vspace{.02 in}
\noindent
\fcolorbox{BGBlue}{BGBlue}{%
\begin{minipage}{#1}
 {\color{AltBlue} #2}
\end{minipage}}%
}


\newcommand{\CmntBox}[1]{
\noindent
\TextInBoxLem{5.3 in }{
\TextInBoxLemOne{5.2 in }{
#1
}}

}

\newcommand{\DefBox}[1]{
%\vspace{.1 in}
\noindent
\TextInBoxLem{6 in }{
\BandInTopBox{5.9 in }{}
\TextInBoxLemOne{5.9 in }{
#1
}}}


\newcommand{\DefBoxL}[1]{
%\vspace{.1 in}
\noindent
\TextInBoxLem{8 in }{
\BandInTopBox{7.9 in }{}
\TextInBoxLemOne{7.9 in }{
#1
}}}




%Old measurements
%\newcommand{\DefBoxOne}[2]{
%%\vspace{.1 in}
%\noindent
%\TextInBoxLem{6 in }{
%\BandInTopBox{5.9 in }{#1}
%\TextInBoxLemOne{5.9 in }{
%#2
%}}}
%

\newcommand{\DefBoxOne}[2]{
%\vspace{.1 in}
\noindent
\TextInBoxLem{6.8 in }{
\BandInTopBox{6.7 in }{#1}
\TextInBoxLemOne{6.7 in }{
#2
}}}


\newcommand{\ThmBox}[2]{
\noindent
\TextInBoxThm{6.8 in }{
\TextInBoxThmOne{6.7 in }{
#1}
#2}
}

\newcommand{\LemBox}[2]{
\noindent
\TextInBoxLem{6.8 in }{
\TextInBoxLemOne{6.7 in }{
#1}
#2}
}

\newcommand{\PropBox}[2]{
\vspace{.1 in}
\noindent
\TextInBoxLem{6.8 in }{
\TextInBoxLemOne{6.7 in }{
#1}
#2}
}




\newcommand{\TextInBoxExc}[2]{
\noindent
\fcolorbox{white}{BGGreenOne}{%
\begin{minipage}{#1}
 {\color{black} #2}
\end{minipage}}%
}


\newcommand{\TextInBoxExample}[2]{
\noindent
\fcolorbox{white}{BGPink}{%
\begin{minipage}{#1}
 {\color{black} #2}
\end{minipage}}%
}


\newcommand{\ExerciseBox}[1]{
\noindent
%\TextInBoxLem{6 in }{
\TextInBoxExc{6 in }{
#1}
%#2}
}


\newcommand{\ExampleBox}[1]{
\noindent
%\TextInBoxLem{6 in }{
\TextInBoxExample{6 in }{
#1}
%#2}
}


\newcommand{\IndicatorA}[2]{\mathbb{I}_{#2}({#1 })}


 
%\newcommand{\proof}{ {\bf Proof:  } }
%\usepackage{enumerate}
%


\newcommand{\MakeVec}[1]{{\utilde{\bf #1}}}
\newcommand{\Ind}[1]{\mathbb{I}\left(#1 \right)}
\newcommand{\StandardLMmod}{\bY=\bX \bbeta +{\epsilonbf}}
\NewDocumentCommand{\YiDotDef}{O{B} m}{ \left(Y_{{#2},1},\ldots,Y_{{#2},{#1}} \right)}
\NewDocumentCommand{\YiDot}{O{i}}{  \utilde{Y}_{{#1, \bullet}}  }


\NewDocumentCommand{\YijDotDef}{O{T}O{B}O{R}}{ \left(Y_{{#1},{#2},1},\ldots,Y_{{#1},{#2},{#3}} \right)}

\NewDocumentCommand{\YijDot}{O{i}O{j}}{  \utilde{Y}_{{#1},{#2}, \bullet}  }

\NewDocumentCommand{\YiDotDot}{O{i}}{  \utilde{Y}_{{#1,\bullet, \bullet}}  }

\NewDocumentCommand{\YDotjDot}{O{j}}{  \utilde{Y}_{{\bullet,#1, \bullet}}  }


\newcommand{\ba}{\MakeVec{a}}
\newcommand{\bb}{\MakeVec{b}}

%%%%%%%%%%%% ONEWAY MODEL


\NewDocumentCommand{\OneWay}{ O{T} O{B} m}{
 \IfEqCase{#3}{%
  {model}{   Y_{i,j}=\mu+ \tau_i+ \epsilon_{i,j} \text{ for } i 		=1, 2,\ldots #1; j = 1,2,\ldots #2  
  	 }
  {Y}{
  \left[ {\begin{array}{c;{2pt/2pt}c;{2pt/2pt}c;{2pt/2pt}c ;{2pt/2pt}c}
   \overbrace{ Y_{1,1},\ldots ,Y_{1,{#2}}}^{\text{$1^{st}$ Treatment}}  & \cdots &  \overbrace{ Y_{i,1},\ldots Y_{i,{#2}}}^{\text{$i^{th}$ Treatment}} & \cdots & \overbrace{Y_{{#1},1} \ldots Y_{{{#1}},{#2}}}^{\text{#1$^{th}$ Treatment }} \end{array} } \right]^T 
  	 }
  	 {YInDot}{\left[ {\begin{array}{c;{2pt/2pt}c;{2pt/2pt}c;{2pt/2pt}c ;{2pt/2pt}c}
  \utilde{Y}_{1, \bullet}^T  & \cdots &   \utilde{Y}_{i, \bullet}^T& \cdots & \utilde{Y}_{{#1}, \bullet}^T \end{array} } \right]^T
  	 }
  	 {response}{
  \left[ {\begin{array}{c;{2pt/2pt}c;{2pt/2pt}c;{2pt/2pt}c ;{2pt/2pt}c}
   \overbrace{ Y_{1,1},\ldots ,Y_{1,{#2}}}^{\text{$1^{st}$ Treatment}}  & \cdots &  \overbrace{ Y_{i,1},\ldots Y_{i,{#2}}}^{\text{$i^{th}$ Treatment }} & \cdots & \overbrace{Y_{{#1},1} \ldots Y_{{{#1}},{#2}}}^{\text{#1$^{th}$Treatment}} \end{array} } \right]^T  
  	 }
  	 {treatments}{ \tau_1,\ldots , \tau_{#1} }
  	  {tau}{ \tau_1,\ldots , \tau_{#1} }
  	 {beta}{\left[\mu, \HLT{$\tau_1,\ldots , \tau_{#1} $}\right]^T}
  	 {error}{
  \left[ {\begin{array}{c;{2pt/2pt}c;{2pt/2pt}c;{2pt/2pt}c ;{2pt/2pt}c}
   \overbrace{ \epsilon_{1,1},\ldots ,\epsilon_{1,{#2}}}^{\text{$1^{st}$ Treatment}}  & \cdots &  \overbrace{ \epsilon_{i,1},\ldots \epsilon_{  i,{#2}}  }^{\text{$i^{th}$ Treatment}} & \cdots & \overbrace{\epsilon_{{#1},1} \ldots \epsilon_{{{#1}},{#2}}}^{\text{#1 $^{th}$ Treatment }} \end{array} } \right]^T   
  	 }
  	 {design}{
 \left[ {\begin{array}{c;{2pt/2pt}cccc}
   \Onebf_{#2} &  \Onebf_{#2} & \ZeroF  & \ldots &  \ZeroF\\
   \Onebf_{#2} &  \ZeroF   &\Onebf_{#2} & \ldots  & \ZeroF\\
   \vdots   & \vdots    & \vdots  & \ddots & \vdots  \\
    \Onebf_{#2} & \ZeroF & \cdots  & \ldots    & \Onebf_{#2}\\
    \end{array}
   } \right] _{{#1}{#2}\times ({#1}+1)}
  }
  {designKP}{ \left[  {\begin{array}{c;{2pt/2pt}c}
   \underbrace{\Onebf_{#1}\otimes  \Onebf_{#2}} &  \underbrace{I_{#1} \otimes \Onebf_{#2} }
   \end{array} }\right]
   }
    {X}{
 \left[ {\begin{array}{c;{2pt/2pt}cccc}
   \Onebf_{#2} &  \Onebf_{#2} & \ZeroF  & \ldots &  \ZeroF\\
   \Onebf_{#2} &  \ZeroF   &\Onebf_{#2} & \ldots  & \ZeroF\\
   \vdots   & \vdots    & \vdots  & \ddots & \vdots  \\
    \Onebf_{#2} & \ZeroF & \cdots  & \ldots    & \Onebf_{#2}\\
    \end{array}
   } \right]
  }
  {XKP}{ \left[  {\begin{array}{c;{2pt/2pt}c}
   \underbrace{\Onebf_{#1}\otimes  \Onebf_{#2}} &  \underbrace{I_{{#1}} \otimes \Onebf_{#2} }
   \end{array} }\right]
   }
   {XMu}{ \Onebf_{#1}\otimes  \Onebf_{#2} }
   {XTau}{I_{{#1}} \otimes \Onebf_{#2}}
   {ProjMat}{
   \left[ {\begin{array}{c;{2pt/2pt}c;{2pt/2pt}c ;{2pt/2pt}c}
    \HLT{$\ProjOne{#2}$}&  \ZeroF& \cdots &\ZeroF\\
   \ZeroF&  \HLT{$\ProjOne{#2}$} & \cdots & \ZeroF\\
   \vdots &\vdots  &  \vdots   & \vdots  \\
    \ZeroF&  \ZeroF & \cdots & \HLT{$\ProjOne{#2}$}
    \end{array}
   } \right]_{{#1}{#2}\times {#1}{#2} }
   }
    {ProjMatKP}{
    I_{#1}\otimes {\ProjOne{#2}}
    }
    {YColVec}{
    \left[ {\begin{array}{c}
  \utilde{Y}_{1 \bullet}\\
  \hdashline[2pt/2pt]\\
   \vdots\\
  \hdashline[2pt/2pt]\\
  \utilde{Y}_{i \bullet}\\
  \hdashline[2pt/2pt]\\
   \vdots\\
  \hdashline[2pt/2pt]\\
   \utilde{Y}_{{#1} \bullet}\\
    \end{array}
   } \right]_{{#1}{#2}\times 1}}
   {YDotBar}{\left[
   {\begin{array}{c}
  \overline{Y}_{1 \bullet}\\
  \hdashline[2pt/2pt]\\
   \vdots\\
  \hdashline[2pt/2pt]\\
  \overline{Y}_{i \bullet}\\
  \hdashline[2pt/2pt]\\
   \vdots\\
  \hdashline[2pt/2pt]\\
   \overline{Y}_{{#1} \bullet}\\
    \end{array}
   }\right]_{{#1}\times 1} }
    }  	 
}


%%%%%%%%%%%%%%%% TWO WAY ANOVA

\NewDocumentCommand{\TwoWay}{ O{T} O{B} O{R} m}{
 \IfEqCase{#4}{%
  {model}{   Y_{i,j, k}=\mu+ \tau_i+\delta_j+ \epsilon_{i,j,k} \text{ for } i =1, 2,\ldots #1; j = 1,2,\ldots #2 ;  j = 1,2,\ldots #3  
  	 }
  {Y}{ {\left[ {\begin{array}{c;{2pt/2pt}c;{2pt/2pt}c;{2pt/2pt}c ;{2pt/2pt}c}
   \overbrace{ Y_{1,1,1},\ldots ,Y_{1,1,{#3}},  Y_{1,2,1},\ldots ,Y_{1,{#2},{#3}}}^{\text{Treatment 1}}  & \cdots &  \overbrace{ Y_{i,1,1},\ldots Y_{i,1,{#3}},\ldots,Y_{i,{#2},1},\ldots Y_{i,{#2},{#3}}  }^{\text{Treatment i}} & \cdots & \overbrace{ Y_{{#1},1,1},\ldots Y_{{#1},1,{#3}},\ldots,Y_{{#1},{#2},1},\ldots Y_{{#1},{#2},{#3}}  }^{\text{Treatment }{#1}} \end{array} } \right]^T}
  	 }
  	 {YInDot}{\left[ {\begin{array}{c;{2pt/2pt}c;{2pt/2pt}c;{2pt/2pt}c ;{2pt/2pt}c}
  \utilde{Y}_{1,1, \bullet}^T  & \cdots &   \utilde{Y}_{i,j, \bullet}^T& \cdots & \utilde{Y}_{{#1},{#2}, \bullet}^T \end{array} } \right]^T
  	 }
  	 {response}{  {\left[ {\begin{array}{c;{2pt/2pt}c;{2pt/2pt}c;{2pt/2pt}c ;{2pt/2pt}c}
   \overbrace{ Y_{1,1,1},\ldots ,Y_{1,1,{#3}},  Y_{1,2,1},\ldots ,Y_{1,{#2},{#3}}}^{\text{Treatment 1}}  & \cdots &  \overbrace{ Y_{i,1,1},\ldots Y_{i,1,{#3}},\ldots,Y_{i,{#2},1},\ldots Y_{i,{#2},{#3}}  }^{\text{Treatment i}} & \cdots & \overbrace{ Y_{{#1},1,1},\ldots Y_{{#1},1,{#3}},\ldots,Y_{{#1},{#2},1},\ldots Y_{{#1},{#2},{#3}}  }^{\text{Treatment }{#1}} \end{array} } \right]^T}  	 
  	 }
  	 {treatments}{ \tau_1,\ldots , \tau_{#1} }
  	  {tau}{ \tau_1,\ldots , \tau_{#1} }
  	  {block}{\delta_1, \ldots , \delta_{#2}}
  	  {delta}{\delta_1, \ldots , \delta_{#2}}
  	 {beta}{{\RowVec{{\mu}, {\taubf^T}, {\deltabf^T}}}^T}
  	 {betaCol}{\ColVec{\mu, \taubf, \deltabf}}
  	 {error}{  {\left[ {\begin{array}{c;{2pt/2pt}c;{2pt/2pt}c;{2pt/2pt}c ;{2pt/2pt}c}
   \overbrace{ {\epsilon}_{1,1,1},\ldots ,{\epsilon}_{1,1,{#3}},  {\epsilon}_{1,2,1},\ldots ,{\epsilon}_{1,{#2},{#3}}}^{\text{Treatment 1}}  & \cdots &  \overbrace{ {\epsilon}_{i,1,1},\ldots {\epsilon}_{i,1,{#3}},\ldots,{\epsilon}_{i,{#2},1},\ldots {\epsilon}_{i,{#2},{#3}}  }^{\text{Treatment i}} & \cdots & \overbrace{ {\epsilon}_{{#1},1,1},\ldots {\epsilon}_{{#1},1,{#3}},\ldots,{\epsilon}_{{#1},{#2},1},\ldots {\epsilon}_{{#1},{#2},{#3}}  }^{\text{Treatment }{#1}} \end{array} } \right]^T}  
  	 }
  	 {design}{  \left[  {\begin{array}{c;{2pt/2pt}c}
   \underbrace{\Onebf_{#1}\otimes  \Onebf_{#2}} &  \underbrace{I_{#1} \otimes \Onebf_{#2} }
   \end{array} }\right]
  }
  {designKP}{ \left[  {\begin{array}{c;{2pt/2pt}c;{2pt/2pt}c}
   \underbrace{\Onebf_{#1}\otimes  \Onebf_{#2} \otimes  \Onebf_{#3}} &  \underbrace{I_{#1} \otimes \Onebf_{#2}\otimes \Onebf_{#3}  } &  \underbrace{\Onebf_{#1} \otimes I_{#2}\otimes \Onebf_{#3}  }
   \end{array} }\right]
   }
    {X}{  \left[  {\begin{array}{c;{2pt/2pt}c;{2pt/2pt}c}
   \underbrace{\Onebf_{#1}\otimes  \Onebf_{#2} \otimes  \Onebf_{#3}} &  \underbrace{I_{#1} \otimes \Onebf_{#2}\otimes \Onebf_{#3}  } &  \underbrace{\Onebf_{#1} \otimes I_{#2}\otimes \Onebf_{#3}  }
   \end{array} }\right]    
  }
  {XKP}{ \left[  {\begin{array}{c;{2pt/2pt}c;{2pt/2pt}c}
   \underbrace{\Onebf_{#1}\otimes  \Onebf_{#2} \otimes  \Onebf_{#3}} &  \underbrace{I_{#1} \otimes \Onebf_{#2}\otimes \Onebf_{#3}  } &  \underbrace{\Onebf_{#1} \otimes I_{#2}\otimes \Onebf_{#3}  }
   \end{array} }\right]
   }
   {XMu}{ \Onebf_{#1}\otimes  \Onebf_{#2} \otimes \Onebf_{#3} }
   {XTau}{I_{{#1}} \otimes \Onebf_{#2} \otimes \Onebf_{#3}}
   {XDelta}{\Onebf_{{#1}} \otimes I_{#2} \otimes \Onebf_{#3}}
   {ProjMat}{P_{\bX}  }
    {ProjMatKP}{P_{\bX}    }
    {YColVec}{
    \left[ {\begin{array}{c}
  \utilde{Y}_{1 \bullet}\\
  \hdashline[2pt/2pt]\\
   \vdots\\
  \hdashline[2pt/2pt]\\
  \utilde{Y}_{i \bullet}\\
  \hdashline[2pt/2pt]\\
   \vdots\\
  \hdashline[2pt/2pt]\\
   \utilde{Y}_{{#1} \bullet}\\
    \end{array}
   } \right]_{{#1}{#2}\times 1}}
   {YDotBar}{\left[
   {\begin{array}{c}
  \overline{Y}_{1 \bullet}\\
  \hdashline[2pt/2pt]\\
   \vdots\\
  \hdashline[2pt/2pt]\\
  \overline{Y}_{i \bullet}\\
  \hdashline[2pt/2pt]\\
   \vdots\\
  \hdashline[2pt/2pt]\\
   \overline{Y}_{{#1} \bullet}\\
    \end{array}
   }\right]_{{#1}\times 1} }
    }  	 
}






%%%%%%%%%%%%Others










%\usepackage{accents}
\newcommand{\SpaceU}{\mathcal{U}}
\newcommand{\Span}[1]{\mathcal{L}(#1)}
%\hypersetup{colorlinks=true}
\newcommand{\N}{\mathbb{N}}
\newcommand{\Z}{\mathbb{Z}}
 \newcommand{\SpaceW}{\mathcal{W}}
\newcommand{\SpaceV}{\mathcal{V}}
\newcommand{\real}[1]{{\mathbb R}^{#1}}
\newcommand{\Pdg}{P_{\alphabfs (\Deltabfs_{y})}}
\newcommand{\spn}{\mathrm{span}}
\newcommand{\diag}{\mathrm{diag}}
\newcommand{\E}{\mathrm{E}}
\newcommand{\var}{\mathrm{Var}}
\newcommand{\cov}{\mathrm{Cov}}
\newcommand{\covhat}{\widehat{\mathrm{Cov}}}
\newcommand{\rank}{\mathrm{rank}}
\newcommand{\stack}{\mathrm{stack}}
\newcommand{\Normal}{\mathrm{Normal}}
\newcommand{\tr}{\mathrm{\,tr}}
\newcommand{\avar}{\mathrm{\,avar}}
\newcommand{\vecc}{\mathrm{\,vec}}
\newcommand{\true}{\mathrm{true}}
\newcommand{\I}{\mathbf I}
\newcommand{\m}{\mathbf m}
\newcommand{\SpaceC}{\mathcal{C}}
\newcommand{\SpaceR}{\mathcal{R}}
%\newcommand{\ols}{ordinary least square}
\newcommand{\save}{\mathrm{SAVE}}
\newcommand{\sir}{\mathrm{sir}}
\newcommand{\mle}{MLE}%{maximum likelihood estimator}
%\newcommand{\Fsos}{F^{\mathrm{bire}}}
%\newcommand{\Fsoshat}{\hat{F}^{\mathrm{bire}}}
\newcommand{\Fsir}{F^{\mathrm{sir}}}
\newcommand{\Fsirhat}{\hat{F}^{\mathrm{sir}}}
\newcommand{\Fsub}{F^{\mathrm{sopt}}}
\newcommand{\Fsubhat}{\hat{F}^{\mathrm{sopt}}}
\newcommand{\Fopt}{F^{\mathrm{ire}}}
\newcommand{\Fopthat}{\hat{F}^{\mathrm{ire}}}
\newcommand{\Ffire}{F^{\mathrm{fire}}}
\newcommand{\Ffirehat}{\hat{F}^{\mathrm{fire}}}
\newcommand{\sdr}{SDR}
\newcommand{\wct}{WCT}
\newcommand{\ct}{CT}
\newcommand{\ire}{IRE}
\newcommand{\oire}{IRE}
\newcommand{\fire}{Fast IRE}
\newcommand{\mda}{MDA}
\newcommand{\name}{block inverse regression estimation}
\newcommand{\idenvec}{{\mathbf 1}}
\newcommand{\env}{\mathrm{env}}
\newcommand{\PLS}{\mathrm{pls}}
\newcommand{\spe}{\mathrm{spe}}

% Bold Face symbols
\newcommand{\vbf}{{\mathbf v}}
\newcommand{\w}{{\utilde{\mathbf w}}}
\newcommand{\X}{\mathbf X}
\newcommand{\Xhat}{\widehat{\X}}
\newcommand{\x}{{\utilde{\mathbf x}}}
\newcommand{\Y}{{\mathbf Y}}
\newcommand{\y}{\mathbf y}
\newcommand{\Xbar}{\bar{\X}}
\newcommand{\Ybar}{\bar{\Y}}
\newcommand{\ellhat}{\hat{\ell}}
\newcommand{\ellbf}{\mathbf{\ell}}
\newcommand{\ellbfhat}{\hat{\ellbf}}
\newcommand{\abf}{{\utilde{\mathbf a}}}
\newcommand{\q}{{\mathbf q}}
\newcommand{\f}{{\mathbf f}}
\newcommand{\Obf}{\mathbf O}


\newcommand{\Xcaln}{{\mathcal X}_{n}}
\newcommand{\Xbarcal}{\bar{{\mathcal X}}}
\newcommand{\Xbb}{\mathbb{X}}
\newcommand{\Fbb}{\mathbb{F}}
\newcommand{\Ybb}{\mathbb{Y}}

\newcommand{\Xbbhat}{\widehat{\mathbb{X}}}
\newcommand{\Ss}{\mathbf{S}}
\newcommand{\Ty}{\T_{y}}
\makeatletter
\renewcommand*{\@seccntformat}[1]{%
   \csname the#1\endcsname.\quad}
\makeatother
%\newcommand{\Z}{{\mathbf Z}}
\newcommand{\z}{{\mathbf z}}
\newcommand{\Zbar}{\bar{\Z}}
\newcommand{\Zhat}{\hat{\Z}}
\newcommand{\Zwidehat}{\widehat{\Z}}
\newcommand{\Sigmabfhatz}{\greekbold{\hat{\Sigma}}_{\Z}}
\newcommand{\Sigmabfhatzy}{\greekbold{\hat{\Sigma}}_{\Z|y}}
\newcommand{\Sigmabfzy}{\Sigmabf_{\Z|y}}
\newcommand{\sigmahat}{\hat{\sigma}}

\newcommand{\fit}{\mathrm{fit}}
\newcommand{\res}{\mathrm{res}}
\newcommand{\rres}{{ 11},\mathrm{res}}

\newcommand{\ffit}{{ 11},\mathrm{fit}}
\newcommand{\G}{\mathbf{G}}
\newcommand{\Ll}{\mathbf{L}}
\newcommand{\Guno}{\mathbf{G_1}}
\newcommand{\Hh}{\mathbf{H}}
\newcommand{\Ww}{\mathbf{W}}
\newcommand{\Mm}{\mathbf{M}}
\newcommand{\bw}{{\utilde{\mathbf{w}}}}


%\newcommand{\pfcpc}{PFC(PC)}
\newcommand{\pfcpc}{$\mathrm{PFC}_{\mathrm{PC}}$}
\newcommand{\pfcall}{$\mathrm{PFC}_{\mathrm{all}}$}

\newcommand{\fbf}{{\mathbf f}}
\newcommand{\fbfhat}{\hat{\fbf}}
\newcommand{\fhat}{\hat{f}}
\newcommand{\D}{\mathbf D}
\newcommand{\cbf}{{\mathbf c}}
\newcommand{\Dfbf}{\D_{\fbf}}
\newcommand{\Dfbfhat}{\D_{\fbfhat}}
\newcommand{\K}{\mathbf K}
\newcommand{\Khat}{\widehat \K}

\newcommand{\ghat}{\hat{g}}
\newcommand{\Bhat}{\widehat{\B}}
\newcommand{\Rhat}{\widehat{R}}
\newcommand{\vhat}{\widehat{\bv}}

\newcommand{\uhat}{\widehat{\bu}}
\newcommand{\gbf}{{\mathbf g}}
\newcommand{\gbfhat}{\hat{\gbf}}

\newcommand{\Dgbf}{\D_{\gbf}}
\newcommand{\Dgbfhat}{\D_{\gbfhat}}

\newcommand{\obf}{\mathbf o}
\newcommand{\Pbf}{{\mathbf P}}
\newcommand{\Qbf}{{\mathbf Q}}
\newcommand{\Qfbf}{\Qbf_{\fbf}}
\newcommand{\Qfbfhat}{\Qbf_{\fbfhat}}
\newcommand{\Qgbf}{\Qbf_{\gbf}}
\newcommand{\Qgbfhat}{\Qbf_{\gbfhat}}
\newcommand{\Pgbf}{\Pbf_{\gbf}}

\newcommand{\T}{\mathbf T}
\newcommand{\tT}{\widetilde{\T}}
\newcommand{\tV}{\widetilde{\V}}
\newcommand{\dT}{\dot{\T}}
\newcommand{\dV}{\dot{\V}}
\newcommand{\ddT}{\ddot{\T}}
\newcommand{\V}{{\mathbf V}}
\newcommand{\Vhat}{\widehat \V}
\newcommand{\bv}{{\utilde{\mathbf v}}}
\newcommand{\bu}{{\utilde{\mathbf u}}}
\newcommand{\Vhalf}{{\mathbf V}^{\half}}
\newcommand{\tL}{{\widetilde L}}
\newcommand{\bd}{\deltabf}

\newcommand{\ahat}{{\hat{a}}}
\newcommand{\bhat}{{\hat{b}}}

\newcommand{\U}{{\mathbf U}}
\newcommand{\tD}{{\tilde{D}}}
\newcommand{\W}{{\mathbf W}}
\newcommand{\dbf}{{\mathbf d}}
\newcommand{\Lbf}{{\mathbf L}}
\newcommand{\F}{{\mathbf F}}
\newcommand{\M}{{\mathbf M}}
%\newcommand{\N}{{\mathbf N}}
\newcommand{\s}{{\mathbf S}}
\newcommand{\sy}{{\mathbf S}_{y}}
\newcommand{\bbf}{{\utilde{\mathbf b}}}
\newcommand{\A}{{\mathbf A}}
\newcommand{\B}{{\mathbf B}}
\newcommand{\Q}{{\mathbf Q}}
\newcommand{\C}{{\mathbf C}}
\newcommand{\Chat}{\widehat{\mathbf C}}
\newcommand{\Dhat}{\widehat{\mathbf D}}
\newcommand{\e}{{\mathbf e}}
\newcommand{\Ebf}{{\mathbf E}}
\newcommand{\g}{\mathbf g}
\newcommand{\R}{{\mathbb{ R}}}
\newcommand{\Ghat}{\widehat{\G}}
\newcommand{\Hbf}{{\mathbf H}}
\newcommand{\h}{\mathbf h}
\newcommand{\tB}{\widetilde{\B}}
\newcommand{\tC}{\widetilde{\C}}
\newcommand{\mpc}{M_{\mathrm{\scriptscriptstyle{PC}}}}
\newcommand{\mpfc}{M_{\mathrm{\scriptscriptstyle{PFC}}}}
\newcommand{\lpc}{L_{\mathrm{\scriptscriptstyle{PC}}}}
\newcommand{\lpfc}{L_{\mathrm{\scriptscriptstyle{PFC}}}}
\newcommand{\tlpfc}{\widetilde{L}_{\mathrm{\scriptscriptstyle{PFC}}}}



% Greek Bold Face symbols

\newcommand{\greekbold}[1]{\mbox{\boldmath $#1$}}
\newcommand{\alphabf}{{\utilde{\greekbold{\alpha}}}}
\newcommand{\alphabfhat}{\widehat{\alphabf}}
\newcommand{\alphahat}{\widehat{\alpha}}
\newcommand{\alphabfs}{\greekbold{\scriptstyle \alpha}}
\newcommand{\etabf}{{\MakeVec{\greekbold{\eta}}}}
\newcommand{\etabftd}{\widetilde{\etabf}}
\newcommand{\etabfs}{\greekbold{\scriptstyle \eta}}
\newcommand{\betabf}{\greekbold{\beta}}
\newcommand{\taubf}{\MakeVec{\tau}}
\newcommand{\lambdabf}{\utilde{\greekbold{\lambda}}}
\newcommand{\etabfhat}{\hat{\greekbold{\eta}}}
\newcommand{\rhobf}{\greekbold{\rho}}
\newcommand{\betabfhat}{\widehat{\greekbold{\beta}}}
\newcommand{\betabfs}{\greekbold{\scriptstyle \beta}}
\newcommand{\taubfhat}{\hat{\greekbold{\tau}}}
\newcommand{\taubfn}{\taubf_{n}}
\newcommand{\taubfhatn}{\hat{\taubf}_{n}}
\newcommand{\Lambdabf}{\greekbold{\Lambda}}
\newcommand{\Lambdabfhat}{\widehat{\greekbold{\Lambda}}}
\newcommand{\Lambdabfs}{\greekbold{\scriptstyle{\Lambda}}}
\newcommand{\epsilonbf}{{\utilde{\greekbold{\epsilon}}}}
\newcommand{\mubfbar}{\bar{\mubf}}
\newcommand{\mubfhat}{\hat{\mubf}}
\newcommand{\mubfs}{{\greekbold{\scriptstyle \mu}}}
\newcommand{\J}{\mathbf J}
\newcommand{\gammabf}{\MakeVec{\gamma}}
\newcommand{\gammabfhat}{\hat{\greekbold{\gamma}}}
\newcommand{\gammabfy}{\greekbold{\gamma}_{y}}
\newcommand{\Gammabf}{\greekbold{\Gamma}}
\newcommand{\Gammabft}{\widetilde{\greekbold{\Gamma}}}
\newcommand{\gammabfs}{\greekbold{{\scriptstyle \gamma}}}
\newcommand{\Gammabfs}{{\greekbold{\scriptstyle \Gamma}}}
\newcommand{\Gammabfhat}{\widehat{\greekbold{\Gamma}}}
\newcommand{\deltabf}{{\MakeVec{\delta}}}
\newcommand{\Deltabf}{\greekbold{\Delta}}
\newcommand{\Deltabfhat}{\widehat{\greekbold{\Delta}}}
\newcommand{\Deltabfs}{{\greekbold{\scriptstyle \Delta}}}
\newcommand{\deltabfs}{{\greekbold{\scriptstyle \delta}}}
\newcommand{\Deltabfshat}{{\widehat{\greekbold{\scriptstyle \Delta}}}}
\newcommand{\omegabf}{\greekbold{\omega}}
\newcommand{\Omegabf}{\greekbold{\Omega}}
\newcommand{\Omegabft}{\widetilde{\greekbold{\Omega}}}
\newcommand{\Omegabfs}{{\greekbold{\scriptstyle \Omega}}}
\newcommand{\Omegabfstd}{{\tilde{\greekbold{\scriptstyle \Omegabf}}}}
\newcommand{\Omegabfsbar}{{\bar{\greekbold{\scriptstyle{\Omegabf}}}}}
\newcommand{\Omegabftd}{\widetilde{\Omegabf}}
\newcommand{\Omegabfbar}{\bar{\Omegabf}}
\newcommand{\Omegabfhat}{\widehat{\greekbold{\Omega}}}
\newcommand{\phibf}{\greekbold{\phi}}
\newcommand{\phibfhat}{\hat{\greekbold{\phi}}}


\newcommand{\Sigmabf}{\greekbold{\Sigma}}
\newcommand{\Sigmabfhat}{\greekbold{\widehat{\Sigma}}}
\newcommand{\Sigmabft}{\greekbold{\widetilde{\Sigma}}}
\newcommand{\Sigmabfhats}{{\greekbold{\scriptstyle \widehat{\Sigma}}}}
\newcommand{\Sigmabfs}{{\greekbold{\scriptstyle \Sigma}}}
\newcommand{\mubf}{\utilde{\greekbold{\mu}}}

\newcommand{\xibf}{{\MakeVec{\xi}}}
\newcommand{\xibfy}{\xibf_{y}}
\newcommand{\xibfs}{{\greekbold{\scriptstyle \xi}}}
\newcommand{\xibfhat}{{\hat{\xibf}}}
\newcommand{\xibfhats}{\hat{\xibfs}}

\newcommand{\xibfhaty}{{\hat{\xibf}_{y}}}
\newcommand{\txibf}{\tilde{\greekbold{\xi}}}
\newcommand{\txibfs}{\tilde{\greekbold{\scriptstyle \xi}}}
\newcommand{\Phibf}{\greekbold{\Phi}}
\newcommand{\Phibfhat}{\widehat{\Phibf}}
\newcommand{\Phibfs}{\greekbold{\scriptstyle \Phi}}
\newcommand{\Phibfshat}{\hat{\Phibfs}}
\newcommand{\thetabf}{\greekbold{\theta}}
\newcommand{\varepsilonbf}{\greekbold{\varepsilon}}

\newcommand{\zetabf}{\greekbold{\zeta}}
\newcommand{\tzetabf}{\tilde{\greekbold{\zeta}}}
\newcommand{\zetabfhat}{{\hat{\zetabf}}}
\newcommand{\zetabfs}{{\greekbold{\scriptstyle \zeta}}}
\newcommand{\zetabfhats}{\hat{\zetabfs}}
\newcommand{\nubf}{\greekbold{\nu}}
\newcommand{\nubfhat}{{\hat{\nubf}}}

\newcommand{\lambdahat}{\hat{\lambda}}
\newcommand{\ic}{(i)}

\newcommand{\fa}[1]{2{#1}}
\newcommand{\fb}[1]{1{#1}}
\newcommand{\Si}[1]{\Gammabf_{#1}\Omegabf_{#1}\Gammabf_{#1}^{T}}
\newcommand{\Sinv}[1]{\Gammabf_{#1}\Omegabf_{#1}^{-1}\Gammabf_{#1}^{T}}


%subspace notation
\newcommand{\syx}{\mathcal{S}_{Y|\X}}
\newcommand{\syz}{\mathcal{S}_{Y|\Z}}
\newcommand{\spc}{{\mathcal S}}
\newcommand{\spchat}{\widehat{\mathcal S}}
\newcommand{\dist}{{\mathcal D}}
\newcommand{\Mhat}{\widehat{{\mathbf M}}}
\newcommand{\Mhatsir}{\widehat{\mathbf M}_{\mathrm{\scriptscriptstyle{SIR}}}}
\newcommand{\Msir}{\mathbf{M}_{\mathrm{\scriptscriptstyle{SIR}}}}
\newcommand{\Msave}{\mathbf{M}_{\mathrm{\scriptscriptstyle{SAVE}}}}
\newcommand{\Mhatsave}{\widehat{\mathbf M}_{\mathrm{\scriptscriptstyle{SAVE}}}}
\newcommand{\ospc}{{\mathcal O}}
\newcommand{\hspc}{{\mathcal H}}
\newcommand{\gspc}{{\mathcal G}}
\newcommand{\mspc}{{\mathcal M}}
\newcommand{\ols}{\mathrm{ols}}
\newcommand{\pfc}{\mathrm{pfc}}
\newcommand{\mse}{\mathrm{MSE}}
\newcommand{\bspc}{\mathcal B}
\newcommand{\espc}{{\cal E}}
\newcommand{\vspc}{{\mathcal V}}
\newcommand{\iseb}{{\cal IE}_{\Sigmabfs}(\bspc)}
\newcommand{\seb}{{\cal E}_{\Sigmabfs}(\bspc)}
\newcommand{\sebhat}{\widehat{{\cal E}}_{\Sigmabfs}(\bspc)}
\newcommand{\sebp}{{\cal E}_{\Sigmabfs}(\bspc_{1})}
\newcommand{\indep}{\;\, \rule[0em]{.03em}{.67em} \hspace{-.25em}
\rule[0em]{.65em}{.03em} \hspace{-.25em}
\rule[0em]{.03em}{.67em}\;\,}
\newcommand{\iespc}{{\cal IE}}
\newcommand{\isebp}{{\cal IE}_{\Sigmabfs}^{\perp}(\bspc)}
\newcommand{\isebjhat}[1]{\widehat{{\cal IE}}_{\Sigmabfs}(\bspc)}
%\newcommand{\ZeroF}{{\bf 0}}
\newcommand{\Onebf}{{\utilde{\bf 1}}}
%\newcommand{\Onebf}{ {\underaccent{\sim}{\bf 1}}}
%\newcommand{\Ll}{\underaccent{\sim}{ l}}
%\newcommand{\ZeroF}{ {\underaccent{\sim}{\bf 0}}}
\newcommand{\ZeroF}{{\utilde{\bf 0}}}
\newcommand{\ProjOne}[1]{\frac{\Onebf_{#1} \Onebf_{#1}^T}{#1} }
\newcommand{\ProjOneK}{\frac{\Onebf_K \Onebf_K^T}{K} }

%\newtheorem{prop}{Proposition}
%\newtheorem{lemma}{Lemma}
%\newtheorem{Lemma}{Lemma}
%\newtheorem{proof}{Proof}



\newcommand{\bI}{ { \bf I }}
\newcommand{\bX}{ { \bf X }}
\newcommand{\bx}{ {\utilde{ \bf x }}}
\newcommand{\bY}{\utilde { \bf Y }}
\newcommand{\by}{ {\utilde{ \bf y} }}
\newcommand{\bZ}{ { \bf Z }}
\newcommand{\bz}{ {\utilde{ \bf z }}}
\newcommand{\bzero}{ {\utilde{ \bf 0 }}}
\newcommand{\EE}{\text{E}}
\newcommand{\Var}{\text{Var}}
\newcommand{\bbeta}{\utilde{\boldsymbol \beta}}
\newcommand{\bOmega}{\mbox{\boldmath{$\Omega$}}}
\newcommand{\bpsi}{\mbox{\boldmath{$\psi$}}}
\newcommand{\btheta}{\utilde{\boldsymbol \theta}}
\newcommand{\sC}{ {\cal C} }
\newcommand{\sM}{ {\cal M} }
\newcommand{\sX}{ {\cal X} }
\newcommand{\sY}{ {\cal Y} }
\newcommand{\sZ}{ {\cal Z} }
\newcommand{\ee}[1]{\mathrm{e}^{ #1 }}
\newcommand{\pr}{\text{pr}}
\newcommand{\RE}{\mathbb{R}}
\newcommand{\bigqm}[1][1]{\text{\larger[#1]{\textbf{?}}}}

\newcommand{\vsa}{\vspace{.05 in}}
\newcommand{\vsb}{\vspace{2 em}}
\newcommand{\vsc}{\vspace{1 em}}
\usepackage{color,soul}


\newcommand*\bigcdot{\mathpalette\bigcdot@{.7}}
%\newcommand*\bigcdot@[2]{\mathbin{\vcenter{\hbox{\scalebox{#2}{$\m@th#1\bullet$}}}}}
\makeatother
 
 \newcommand{\RowVecSymbol}[2]{  \left(\begin{array}{c}\rvert\\{#1}_{#2,\bigcdot}\\\rvert\end{array}\right)  }
 \newcommand{\ColumnVecSymbol}[2]{  \left(\begin{array}{c}\rvert\\{#1}_{\bigcdot,#2}\\\rvert\end{array}\right)  }
 \newcommand{\ColumnVecSymbolNoBracket}[2]{  \begin{array}{c}\rvert\\{#1}_{\bigcdot,#2}\\\rvert\end{array} }
  \newcommand{\ColumnVecAll}[3]{  \left(\begin{array}{c} {#1}_{1,#2}\\\vdots\\{#1}_{#3,#2}\end{array}\right)  }
  \newcommand{\ColumnVecAllNoBracket}[3]{  \begin{array}{c} {#1}_{1,#2}\\\vdots\\{#1}_{#3,#2}\end{array}  }
  \newcommand{\RowVecAll}[3]{  \left(\begin{array}{c} {#1}_{#2,1}\\\vdots\\{#1}_{#2,#3}\end{array}\right)  }
  \newcommand{\Vector}[2]{  \left(\begin{array}{c}{#1_{}}\\{}\\{} \end{array}\right)  }
% 
% 
%\newenvironment{definition}[2][Definition]{\begin{trivlist}
%\item[\hskip \labelsep {\bfseries #1}\hskip \labelsep {\bfseries #2.}]}{\end{trivlist}}
%

%\newenvironment{theorem}[2][Theorem]{\begin{trivlist}
%\item[\hskip \labelsep {\bfseries #1}\hskip \labelsep {\bfseries #2.}]}{\end{trivlist}}
%\newenvironment{lemma}[2][Lemma]{\begin{trivlist}
%\item[\hskip \labelsep {\bfseries #1}\hskip \labelsep {\bfseries #2.}]}{\end{trivlist}}
%\newenvironment{exercise}[2][Exercise]{\begin{trivlist}
%\item[\hskip \labelsep {\bfseries #1}\hskip \labelsep {\bfseries #2.}]}{\end{trivlist}}

\newenvironment{reflection}[2][Reflection]{\begin{trivlist}
\item[\hskip \labelsep {\bfseries #1}\hskip \labelsep {\bfseries #2.}]}{\end{trivlist}}
%\newenvironment{proposition}[2][Proposition]{\begin{trivlist}
%\item[\hskip \labelsep {\bfseries #1}\hskip \labelsep {\bfseries #2.}]}{\end{trivlist}}
%
%\newenvironment{corollary}[2][Corollary]{\begin{trivlist}
%\item[\hskip \labelsep {\bfseries #1}\hskip \labelsep {\bfseries #2.}]}{\end{trivlist}}
%\newcommand{\TextInBox}[2]{\fbox{\begin{minipage}{#1} #2 \end{minipage}}}
  \newcommand{\MatrSpace}[2]{\R^{{#1}\times {#2}}}

\newcommand{\IfAndOnlyIfArrow}{\stackrel{\text{  if and only if}}{\Longleftrightarrow}}

\newcommand{\IffArrow}{\IfAndOnlyIfArrow}

\newcommand{\support}{\mathcal{S}}



\newcommand{\TwoColFunction}[2]{
\left\{
\begin{array}{ll}
#1 & \text{ if } #2\\
0 & \text{ otherwise. }
\end{array}
\right.
}
\newcommand{\HLTY}[1]{\HLTEQ[yellow]{#1}}


\newcommand \rbind[1]{%
    \saveexpandmode\expandarg
    \StrSubstitute{\noexpand#1}{,}{&}[\fooo]%
    %\StrSubstitute{\fooo}{,}{&}[\fooo]%
    \StrSubstitute{\fooo}{;}{\noexpand\\}[\fooo]%
    \StrSubstitute{\fooo}{:}{\noexpand\\}[\fooo]%
    \restoreexpandmode
   \left[ \begin{matrix}\fooo\end{matrix}\right]
    }
    
    
    
   \newcommand \ColVec[1]{%
    \saveexpandmode\expandarg
    \StrSubstitute{\noexpand#1}{,}{\noexpand\\}[\fooo]%
    %\StrSubstitute{\fooo}{,}{&}[\fooo]%
    \StrSubstitute{\fooo}{;}{\noexpand\\}[\fooo]%
    \StrSubstitute{\fooo}{:}{\noexpand\\}[\fooo]%
    \restoreexpandmode
   \left[ \begin{matrix}\fooo\end{matrix}\right]
    }
     \newcommand \RowVec[1]{%
    \saveexpandmode\expandarg
    \StrSubstitute{\noexpand#1}{,}{&}[\fooo]%
    %\StrSubstitute{\fooo}{,}{&}[\fooo]%
    \StrSubstitute{\fooo}{;}{&}[\fooo]%
    \StrSubstitute{\fooo}{:}{&}[\fooo]%
    \restoreexpandmode
   \left[ \begin{matrix}\fooo\end{matrix}\right]
    }



  \newcommand \Row[1]{%
    \saveexpandmode\expandarg
    \StrSubstitute{\noexpand#1}{,}{&}[\fooo]%
    %\StrSubstitute{\fooo}{,}{&}[\fooo]%
    \StrSubstitute{\fooo}{;}{&}[\fooo]%
    \StrSubstitute{\fooo}{:}{&}[\fooo]%
    \restoreexpandmode
    \begin{matrix}\fooo\end{matrix}
    }
        
    
    
    
    \newcommand \Col[1]{%
    \saveexpandmode\expandarg
    \StrSubstitute{\noexpand#1}{,}{\noexpand\\}[\fooo]%
    %\StrSubstitute{\fooo}{,}{&}[\fooo]%
    \StrSubstitute{\fooo}{;}{\noexpand\\}[\fooo]%
    \StrSubstitute{\fooo}{:}{\noexpand\\}[\fooo]%
    \restoreexpandmode
    \begin{matrix}\fooo\end{matrix}
    }

%%%%%%%%%%%%%%%%%%%%% Experimental %%%%%%%%%%%%%%%%%


\ExplSyntaxOn
\DeclareExpandableDocumentCommand{\replicate}{O{}mm}
 {
  \int_compare:nT { #2 > 0 }
   {
    {#3}\prg_replicate:nn {#2 - 1} { #1#3 }
   }
 }
\ExplSyntaxOff


\ExplSyntaxOn
\DeclareExpandableDocumentCommand{\repdiag}{O{}mm}
 {
  \int_compare:nT { #2 > 0 }
   {
    {\prg_replicate:nn {#2}{#3#1}}{#3}
   }
 }
\ExplSyntaxOff


\newcommand \StrRowDiag[1]{%
    \saveexpandmode\expandarg
    \StrSubstitute{\noexpand#1}{,}{&}[\fooo]%
    %\StrSubstitute{\fooo}{,}{&}[\fooo]%
    \StrSubstitute{\fooo}{;}{&}[\fooo]%
    \StrSubstitute{\fooo}{:}{&}[\fooo]%
    \StrCount{\fooo}{&}[\countfooo]
    \restoreexpandmode
    \repdiag[0]{\countfooo+1}{{,}}
   %\left[ \begin{matrix}\fooo\end{matrix}\right]
    }


\newcommand \DiagStrOne[2]{%
    \saveexpandmode\expandarg
    \StrSubstitute{\noexpand#1}{,}{\noexpand#2}[\fooo]%
    \restoreexpandmode
   %\left[ \begin{matrix}\fooo\end{matrix}\right]
   \fooo
    }
    
    \newcommand \DiagStr[1]{%
    \DiagStrOne{#1}{{\StrRowDiag{#1}}}
    }


%$\rbind{\replicate[,]{10}{\Col{\replicate[;]{7}{0}}}}$

%$\Col{1,2,3}$
%$\ColVec{\replicate[;]{5}{B}}$
%$ \StrRowDiag{1,2} $

%$\DiagStr{1,2,3}$

%\repdiag[-]{3}{A}
\ExplSyntaxOn
\NewDocumentCommand{\Split}{ m m o }
 {
  \tarass_split:nn { #1 } { #2 }
  \IfNoValueTF { #3 } { \tl_use:N } { \tl_set_eq:NN #3 } \l_tarass_string_tl
 }

\tl_new:N \l_tarass_string_tl

\cs_new_protected:Npn \tarass_split:nn #1 #2
 {
  \tl_set:Nn \l_tarass_string_tl { #2 }
  % we need to start from the end, so we reverse the string
  \tl_reverse:N \l_tarass_string_tl
  % add a comma after any group of #1 tokens
  \regex_replace_all:nnN { (.{#1}) } { \1\, } \l_tarass_string_tl
  % if the length of the string is a multiple of #1 a trailing comma is added
  % so we remove it
  \regex_replace_once:nnN { \,\Z } { } \l_tarass_string_tl
  % reverse back
  \tl_reverse:N \l_tarass_string_tl
 }
\ExplSyntaxOff

%%%%%%%%%%%%%%%%%%%%%%%%%%%%%%%%

\newcommand{\ShowRowMatrix}[3]{ \left[ {\begin{array}{ccc}
  \line(1,0){22} &{#1} &  \line(1,0){22} \\
     & \vdots& \\
  \line(1,0){22}  &{#2}& \line(1,0){22} \\
   &  \vdots & \\
    \line(1,0){22} &{#3}& \line(1,0){22}  \\
    \end{array}
   } \right]}
 


\newcommand{\ShowColMatrix}[3]{ \left[ {\begin{array}{ccccc}
  \line(0,1){25} & &\line(0,1){25} &  &  \line(0,1){25} \\
  {#1}  & \ldots & {#2} &\ldots   &{#3} \\
 \line(0,1){25} &  & \line(0,1){25}  &  &  \line(0,1){25} \\
    \end{array}
   } \right]}
   
   
   
   
\newcommand{\ShowRowVector}[1]{ \left[ {\begin{array}{ccc}
  \line(1,0){25} &{#1} &  \line(1,0){25} 
    \end{array}
   } \right]}   
   
   
\newcommand{\ShowColVector}[1]{ \left[ {\begin{array}{c}
  \line(0,1){25} \\    {#1} \\   \line(0,1){25}     \end{array}  } \right]}
  
\newcommand{\ColVector}[3]{ \left[ {\begin{array}{c}
  {#1}\\ \vdots \\    {#2}\\ \vdots\\{#3}  \end{array}  } \right]}
  
  
  
  
  
\newcommand{\EqSetThree}[3]{ \left\{ {\begin{array}{c}
  {#1}\\ \vdots \\    {#2}\\ \vdots\\{#3}  \end{array}  } \right.}  
  



\newcommand{\MatrixTypeA}[3]{ \left[ {\begin{array}{ccc}
 {#1}_{1,1} & \cdots & {#1}_{1,{#3}}   \\
  {#1}_{2,1} & \cdots & {#1}_{2,{#3}}   \\
    \vdots  & \ddots& \vdots  \\
     {#1}_{{#2},1} & \cdots & {#1}_{{#2},{#3}}   \\
    \end{array}
   } \right]}
 
\newcommand{\MatrixTypeAKronecker}[4]{ \left[ {\begin{array}{ccc}
 {#1}_{11}{#4} & \cdots & {#1}_{1{#3}}{#4}   \\
  {#1}_{21} {#4} & \cdots & {#1}_{2{#3}} {#4}   \\
    \vdots  & \ddots& \vdots  \\
     {#1}_{{#2}1} {#4} & \cdots & {#1}_{{#2}{#3}} {#4}   \\
    \end{array}
   } \right]}
 



\newcommand{\ShowIMat}{ {\begin{array}{cccc}
 1&  &  &    \\
  & 1 &  &  \\
    &  & \ddots &    \\
     & & & 1   \\
    \end{array}
   } }
 
\newcommand{\ShowVecOne}{
\begin{array}{c}
 1\\ 1 \\    1  
\end{array}
}

 
\newcommand{\ShowUnitVecOne}{
\begin{array}{c}
 1\\ 0 \\   0  
\end{array}
}


\newcommand{\ShowUnitVecTwo}{
\begin{array}{c}
 0\\ 1 \\   0  
\end{array}
}


\newcommand{\ShowUnitVecThree}{
\begin{array}{c}
 0\\ 0\\   1  
\end{array}
}

\newcommand{\ShowZeroThree}{
\begin{array}{c}
 0\\ 0\\   0 
\end{array}
}


\newcommand{\TwoBlockMatrix}[2]{\left[  {\begin{array}{c;{2pt/2pt}c}
   {#1} &  {#2}
   \end{array} }\right]}
   
   \newcommand{\TwoBlockMatrixH}[2]{\left[  {\begin{array}{c}
   {#1} \\
   \hdashline[2pt/2pt]
    {#2}
   \end{array} }\right]}
   
   \newcommand{\TwoBlockH}[2]{ {\begin{array}{c}
   {#1} \\
   \hdashline[2pt/2pt]
    {#2}
   \end{array} }}
   
   
\newcommand{\TwoBlock}[2]{ {\begin{array}{c;{2pt/2pt}c}
   {#1} &  {#2}
   \end{array} }}
   

      
   
   
   
 \newcommand{\ThreeBlockColVec}[3]{
   \left[ {\begin{array}{c}
  #1\\
  \hdashline[2pt/2pt]\\
   \vdots\\
  \hdashline[2pt/2pt]\\
  #2\\
  \hdashline[2pt/2pt]\\
   \vdots\\
  \hdashline[2pt/2pt]\\
   #3\\
    \end{array}
   } \right]
   }



\NewDocumentCommand{\ColDyn}{>{\SplitList{;}}m}
   {
     \left[\begin{array}{c}
       \ProcessList{#1}{ \inserColtitem }
     \end{array}\right]
   }
   \newcommand \inserColtitem[1]{ #1 \\}


\makeatletter
\newcommand{\ColDynAlt}[2][r]{%
  \gdef\@VORNE{1}
  \left[\hskip-\arraycolsep%
    \begin{array}{#1}\vekSp@lten{#2}\end{array}%
  \hskip-\arraycolsep\right]}

\def\vekSp@lten#1{\xvekSp@lten#1;vekL@stLine;}
\def\vekL@stLine{vekL@stLine}
\def\xvekSp@lten#1;{\def\temp{#1}%
  \ifx\temp\vekL@stLine
  \else
    \ifnum\@VORNE=1\gdef\@VORNE{0}
    \else\@arraycr\fi%
    #1%
    \expandafter\xvekSp@lten
  \fi}
\makeatother


\NewDocumentCommand{\eVec}{m O{}}{\MakeVec{e}_{#1, (#2)}}

\NewDocumentCommand{\Ones}{O{3}}{\Col{\replicate[,]{#1}{1}}}
\NewDocumentCommand{\Zeros}{O{3}}{\Col{\replicate[,]{#1}{0}}}
\input{../MacroDefs/Altstructure} 
 
\usepackage{tfrupee}
\usepackage{fdsymbol}
\usepackage{latexcolors}
\usepackage{mathrsfs}
%\NewDocumentCommand{\support}{O{ }}{{\mathcal{S}}_{_{#1}}}
%\NewDocumentCommand{\support}{O{ }}{{\mathbb{S}}_{_{#1}}}
%\newcommand{\SampleS}{\mathcal{S}}
\newcommand{\SampleS}{\mathscr{S}}
\newcommand{\Not}[1]{\overline{#1}}
\newcommand{\Center}[1]{{\begin{center}#1 \vspace{-.2in}
\end{center}}}

 \begin{document}
 
\title{}%replace X with the appropriate number
\author{} %if necessary, replace with your course title
%\date{ $ 26^{th} $ August, 2019}
\date{}
% \maketitle\vspace{-.in}
 \noindent
% \TextInBox{6 in }{Name: \vspace{.22 in}}\\
%\; \TextInBox{6.05 in }{ Enrolled As: \MCOption[1.2in]{PHD Student} \MCOption[1in]{RA}\MCOption[1 in]{TA}}\\
\TextInBox{6 in }{ 

  }\\
 \vspace{.4in}
 
 
 

%
%\DefBoxOne{}{
%\begin{center}\color{black}
%Part I:  Short answer type questions.
%\end{center}
%}


%\DefBoxOne{Notations}{
%$$ \Onebf_k=\left[ \begin{array}{c}1\\1\\\vdots \\1\end{array}\right]_{k\times 1} \text{ , }  J_{k}=\left[ \begin{array}{cccc}1&1 & \cdots & 1\\1& 1& \cdots&  1\\\vdots & \vdots & \ddots  &\vdots \\1& 1 &\cdots & 1\end{array}\right]_{k\times k} \text{ and } I_{k}=\left[ \begin{array}{cccc}1&0 & \cdots & 0\\0& 1& \cdots&  0\\\vdots & \vdots & \ddots  &\vdots \\0& 0 &\cdots & 1\end{array}\right]_{k\times k} \text{ for } k\in \Z_{+}.   $$
%
%}


\vspace{-.2in}
\DefBoxOne{}{
\begin{center}\color{black}
\Large Exam Assistance Note
\end{center}
}


\vspace{.2in}





%%
%%\qBrd[6.2in]{applegreen!30}{
%%Let $n$ be a {\bf non-negative integer,} then the {\bf factorial of n}, denoted as $\HLTY{n!}$ is defined to be
%%\begin{enumerate}
%%\qBrd[.75in]{amethyst!30}{\item[\sqBullet{amethyst}] $0!=\HLTY{1}$} \hspace{.4in}
%%\qBrd[.75in]{amethyst!30}{\item[\sqBullet{amethyst}]  $1!=\HLTY{1}$}\hspace{.4in}
%%\qBrd[2.3in]{amethyst!30}{\item[\sqBullet{amethyst}]  $n!= \HLTY{n\times(n-1)\times \ldots \times 1}$ for $n\geq 2$}
%%\end{enumerate} 
%%}\\
%%
%%\qBrd[6.2in]{babyblue!30}{
%%Let $n, r$ be two {\bf non-negative integes,} such that $r\leq n$,  then the {\bf n choose r}, denoted by $\HLTY{{n\choose r}}$, is defined to be $${n \choose r}:= \frac{n!}{\left(r!\right) \times \left((n-r)!\right)}$$
%%}\\
%%
%%
%%
%%\qBrd[6.2in]{amethyst!20}{
%%For any real number $x\in \R$, the exponential series $\HLTY{e^x}$  (or sometimes denoted as $\HLTY{\exp(x)}$) is defined as, 
%%$$\HLTY{\displaystyle e^{x}}=\HLTEQ[white]{\displaystyle \sum_{n=0}^{\infty}\frac{x^n}{n!}} = \HLTY{\displaystyle 1+\frac{x}{1!}+\frac{x^2}{2!}+ \frac{x^3}{3!}+\cdots}  $$
%%}\\
%%
%%\qBrd[6.2in]{babyblue!30}{
%%Let $x\in \R$ be any real number,  and $n\in\mathbb{Z}_{+}$ be any positive integer, then   $$ \HLTEQ[white]{\displaystyle (1+x)^n}=  \HLTY{\displaystyle\sum_{i=0}^{n}   {n \choose i  }  x^{i}}$$
%%$$\ \HLTEQ[white]{\displaystyle (1+x)^n} = \HLTY{\displaystyle 1+{n\choose 1}x+{n\choose 2}x^2+  \cdots+ {n\choose {n-1}}x^{n-1}+{n\choose n}x^n}  $$ 
%%}\\
%%
%%\qBrd[6.2in]{amethyst!20}{
%%Let $p\in \R$ be such that $|p|<1$,  then   $ \HLTY{\displaystyle \sum_{i=0}^{\infty}p^{i}}=  \HLTEQ[white]{\displaystyle 1+p+p^2+p^3+  \cdots}   =\HLTY{\displaystyle \frac{1}{1-p}}.$
%%}\\
%%
%%
%%%\newpage
%%\qBrd[6.5in]{babyblue!20}{
%%{\bf (Ordered, without replacement)}{ Let $r$, and $n$ be two positive integers such that $r\leq n$.  An ordered arrangement of $r$ distinct objects is called a permutation.  The number of ways of ordering $n$ distinct objects taken $r$  at a time, denoted by the symbol $^nP_r$,  is given as   $$\HLTEQ[applegreen!50]{ \HLTW{\displaystyle  ^nP_r}=  \HLTEQ[white]{ \displaystyle n(n - 1)(n- 2)(n - r + 1) }= \HLTW{ \displaystyle \frac{n!}{(n - r )!}  }   }.$$
%%}}\\
%%
%%
%%
%%\qBrd[6.5in]{amethyst!20}{
%%Let $n\geq r $ be two non-negative integers.  The number of different ways to select (/choose) $\HLTY{r}$ distinct objects from a list of $\HLTY{n}$ distinct (non-identical) objects  is given as( {\tiny or $^nC_r$)},
%% \vspace{-.1in} $$\HLTEQ[babyblue!70]{\HLTW{ \displaystyle {n \choose r }}= \HLTW{ \displaystyle\frac{n!}{r! (n-r)!}}} $$
%%}\\
%%
%%
%%\qBrd[6.5in]{babyblue!40}{
%%The number of ways of partitioning $n$ distinct objects into $k$ distinct groups
%%containing $n_1$, $n_2$, . . . , $n_k$ objects, respectively, where each object appears exactly in  one group and $\HLTW{\sum_{i=1}^{k}n_i=n}$, is  
%% $$\HLTEQ[lime!40]{\large \displaystyle { n\choose {n_1, n_2, \ldots , n_k}}:= \frac{n!}{(n_1!)(n_2!)\ldots (n_k!)}}$$
%%}\\
%%
%%
%%
%%
%%\qBrd[6.5in]{amethyst!40}{
%%Number of ways $n$ indistinguishable/identical objects can be organized into r different (ordered) groups is $$\HLTW{\displaystyle \frac{(n+r-1)!}{n! (r-1)!}={{n+r-1} \choose {n}}}.$$
%%}\\
%%
%%\qBrd[6.6in]{olive!40}{
%%De-Arrangement probability with N distinct objects: 
%%$\HLTW{ \HLTY{1- \frac{1}{1!}}+  \frac{1}{2!}-\frac{1}{3!}+\frac{1}{4!}+\cdots + (-1)^{N}\frac{1}{N!}}$.
%%}\\
%%





%\newpage 
\qbx[6.8in]{antiquefuchsia!60}{
\Center{
\qBrd[3.2in]{amethyst!40}{\Large \bf Properties of Probability }}
\vspace{.1in}
Let $\HLTY{(\SampleS , P)}$ be a sample space along with the Probability measure. Let $\HLTW{A}, \HLTW{B}$ be two events.  Then, \\
\begin{enumerate}
\qBrd[3.8in]{applegreen!40}{  \item[{\sqBullet{olive}}] $\HLTW{P(\emptyset)=0}$ where $\HLTW{\emptyset}$ denotes the Empty set (Null set). }\\
\vspace{.05in}
\qBrd[2in]{apricot!40}{ \item[{\sqBullet{apricot}}]  $\HLTW{P(A)\leq 1}$. }\\
\vspace{.05in}
\qBrd[2.8in]{lime!40}{ \item[{\sqBullet{applegreen}}] If $\HLTW{
A\subseteq B}$ then $\HLTW{P(A)\leq P(B)} $.}\\
\vspace{.05in}
\qBrd[4.6in]{brightlavender!40}{\item[{\sqBullet{amethyst}}] $\HLTW{P\left(\Not{A}\right)=1-P(A)}$, where  $\HLTW{\Not{A}}$ denotes the complementary event to $A$.}\\
\vspace{.05in}
\qBrd[4in]{cadmiumorange!40}{\item[{\sqBullet{cadmiumorange}}]  $ \HLTW{P(A\cup B)= P(A)+P(B)-P(A\cap B) }$.}
\end{enumerate}
}\\

%
%
%\qbx[6.5in]{antiquefuchsia!30}{
%\begin{center}
%	\qBrd[4in]{blue!40}{
%\begin{itemize}
%\item[]\qBrd[3in]{babyblue!40}{ $A \cup\left( B\cup C\right)= \left(A \cup B\right)\cup C= A \cup B\cup C $}
%\item[]  \qBrd[3in]{babyblue!70}{$A \cap\left( B\cap C\right)= \left(A \cap B\right)\cap C= A \cap B\cap C $}
%\end{itemize}	
%}
%\vspace{.05in}
%	\qBrd[3in]{amethyst!70}{
%	\HLTY{\text{\it \small Distributive laws of Union \& Intersection}}
%\begin{itemize}
%\item[] \qBrd[2.5in]{blush!40}{$A \cap\left( B\cup C\right)= \left(A \cap B\right)\cup \left(  A \cap C \right)$}
%\item[] \qBrd[2.5in]{blush!60}{ $A \cup\left( B\cap C\right)= \left(A \cup B\right)\cap \left(  A \cup C \right)$}
%\end{itemize}	
%}
%\vspace{.1in}
%	\qBrd[3in]{asparagus!80}{
%	\HLTY{\text{\small \it DeMorgan's laws}}
%\begin{itemize}
%\item[] \qBrd[2.1in]{applegreen!40}{$\Not{\left( A\cap B\right)}= \left(\Not{A} \cup \Not{B}\right)$}
%\item[] \qBrd[2.1in]{applegreen!60}{$\Not{\left( A\cup B\right)}= \left(\Not{A} \cap \Not{B}\right)$}
%\end{itemize}	
%}	
%\end{center}
%}\\
%
%





%
%\qBrd[6.6in]{ceil!50}{
%Let $A_1, A_2, A_3$ are three events.  Then 
%\begin{eqnarray*}
% P\left(A_1\cup A_2 \cup A_3\right) 
% & =&  \left\{ \HLTEQ{P(A_1)+P(A_2)+P(A_3)}\right\}  \nonumber\\
% & &-  \left\{\HLTEQ[lightBlueOne]{ P(A_1\cap A_2) + P(A_1\cap A_3) + P(A_2\cap A_3) }  \right\}+ \left\{ \HLTEQ{P(A_1\cap A_2 \cap A_3)}     \right\}  
%\end{eqnarray*}  
%\vspace{-.15in}
%}\\





%
%
%
%\begin{center}
%\qBrd[5in]{babyblueeyes!50}{
%\begin{center}
%\vspace{-.2in}
%\qBrd[3.3in]{blue!30}{\Large Structure of Two $\times $ Two Table  }\\
%\end{center}
%\begin{center}
%\begin{tabular}{|l||l|l||l|l|}
%\hline 
%     &  & &   \vspace{-.1in }  \\
%      & $A$ & $\Not{A}$ & Marginal of $B$   \vspace{.1in }  \\ \hline\hline
%      &  & &   \vspace{-.1in }  \\
%$B$ &  $\HLTEQ[lime!30]{P\left(A\cap B\right)}$   &    $\HLTY{P\left(\Not{A}\cap B\right)}$ &  $\HLTW{P(B)}$    \vspace{.1in }  \\ \hline \vspace{-.1in }
%      &  & &    \\
%$\Not{B}$ &   $\HLTY{P\left(A\cap \Not{B}\right)}$  &  $\HLTEQ[amethyst!40]{P\left(\Not{A}\cap \Not{B}\right)}$   &   $\HLTW{P\left(\Not{B}\right)}$    \vspace{.1in }  \\ \hline\hline \vspace{-.1in }
%      &  & &    \\
%Marginal of $A$ & $\HLTW{P(A)}$     & $\HLTW{P\left(\Not{A}\right)}$     &$ \HLTW{\text{Total Probability}= \HLTEQ[teal!40]{1}} $   \\ \hline
%\end{tabular}
%\end{center}
%}
%\end{center}

\qBrd[6.6in]{applegreen!30}{
Let $E$,  and  $F$ are two events such that $P\left(\HLTY{F} \right)>0$, then the conditional probability of {\bf $\HLTEQ[amethyst!30]{E}$ given $\HLTY{F}$} is defined to be, 
\begin{center}
\vspace{-.25in}
\qBrd[2.55in]{teal!30}{ $$ \vspace{-.01in}
\HLTW{\displaystyle P\left(\HLTEQ[amethyst!30]{E}\mid \HLTY{ F} \right):= \frac{P\left(\HLTEQ[amethyst!30]{E}\cap \HLTY{F}\right)}{P( \HLTY{F})}.} \vspace{-.01in}$$ }
\vspace{-.1in}
\qBrd[5.6in]{amethyst!20}{
Let $E$ and $F$ are two events, then 
$\HLTEQ[babyblue!70]{ \HLTY{P(E\cap F)}:=\HLTW{P(E\mid F)}\times \HLTW{P(F)}.}$
}\end{center}
}\\
\vspace{-.1in}

\vspace{.5in}

%\qBrd[6.5in]{teal!40}{
%\begin{center}
%\vspace{-.2in}
%\qBrd[3.3in]{yellow-green!50}{\Large Law of Total Probability   }\\
%\end{center}
%Let E and F be two events, then 
%\qBrd[4.1in]{amethyst!60}{$\Large \HLTEQ[white!20]{ P(E)} = {\HLTEQ[white!70]{P(E\mid F)P(F) } +  \HLTW{P\left(E\mid  \Not{F}\right)P\left(\Not{F}\right)}} $}
%}\\
%
%\qBrd[6.5in]{turquoisegreen!90}{\begin{center}
%\vspace{-.2in}
%\qBrd[3.4in]{yellow-green!50}{\Large Law of  Total Probability (General):} \\
%\end{center}
%Let E be an event.  Assuming that the collection of sets   $\{F_1, F_2,\ldots,  F_k\}$  forms a partition of $\SampleS$,  we have
%\vspace{-.15in}
%\begin{center}
%%\qBrd[2in]{applegreen!40}{$ E= (E\cap F) \cup (E\cap \Not{F}) $}\\
%\qBrd[3in]{green!40}{\Large $\displaystyle \HLTW{P(E)}=\HLTW{\sum_{j=1}^{k}  P(E\mid F_j)P(F_j)  }$}
%\vspace{-.1in}
%\end{center}
%}\\
%\vspace{.15in }

%\qBrd[6.5in]{amethyst!40}{
%\begin{center}
%\vspace{-.2in}
%\qBrd[2.6in]{amber!70}{\Large Bayes' Theorem (General) }\\
%\end{center}
%\vspace{-.15in}
%Let $\HLTW{F_1, F_2, \ldots,  F_{_K}}$ be a set of mutually exclusive and exhaustive events ( partition of the sample space $\SampleS$).  Suppose now that $E$ be an event such that $P(E)>0$, then 
%$$\HLTW{ \displaystyle P(F_i \mid E)=\HLTY{ \frac{ \displaystyle P(E\mid F_i) P(F_i )}{\displaystyle \sum_{j=1}^{K}P(E\mid F_j) P(F_j) }}}$$
%}\\
%\vspace{.1in}


\qBrd[6.5in]{amethyst!40}{
\begin{center}
\vspace{-.2in}
\qBrd[2.1in]{amber!70}{\Large Bayes' Theorem }\\
\end{center}
\vspace{-.15in}
Let $A, B$ are two events such that $P(A)>0$, and $P(B)>0$, then 
$$\HLTW{ \displaystyle P(B\mid A)=\HLTY{ \frac{ \displaystyle \HLTEQ[green!30]{P(A\mid B) P(B )}}{\displaystyle \HLTEQ[green!30]{P(A\mid B) P(B )}+ \HLTEQ[orange!30]{P\left(A\mid \Not{B}\right) P\left(\Not{B} \right) }     }}}$$
}\\
\vspace{.1in}

\qBrd[6.6in]{teal!30}{
\begin{center}
\vspace{-.2in}
\qBrd[2.6in]{teal!40}{\Large Statistical Independence}\\
\end{center}
\vspace{-.15in}
Two events $E$ and $F$ are said to be  statistically independent if
$\HLTW{\displaystyle P(E\cap F ) = P(E)\times P(F)    }$}\\

\vspace{.1in}
\newcommand{\pmf}{p}
\qBrd[6.6in]{amber!40}{
\begin{center}
\vspace{-.2in}
\qBrd[2.6in]{amber!70}{\Large Characterization of a pmf }\\
\end{center}
\vspace{-.15in}
Let $\HLTW{\pmf(x)}$ is {\bf probability mass function } of a discrete random variable on the support $\support$,  {\bf if and only if} it satisfies the following conditions:
\vspace{-.1in}
\begin{enumerate}
\qBrd[2.3in]{babyblue!40}{\item {\it Positivity: }$ \HLTW{ \displaystyle  \pmf(x)> 0}$ for all $\HLTW{ x\in \support}$}\hspace{.3in}
\qBrd[3.2in]{applegreen!40}{\item {\it Total Probability:} $\HLTW{   \displaystyle  \sum_{\{x\in \support\}} \pmf(x)=1}$. }
\end{enumerate}
}\\

\vspace{.3in}
\newcommand{\cdf}{F}
\qbx[6.5in]{amethyst!30}{
\begin{center}
\vspace{-.35in}
\qBrd[4in]{amethyst!60}{ \Large  {\bf ``CDF''}  of a Discrete Random Variable }
\vspace{-.1in}
\end{center}
Let $X$ be a discrete random variable on the support $\support[X]$ with the corresponding probability mass function
$$\HLTEQ[teal!50]{\HLTW{ \displaystyle P(X=x)= \pmf_{_X}(x)} \text{ for } \HLTW{x\in \support[X]}.}$$
Then for any $a\in \R$, the cumulative distribution function (cdf), denoted by $\cdf_{_X}(\cdot)$ is the following quantity
$$\HLTW{ \displaystyle \cdf_{_X}(\HLTY{a})= P(X\leq \HLTY{a})= \sum_{ \HLTEQ[teal!40]{ \{\HLTW{x}\leq \HLTY{a}: \HLTW{x}\in \support[X]\}}}\pmf_{_X}(\HLTW{x})}$$	
}\\

\vspace{.35in}
\qBrd[6.6in]{amber!35}{
\vspace{-.35in}
\begin{center}
\qBrd[4.6in]{amber!70}{\begin{center} \Large {\Large \bf ``Expected Value''} or {\bf ``Mean''} of a Discrete Random Variable \end{center} }
\end{center}
If $X$ is a random variable with pmf $\HLTY{\pmf_{_X}(x)} $ on the support $\support[X]$ , then the expected value (the mean) of $X$ denoted by $E(X)$ ( or $\mu_{_X}$)  is given by
$$\HLTEQ[amethyst!30]{ \displaystyle E\left(\HLTW{X}\right)= \sum_{\left\{ \HLTEQ[teal!40]{\HLTW{x}\in \support[X]} \right\}} \HLTW{x}  \; \HLTY{\pmf_{_X}(\HLTW{x})},}$$
assuming the  above summation/series exists /well-defined.  Additionally,  assuming it exists,  for any* funciton $h(x)$, 
$$\HLTEQ[amethyst!30]{ \displaystyle E\left(\HLTW{h(X)}\right)= \sum_{\left\{ \HLTEQ[teal!40]{\HLTW{x}\in \support[X]} \right\}} \HLTW{h(x)}  \; \HLTY{\pmf_{_X}(\HLTW{x})},}$$
}\\
\vspace{.5in}

\qBrd[7.3in]{applegreen!35}{
\vspace{-.35 in }
\begin{center}
\qBrd[4.8in]{applegreen!60}{\begin{center}\Large  {\Large \bf ``Variance \& Standard Deviation (SD) ''} of a Random Variable  \end{center}}
\end{center}
\qBrd[2.2in]{amber!60}{ The variance of $X$, denoted by $\text{Var}(X)$ is deifined as\\ $\HLTW{ \displaystyle \text{Var}(X):=\HLTW{ E(X^2)- \left(\HLTY{E(X)} \right)^2}}$}
\qBrd[2.2in]{amber!60}{ $$\HLTW{ \displaystyle \text{E}(X^2):=\HLTW{ \text{Var}(X)+ \left(\HLTY{E(X)} \right)^2}}$$\vspace{-.1in}}
\qBrd[2in]{amber!60}{ $\HLTW{ \displaystyle \HLTW{\sigma_{_X}}=  \text{SD}(X):=\HLTW{\sqrt{\text{Var}(X)}}}$}
}\\
\vspace{.5in}

\qBrd[7.2in]{antiquefuchsia!50}{
\vspace{-.35in}
\begin{center}
\qBrd[4.9in]{antiquefuchsia!90}{ \begin{center}\Large {\bf \Large ``Moment Generating Function (MGF) ''}  \end{center} }\\
The Moment Generating Function (mgf) of $X$, denoted by $\text{M}_{_X}(t)$ is deifined as 
$${ \HLTW{\displaystyle \text{M}_{_X}(t):= E\left(\HLTW{e^{tX}} \right)}} \text{whenever it exists. }$$
\qBrd[3.4in]{babyblue!30}{ If X is a discrete random variable with p.m.f $ \HLTY{\pmf_{_X}(\HLTW{x})}$ then 
$$\HLTW{M_{_X}(t):= E\left(\HLTW{e^{tX}} \right)= \displaystyle  \sum_{\left\{\HLTEQ[teal!40]{ \displaystyle \HLTW{x}\in \support[X]} \right\}} \HLTW{e^{tx}}  \; \HLTY{\pmf_{_X}(\HLTW{x})}}. \vspace{-.2in}$$\\
\vspace{-.1in}}
\qBrd[3.4in]{apricot!30}{ If X is a continuous random variable with a probability density function $ \HLTY{f_{_X}(\HLTW{x})}$ then 
$$\HLTW{M_{_X}(t):= E\left(\HLTW{e^{tX}} \right)= \displaystyle  \int\limits_{\left\{\HLTEQ[teal!40]{ \displaystyle \HLTW{x}\in \support[X]} \right\}} \HLTW{e^{tx}}  \; \HLTY{f_{_X}(\HLTW{x})}dx }. \vspace{-.2in}$$\\
\vspace{-.1in}}
\qbx[6.9in ]{amethyst!40}{
Let X be a r.v.  with the moment generating function $M_{_X}(t)$, then, assuming existence,  the $r^{\text{th}}$ raw moments (non-centered), for the random variable can be obtained as 
$$ \HLTW{E(X^r)= \frac{d^{{r}} M_{_X}(t) }{dt^{{r}}}\Big\vert_{t=0} } \text{ for  any positive integer} \HLTW{r}.  $$
Specifically, 
$$ \HLTW{E(X)= \frac{d M_{_X}(t) }{dt}\Big\vert_{t=0} }$$
}
\end{center}
}\\




\qbx[7.1in]{apricot!40}{
\vspace{-.35in}
\begin{center}
\qBrd[4.9in]{apricot!90}{ \begin{center}\Large {\bf \Large ``Probability Density Function ''}  \end{center} }
\end{center}
Let $\HLTW{F_{_X}(x)}$ be a cumulative distribution function (CDF) of a continuous random variable,  then the corresponding {\bf  probability density function} or {\bf pdf}, denoted as $\HLTW{f_{_X}(x)}$  is a function that satisfies the following creteria. 
$$\HLTW{ \displaystyle F_{_X}(\HLTY{x})= \int_{-\infty}^{\HLTY{x}} f_{_X}(y) dy }$$
}

\qBrd[7.2in]{babyblue!30}{
{\bf pdf } of a continuous random variable can be obtained by differentiating the corresponding {\bf CDF} 
$$\HLTW{\displaystyle f_{_X}(x)= \frac{d}{dx} F_{_X}(x)}$$}\\
\vspace{.1in}

\qBrd[7.2in]{airforceblue!30}{
\sqBullet{blue} If X is a continuous random variable, then probabilities can be obtained by integrating its pdf over suitable region.  Specifically,  for $\HLTW{a, b\in \R}$, $\HLTW{a<b}$, 
$$\HLTEQ[white]{\displaystyle P\left(\HLTY{a}< X\leq \HLTEQ[amethyst!40]{b}\right)= \HLTEQ[babyblue!35]{F_{_X}\left( \HLTEQ[amethyst!40]{b}\right)-F_{_X}\left(\HLTY{a}\right)}= \int\limits_{\HLTY{a}}^{ \HLTEQ[amethyst!40]{b}} f_{X}(x) dx.}$$\vspace{-.2in}
}\\

\vspace{.1in}
\qBrd[7.1in]{apricot!40}{
\vspace{-.35in}
\begin{center}
\qBrd[4.9in]{apricot!90}{ \begin{center}\Large {\bf \Large ``{Characterization of  Probability Density Function (pdf)}''}  \end{center} }
\end{center}
A function $f(x)$ is is a  probability density function (pdf) for some continuous random variable if  and only if it satisfies the following two conditions:
\begin{enumerate}
 \qBrd[2.5in]{applegreen!50}{\item {\bf Non-negativity:} $f(x)\geq 0 $ for all $x$.}
\qBrd[2.8in]{babyblue!50}{\item {\bf Total Probability: } $\displaystyle \int_{-\infty}^{\infty}f(x)dx= 1$}
%\item $\HLTW{\displaystyle P(a\leq X\leq b)= \int_{a}^{b}f(x)dx}$ for all $a<b$.
\end{enumerate}
}\\
\vspace{.2in}

\qbx[7.1in]{airforceblue!40}{
\vspace{-.35in}
\begin{center}
\qBrd[4.9in]{airforceblue!90}{ \begin{center}\Large {\bf \Large ``Quantiles, and Percentiles''}  \end{center} }
\end{center}
Let $p$ be a number between 0 and 1. The $(100)^{\text{th}}$ percentile of the distribution of a continuous random variable X, we shall denote by c, is that value for which
$$ \HLTW{ \displaystyle F\left(\HLTY{c}\right)= \HLTEQ[amethyst!30]{p}}$$
i.e. $\HLTW{\HLTY{c}= F^{-1}\left(\HLTEQ[amethyst!30]{p}\right)}$.
where $F^{-1}(\cdot)$ is the inverse cumulative distribution function.
}

\qbx[7.2in]{antiquefuchsia!60}{
\vspace{.1in}
\qBrd[2.2in]{amber!40}{\sqBullet{amber}  \qBrd[.8in]{amber!70}{Median}The median of a continuous distribution, denoted by m, is the 50th percentile. So $m$ satisfies\vspace{-.1in}  $$\HLTW{m = F^{-1}(0.5)}\vspace{-.2in}$$}
\qBrd[2.2in]{teal!30}{\sqBullet{teal}  \qBrd[1in]{teal!50}{First Quartile} The first  quartile is defined to be \vspace{-.1in} $$\HLTW{Q_1=  F^{-1}(0.25)}\vspace{-.2in}$$}
\qBrd[2.2in]{applegreen!40}{\sqBullet{applegreen}   \qBrd[1 in]{applegreen!70}{Third Quartile}  The  third quartile is defined to be \vspace{-.1in} $$\HLTW{Q_3=  F^{-1}(0.75)}\vspace{-.2in}$$}
\vspace{.1in}
}


\vspace{.35in}
\qBrd[6.6in]{amber!35}{
\vspace{-.35in}
\begin{center}
\qBrd[4.6in]{amber!70}{\begin{center} \Large {\Large \bf ``Expected Value''} or {\bf ``Mean''} of a Continuous Random Variable \end{center} }
\end{center}
If X is a continuous random variable with pdf $f(x)$ on the support $\support[X]$, then the expected value (the mean) of X denoted by $E(X)$ is given by
$$\HLTEQ[amethyst!30]{ \displaystyle E\left(\HLTW{X}\right)= \int\limits_{\left\{ \HLTEQ[teal!40]{\HLTW{x}\in \support[X]} \right\}} \HLTW{x}  \; \HLTY{f_{_X}(\HLTW{x})}\;dx,}$$
assuming the  above summation/series exists /well-defined.  Additionally,  assuming it exists,  for any* funciton $h(x)$, 
$$\HLTEQ[amethyst!30]{ \displaystyle E\left(\HLTW{h(X)}\right)= \int\limits_{\left\{ \HLTEQ[teal!40]{\HLTW{x}\in \support[X]} \right\}} \HLTW{h(x)}  \; \HLTY{f_{_X}(\HLTW{x})} dx ,}$$
}\\
\vspace{.2in}



\qBrd[7.2in]{amber!40}{
\qBrd[1.9in]{amber!70}{\Large Z-Transformation } 
 If $\HLTW{\Large X\sim \text{Normal}(\mu, \sigma^2)}$ for some $\mu\in \R$ and $\sigma^2>0$ then , 
 $$\DBX{\HLTW{ \Large Z \sim \text{Normal}(\Large 0,1) }\text{where } \HLTY{\displaystyle \Large Z= \frac{X-\mu}{\sigma}}}$$
}


\qBrd[7.2in]{babyblue!70}{
\qBrd[2.35in]{babyblue!40}{Probability Mass Function (pmf)}
For a discrete random vector  $(X,Y)$, we define the probability mass function (pmf) $\pmf_{{_X},{_Y}}(x,y)$ of $X$ by 
$\HLTW{\HLTY{\pmf_{{_X},{_Y}}(x,y)}= P(X=x, Y=y ) }\text{ for all } \HLTW{ (x,y)\in \support[X,Y]}$
}


\qBrd[7.2in]{amber!50}{
\qBrd[1in]{amber!30}{Bivariate CDF}
Let X, Y be two discrete random variables. The joint cumulative distribution function is given by
\qBrd[3in]{antiquefuchsia!40}{
$\displaystyle \HLTW{ F_{_{X,Y}}(\HLTY{x},\HLTY{y})}:= P\left(X\leq \HLTY{x}, Y\leq \HLTY{y} \right) .$}
\vspace{.1in}
}\\
\vspace{.2in}



\qBrd[7.2in]{antiquefuchsia!50}{
\vspace{-.35in}
\begin{center}
\qBrd[5.6in]{antiquefuchsia!70}{\begin{center} \Large Marginal Distribution of a Discrete Random Variable \end{center} }
\qbx[3.2in]{babyblue!40}{
The marginal probability mass function of $X$ is given by
$$\pmf_{_X}(\HLTY{x})=  \sum_{\left\{\HLTW{t}: (\HLTY{x}, \HLTW{t})\in \support[XY]\right\}}  \pmf_{_{X,Y}}(\HLTY{x}, \HLTW{t})$$
}
\qbx[3.2in]{babyblueeyes!40}{
The marginal probability mass function of $X$ is given by
$$\pmf_{_Y}(\HLTY{y})=  \sum_{\left\{\HLTW{s}: (\HLTW{s},\HLTY{y})\in \support[XY]\right\}}  \pmf_{_{X,Y}}(\HLTW{s}, \HLTY{y})$$
}
\end{center}
}



\qbx[7.2in]{amber!40}{
\qBrd[1.5in]{amber!70}{ \Large Joint Density}
We say that X and Y are jointly continuous if there exists a function \HLTW{f_{_{X,Y}}(x, y)}, defined
for all real x and y, having the property that, for every set $C$  of pairs of real numbers (that is, C is a set in the two-dimensional plane), 
 $$\DBX{P\left(\HLTW{(X,Y) \in C}\right)=  \iint\limits_{\{(x,y)\in C\}}  f_{_{X,Y}}(\HLTY{s}, \HLTW{t})ds \;dt}$$
 \HLTW{f_{_{X,Y}}(x, y)} is called the joint probability density function of the random vector $(X,Y)$.
}\\


\qBrd[4.2in]{olive!40}{
$$\qBrd[3.0in]{teal!40}{ \Large  Joint CDF from Joint p.d.f.   }$$
If  the joint probability density function of X and Y is  $f_{_{X,Y}}(x, y).$ then 
$$ \HLTW{F_{_{X,Y}}(\HLTY{x},\HLTY{y})}= \HLTEQ[teal!30]{\displaystyle \mathop{\int\int}_{\left\{ \Col{{\HLTW{s}\leq \HLTY{x}, \HLTW{t} \leq \HLTY{y} },{\text{ where } (s,t)\in \HLTW{\support[ X,Y]}}}\right\}} f_{_{X,Y}}(\HLTW{s}, \HLTW{t}) ds\; dt }$$
}
\qBrd[3in]{babyblue!40}{\sqBullet{babyblue} If CDF of a bivariate continuous random variable is provided, then the corresponding p.d.f is obtained by following:  $$\DBX{\displaystyle f_{_{X,Y}}(x, y)  = \frac{d^2F(x,y)}{dx\;dy}}
$$}

\vspace{.2in}

\qBrd[7.5in]{amber!50}{
\vspace{-.35in}
\begin{center}
\qBrd[4.9in]{amber!70}{\begin{center} \Large Marginal Distribution of a Continuous Random Variable \end{center} }
$\Row{{
\qbx[3.2in]{babyblue!40}{
The marginal probability mass function of $X$ is given by\\
\qBrd[3.1in]{amethyst!40}{
$$f_{_X}(\HLTY{x})=  \int\limits_{\{y: (\HLTY{x},\HLTW{y})\in \support[XY]\}}  f_{_{X,Y}}(\HLTY{x}, \HLTW{y})dy $$
}
}},
{\qbx[3.2in]{amethyst!50}{
The marginal probability mass function of $Y$ is given by\\
\qBrd[3.1in]{babyblue!40}{
$$f_{_Y}(\HLTY{y})=  \int\limits_{\{x: (\HLTY{x},\HLTW{y})\in \support[XY]\}}  f_{_{X,Y}}(\HLTW{x}, \HLTY{y})dx $$}
}
}}$
\end{center}
}

\vspace{.2in}

\qBrd[7.2in]{amber!40}{
\vspace{-.35in}
$$\qBrd[5.8in]{olive!50}{\begin{center} \Large Conditional p.m.f.  of a Discrete Random Variable \end{center} }$$
If $\pmf_{_{XY}}(x, y)$ denotes the joint probability mass function (pmf) of two discrete random variables X and Y and if $\pmf_X(x)$ and $\pmf_Y (y)$ denote the marginal probability function of $X,$ ($Y $ respectively), then \\
\qBrd[3.2in]{babyblue!40}{
The conditional probability of X given $Y = y $ is given by $$\DBX{\pmf_{_{X\mid Y}}(x\mid y)=  \frac{\pmf_{_{X,Y}}(x,y)}{\pmf_{_Y}(y)}}$$
}
\qBrd[3.2in]{babyblue!40}{
The conditional probability of Y given X = x is given by
$$\DBX{\pmf_{_{ Y\mid X}}(y \mid x)=  \frac{\pmf_{_{X,Y}}(x,y)}{\pmf_{_X}(x)}.}$$
}
}




\vspace{.2in}

\qBrd[7.2in]{antiquefuchsia!50}{
\vspace{-.35in}
$$\qBrd[5.8in]{olive!50}{\begin{center} \Large Conditional p.d.f.  of a Continuous Random Variable \end{center} }$$
If $f_{_{XY}}(x, y)$ denotes the joint probability density function of two continuous random variables X and Y and if $f_X(x)$ and $f_Y (y)$ denote the marginal probability density function of $X,$ ($Y $ respectively), then \\
\qBrd[3.2in]{babyblue!40}{
The conditional probability density of X given $Y = y $ is given by $$\DBX{f_{_{X\mid Y}}(x\mid y)=  \frac{f_{_{X,Y}}(x,y)}{f_{_Y}(y)}}$$
}
\qBrd[3.2in]{babyblue!40}{
The conditional probability density of Y given X = x is given by
$$\DBX{f_{_{ Y\mid X}}(y \mid x)=  \frac{f_{_{X,Y}}(x,y)}{f_{_X}(x)}.}$$
}
}
\vspace{.2in}


\qBrd[7.2in]{amber!40}{
\vspace{-.35in}
$$\qBrd[5.8in]{amber!50}{\begin{center} \Large Statistically Independent Random Variable \end{center} }$$
The random variables $X$ and $Y$ are said to be {\bf statistically}  {\bf independent } random variables if, for any two events A and B,
$$\qBrd[4in]{babyblue!60}{$$ \HLTW{ \displaystyle P(X\in A, Y\in B)= P(X\in A) \times P(Y\in B)}$$}$$
}


\qBrd[7.2in]{amethyst!30}{
\vspace{-.35in}
$$\qBrd[5.8in]{amethyst!50}{\begin{center} \Large Statistically Independent Continuous Random Variables \end{center} }$$
\sqBullet{babyblue}Let $(X,Y)$ be bivariate continuous random vector with a probability density function $f_{_{X,Y}}(x,y)$ on the support $(x,y)\in \support[X,Y]$. \\
\sqBullet{babyblue} Let $f_{_X}(x)$ be the marginal density of the random variable $X$ on the support $\support[X]$\\
\sqBullet{babyblue} Let $f_{_Y}(y)$ be the marginal density of the random variable $X$ on the support $\support[Y]$. \\
The continuous random variables X and Y are {\bf statistically} {\bf  independent } if the corresponding joint probability density function  $$ \HLTW{ \displaystyle f_{_{X,Y}}(x, y) = f_{_X}(x) \times  f_{_Y} (y)}$$ for all x and y, and $\support[X,Y]= \support[X]\times \support[Y].$
}


\qBrd[7.2in]{olive!30}{
\vspace{-.35in}
$$\qBrd[5.8in]{olive!50}{\begin{center} \Large Statistically Independent Discrete Random Variables \end{center} }$$
\sqBullet{babyblue}Let $(X,Y)$ be bivariate continuous random vector with a probability density function $f_{_{X,Y}}(x,y)$ on the support $(x,y)\in \support[X,Y]$. \\
\sqBullet{babyblue} Let $f_{_X}(x)$ be the marginal density of the random variable $X$ on the support $\support[X]$\\
\sqBullet{babyblue} Let $f_{_Y}(y)$ be the marginal density of the random variable $X$ on the support $\support[Y]$. \\
The continuous random variables X and Y are {\bf statistically} {\bf  independent } if the corresponding joint probability density function  $$ \HLTW{ \displaystyle f_{_{X,Y}}(x, y) = f_{_X}(x) \times  f_{_Y} (y)}$$ for all x and y, and $\support[X,Y]= \support[X]\times \support[Y].$
}\\

%}
\vspace{.35in}
\qBrd[7.2in]{amber!40}{
\vspace{-.35in}
$$
\qBrd[4.6in]{amber!70}{\begin{center} \Large A Few Properties of Statistically Indipendent Random Variables \end{center} }$$
 Let $X, Y$ be any two statistically independent random variables then then the following facts are true: 
\begin{itemize}
\qBrd[6.1in]{babyblue!40}{
\item[\sqBullet{babyblue}] For any two events $A,B$
$\HLTW{P(X\in A, Y\in B)= P(X\in A)\times P(Y\in B)}$
}\\\vspace{.05in}
\qBrd[6.1in]{teal!40}{
\item[\sqBullet{teal}] For any two functions* $h(x)$ and $g(y)$ \vspace{.05in}
$\HLTW{E\left(g(X) h(Y) \right) = E\left(g(X) \right) E\left( h(Y)\right)}$
}\\\vspace{.05in}
\qBrd[6.1in]{amethyst!40}{
\item[\sqBullet{amethyst}] If the X, Y has the marginal CDFs $F_{_X}(x)$ and $F_{_Y}(y)$ respectively,  then the joint CDF
$\HLTW{F_{_{X,Y}}(x,y) =F_{_X}(x) \times  F_{_Y}(y)}$
for all x, y. 
}
\end{itemize}
\vspace{.2in}
}


\vspace{.35in}
\qBrd[7.4in]{amber!35}{
\vspace{-.35in}
\begin{center}
\qBrd[4.6in]{amber!70}{\begin{center} \Large {\Large \bf ``Expected Value''} of a  function of a Bivariate Random Vectors \end{center} }
$\Row{{
\qbx[3.3in]{babyblue!40}{
Let X, Y be two discrete random variables with joint probability function $\pmf_{_{X,Y}}(x, y)$. Then the expected value of $g(X, Y )$ is given by
$$\HLTW{ \displaystyle  E\left( g(X,Y) \right)=  \mathop{\sum\sum}_{(x,y)\in \support[XY]} g(x,y) \pmf_{_{X,Y}}(x,y)}$$
}},{
\qbx[3.3in]{amethyst!40}{
Let X, Y be two continuous random variables with joint probability density function $f_{_{X,Y}}(x, y)$. Then the expected value of $g(X, Y )$ is given by \vspace{-.1in}
$$\HLTW{ \displaystyle E\left( g(X,Y) \right)= \mathop{ \int\int}\limits_{(x,y)\in \support[XY]} g(x,y) f_{_{X,Y}}(x,y) dx\; dy} $$
}}}$
\end{center}
}



\vspace{.35in}
\qBrd[7.2in]{amber!35}{
\vspace{-.35in}
\begin{center}
\qBrd[4.6in]{amber!70}{\begin{center} \Large {\Large \bf ``Covariance''} Between Two Random Variables \end{center} }\\
Let $X$, and $Y$ be two random variables with a joint distribution.  Then $$\HLTW{\text{Cov}(X,Y)=E \left( (X-\mu_{_X}) (Y-\mu_{_Y})\right),} $$
where $\mu_{_X}$ and $\mu_{_Y}$ denotes the mean of the random variables $X$, and $Y$ respectively. \\
\vspace{.1in}
\qBrd[4.2in ]{amethyst!40}{
An Alternative Formulation for the covariance is the following:\\
$$\qBrd[2.5in ]{amethyst!60}{{ \HLTW{\text{Cov}(X,Y)}= \HLTW{E(XY)- E(X)E(Y)} }}$$
\vspace{-.1in}
}\\\vspace{.1in}
\qBrd[6.8in ]{airforceblue!40}{{\bf Theorem:}
If $X$, and $Y$ are two {\bf statistically independent} random variables, then 
$\DBX{\HLTW{\text{Cov}(X,Y)=0 }.}$
{ However, the converse of the result is not true in general. i.e.  $\HLTW{\text{Cov}(X,Y)=0}$ does not imply that $X$, $Y$ must be Statistically Independent. }
}
\end{center}
}



\vspace{.35in}
\qBrd[7.2in]{amber!35}{
\vspace{-.35in}
\begin{center}
\qBrd[4.6in]{amber!70}{\begin{center} \Large {\Large \bf Mean, Variance, and Covariance''}  of  Linear Combinations of Random Variables \end{center} }\\
\qbx[6.4in ]{amethyst!40}{
Let $X_1, X_2, \ldots X_n$ are random variables and $\HLTW{\displaystyle  Y =a_0+ \sum_{i=1}^{n} a_i X_i }$, where $a_i's$ are constants then 
\begin{enumerate}
 \qBrd[3.1in ]{amber!40}{ \item[\sqBullet{amber}] $\displaystyle E(Y)=  a_0+\sum_{i=1}^{n} a_i E\left(X_i\right) $}\\
 \qBrd[4.1in ]{applegreen!40}{  \item[\sqBullet{olive}] $\displaystyle \text{Var}(Y)=  \sum_{i=1}^{n} a^2_i \text{Var}\left(X_i\right)+ 2 \mathop{\sum\sum}_{1\leq i <j \leq n}  a_i a_j \text{Cov}(X_i, X_j) $}
\end{enumerate}
}
\qbx[6.4in ]{amethyst!40}{
Let $X_1, X_2, \ldots X_n$ are random variables$\HLTW{\displaystyle Y_1 =a_0+ \sum_{i=1}^{n} a_i X_i },  \text{ and }  \HLTW{\displaystyle Y_2 =b_0+ \sum_{i=1}^{n} b_i X_i },$
where $a_i's$ abd $b_i's$ are constants then \\
\qBrd[4.2in ]{applegreen!40}{$\HLTW{\displaystyle \text{Cov}(Y_1, Y_2)=  \sum_{i=1}^{n} a_i b_i \text{Var}\left(X_i\right)+ 2\mathop{\sum\sum}_{1\leq i <j \leq n}  a_i b_j \text{Cov}(X_i, X_j)} $}
}
\vspace{.01in}
 \qBrd[6.4in ]{green!40}{ \sqBullet{teal} If $X_1, X_2, \ldots X_n$are 
 mutually pairwise {\bf statistically independent} then, 
 $$\HLTW{\text{Var}(Y)=  \sum_{i=1}^{n} a^2_i \text{Var}\left(X_i\right)}, \HLTW{ \displaystyle \text{Cov}(Y)=  \sum_{i=1}^{n} a_i b_j \text{Var}\left(X_i\right)}$$}
 \end{center}
}


\vspace{.5in}
\newpage
\newcommand{\binP}{\pi}
\qBrd[7.2in]{antiquefuchsia!50}{\vspace{-.35in}
\begin{center}
\qBrd[4.9in]{apricot!80}{ \begin{center} \bf \Large Standard Properties of  a few Discrete Distributions \end{center} }
\end{center}
\begin{center}
\qBrd[6.99in]{white!40}{
{
\begin{tabular}{|c|c|c|c|c|c|}
\hline
 Distribution & Support  &  pmf    & Mean   &  Variance  & mgf   \\
& $\support[X]$  &   $\pmf_{_X}(x)$   &  $E(X)$  &   $\text{Var}(X)$  &  $M_{_X}(t)$  \\
 \hline \hline
 & & & & & \\
Binomial$(n, \binP)$ &$ \{0, 1, \ldots, n\}$ & ${n \choose x} \binP^x (1-\binP)^{n-x}  $ & $n\binP$  & $n\binP(1-\binP)$&   $\left( 1-\binP+\binP e^t\right)^n$   \\
 & & & & & \\
 \hline
 \hline
 & & & & & \\
Poisson$( \lambda)$ &$ \{0, 1, 2, \ldots \}$ & $  \frac{e^{-\lambda}\lambda^x}{x!} $ & $ \lambda$  & $ \lambda$&   $e^{ \lambda e^t-  \lambda}  $   \\
 & & & & & \\
 \hline
  \hline
   & & & & & \\
  Geometric$( \binP)$ &$ \{1, 2, \ldots \}$ & $ (1-\binP)^{x-1}\binP  $ & $ \frac{1}{\binP} $  & $ \frac{1-\binP}{\binP^2} $&   $ \frac{ {\binP} e^t}{1-(1-\binP) e^{t}}    $   \\
 & & & & & \\
 \hline
 \hline 
 & & & & & \\
Negative-Binomial$(r,  \binP)$ &$ \{r, r+1, r+2, \ldots \}$ & $ {{x-1}\choose{r-1}}(1-\binP)^{x-r}\binP^{r}  $ & $ \frac{r}{\binP} $  & $ \frac{r(1-\binP)}{\binP^2} $&   $\left( \frac{  \binP e^t}{1-(1-\binP) e^{t}}  \right)^r  $   \\
 & & & & & \\
 \hline
  \hline
\end{tabular}
}}\end{center}}

\vspace{.2in}


\qBrd[7.2in]{antiquefuchsia!50}{\vspace{-.35in}
\begin{center}
\qBrd[4.9in]{slateblue!80}{ \begin{center} \bf \Large Standard Properties of  a few Continuous Distributions \end{center} }
\end{center}
\begin{center}
\qBrd[6.99in]{white!40}{
{\small
\begin{tabular}{|c|c|c|c|c|c|}
\hline
 Distribution & Support  &  pdf    & Mean   &  Variance  & mgf   \\
& $\support[X]$  &   $f_{_X}(x)$   &  $E(X)$  &   $\text{Var}(X)$  &  $M_{_X}(t)$  \\
 \hline \hline
 & & & & & \\
Uniform$(a,b)$ &$ [a, b]$ & $ \begin{cases} \frac{1}{b-a} & \text{ if } x\in [a,b]\\ 0 & \text{ otherwise .} \end{cases} $ & $  \frac{a+b}{2}  $  & $  \frac{(b-a)^2}{12}$&   $M_{_X}(t)
=\begin{cases} 
 \frac{e^{tb}- e^{ta}}{t(b-a)} & \text{ if } t\neq 0 \\
 1 & \text{ if } t= 0  
 \end{cases}$\\
 & & & & & \\
\hline \tiny
 & & & & & \\
 $\text{Exponential}(\lambda)$&$ (0, \infty)$ & $ \begin{cases} \lambda e^{-\lambda x} & \text{ if } x>0\\ 0 & \text{ otherwise .} \end{cases} $ & $ \frac{1}{\lambda} $  & $ \frac{1}{\lambda^2}$&   $
 \frac{\lambda}{\lambda-t}  \text{ if } 0\leq  t < \lambda$\\
 & & & & & \\
 \hline
 & & & & & \\
 $\text{Gamma}(\alpha, \lambda )$&$ (0, \infty)$ & $  \frac{\lambda^{\alpha}}{\Gamma(\alpha)} x^{\alpha-1}e^{-\lambda x}  $ & $ \frac{\alpha}{\lambda} $  & $ \frac{\alpha}{\lambda^2}$&   $ \frac{1}{\left( 1-\frac{t}{\lambda}\right)^{\alpha}}  \text{ if } 0\leq  t < \lambda$\\
 $\HLTW{\text{shape}= \alpha, \text{rate}}= \lambda$ & & $\text{ if } x>0$ & & & \\
 \hline
 & & & & & \\
 $\text{Beta}(\alpha, \beta )$&$ (0, 1)$ & $   \frac{\Gamma(\alpha+\beta)}{\Gamma(\alpha)\Gamma(
 \beta)}x^{\alpha-1}x^{\beta-1}  $ & $\frac{\alpha}{\alpha+\beta}  $  & $ \frac{\alpha \beta}{ (\alpha+\beta)^2(\alpha+\beta+1)} $&   $ --- $\\
  & & $\text{ if } 0<x<1$ & & & \\
 \hline
 & & & & & \\
 $\text{Normal}(\mu, \sigma^2 )$&$ (-\infty , \infty )$ & $  \frac{1}{\sigma\sqrt{2\pi}}e^{-\frac{(x-\mu)^2}{2\sigma^2}} $ & $\mu  $  & $\sigma^2 $&   $  e^{\mu t +\frac{t^2\sigma^2}{2}}$\\
 $\HLTW{\text{mean}= \mu, \text{Var}}= \sigma^2$ & & $\text{ for  } x\in \R $ & & & \\
 \hline
  \hline
\end{tabular}
}}\end{center}}




%\qbx[7.1in]{teal!40}{
%\Center{\qBrd[3in]{applegreen!30}{\Center{\Large A Standard Deck of Cards\vspace{.2in}}}\vspace{.2in}}
%\begin{center}
%\includegraphics[scale=.5]{figs/Cards.png}
%\end{center}
%}
$\Col{{\includegraphics[scale=.5]{figs/NormalTableDiagram2.png}},{
\includegraphics[scale=.9]{figs/NormalTable2.png}}}$\\
$\Col{{\includegraphics[scale=.5]{figs/NormalTableDiagram1.png}},{
\includegraphics[scale=.9]{figs/NormalTable1_entire.png}}}$




\end{document}