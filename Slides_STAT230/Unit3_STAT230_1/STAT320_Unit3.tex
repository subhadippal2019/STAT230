\documentclass[compress]{beamer}
\mode<presentation>
\setbeamercovered{transparent}
\usetheme{Warsaw}
%\useoutertheme{smoothtree}
\usepackage{multirow}
\usepackage[english]{babel}
\usepackage[latin1]{inputenc}
\usepackage{times}
\usepackage[T1]{fontenc}
\usepackage{xmpmulti}
\usepackage{multicol}
\usepackage{colortbl}

%\setbeamersize{text margin left=.25 in,text margin right=.25 in}
\setbeamersize{text margin left=.15 in,text margin right=.15 in}
\usepackage[authoryear]{natbib}


\usepackage{epstopdf}
\usepackage{xcolor}
\usepackage{latexcolors}
%\usepackage[dvipsnames]{xcolor}
\definecolor{antiquebrass}{rgb}{0.8, 0.58, 0.46}
\definecolor{babyblueeyes}{rgb}{0.63, 0.79, 0.95}
\definecolor{babyblue}{rgb}{0.54, 0.81, 0.94}
\definecolor{bistre}{rgb}{0.24, 0.17, 0.12}
\definecolor{brightlavender}{rgb}{0.75, 0.58, 0.89}
\definecolor{bulgarianrose}{rgb}{0.28, 0.02, 0.03}
\definecolor{slateblue}{rgb}{0.56, 0.74, 0.56}
\definecolor{cordovan}{rgb}{0.54, 0.25, 0.27}
\definecolor{darkbyzantium}{rgb}{0.36, 0.22, 0.33}

\setbeamercolor{structure}{fg=babyblue!100, bg= black!60}







\usepackage{tikz}
\usetikzlibrary{shadows,calc}
\usetikzlibrary{shadows.blur}
\usetikzlibrary{shapes.symbols}
\usepackage{hyperref}
\usepackage{booktabs}
\usepackage{colortbl}
\usepackage{multirow}
%%%%%%%%% shaddow image %%%%%
% some parameters for customization
\def\shadowshift{3pt,-3pt}
\def\shadowradius{6pt}
\colorlet{innercolor}{black!60}
\colorlet{outercolor}{gray!05}
% this draws a shadow under a rectangle node
\newcommand\drawshadow[1]{
\begin{pgfonlayer}{shadow}
    \shade[outercolor,inner color=innercolor,outer color=outercolor] ($(#1.south west)+(\shadowshift)+(\shadowradius/2,\shadowradius/2)$) circle (\shadowradius);
    \shade[outercolor,inner color=innercolor,outer color=outercolor] ($(#1.north west)+(\shadowshift)+(\shadowradius/2,-\shadowradius/2)$) circle (\shadowradius);
    \shade[outercolor,inner color=innercolor,outer color=outercolor] ($(#1.south east)+(\shadowshift)+(-\shadowradius/2,\shadowradius/2)$) circle (\shadowradius);
    \shade[outercolor,inner color=innercolor,outer color=outercolor] ($(#1.north east)+(\shadowshift)+(-\shadowradius/2,-\shadowradius/2)$) circle (\shadowradius);
    \shade[top color=innercolor,bottom color=outercolor] ($(#1.south west)+(\shadowshift)+(\shadowradius/2,-\shadowradius/2)$) rectangle ($(#1.south east)+(\shadowshift)+(-\shadowradius/2,\shadowradius/2)$);
    \shade[left color=innercolor,right color=outercolor] ($(#1.south east)+(\shadowshift)+(-\shadowradius/2,\shadowradius/2)$) rectangle ($(#1.north east)+(\shadowshift)+(\shadowradius/2,-\shadowradius/2)$);
    \shade[bottom color=innercolor,top color=outercolor] ($(#1.north west)+(\shadowshift)+(\shadowradius/2,-\shadowradius/2)$) rectangle ($(#1.north east)+(\shadowshift)+(-\shadowradius/2,\shadowradius/2)$);
    \shade[outercolor,right color=innercolor,left color=outercolor] ($(#1.south west)+(\shadowshift)+(-\shadowradius/2,\shadowradius/2)$) rectangle ($(#1.north west)+(\shadowshift)+(\shadowradius/2,-\shadowradius/2)$);
    \shade[outercolor,right color=innercolor,left color=innercolor] ($(#1.north west)+(-\shadowradius/12,\shadowradius/12)$) rectangle ($(#1.south east)+(\shadowradius/12,-\shadowradius/12)$);%Frame
    \filldraw ($(#1.south west)+(\shadowshift)+(\shadowradius/2,\shadowradius/2)$) rectangle ($(#1.north east)+(\shadowshift)-(\shadowradius/2,\shadowradius/2)$);
\end{pgfonlayer}
}
% create a shadow layer, so that we don't need to worry about overdrawing other things
\pgfdeclarelayer{shadow} 
\pgfsetlayers{shadow,main}
% Define image shadow command
\newcommand\shadowimage[2][]{%
\begin{tikzpicture}
\node[anchor=south west,inner sep=0] (image) at (0,0) {\includegraphics[#1]{#2}};
\drawshadow{image}
\end{tikzpicture}}
\usepackage{calligra}

\DeclareMathOperator*{\argmax}{Arg\,max}
\DeclareMathOperator*{\argmin}{Arg\,min}
\newcommand{\norm}[1]{\left\Vert #1 \right\Vert }
\newcommand{\bbetaHat}{ \widehat{\bbeta}}
\newcommand{\bbetaLSE}{ \widehat{\bbeta}_{_{\text{LSE}}}}
\newcommand{\bbetaMLE}{ \widehat{\bbeta}_{_{\text{MLE}}}}
\newcommand{\sqBullet}[1]{  {\tiny \tiny \tiny \qBoxCol{#1!60}{ }} }
%***************
%\newtheorem{thm}{Theorem}
%\documentclass[noinfoline]{imsart}
%\usepackage{amsmath,amstext,amssymb}
%%\usepackage[top=1.5in, bottom=1.5in, left=1.2in, right=1.2in]{geometry}
%% settings
%%\pubyear{2005}
%%\volume{0}
%%\issue{0}
%%\firstpage{1}
%%\lastpage{8}
%\arxiv{arXiv:0000.0000}

%\startlocaldefs
%\numberwithin{equation}{section}
%\theoremstyle{plain}
%\newtheorem{thm}{Theorem}
%\endlocaldefs
\usepackage{lipsum} 
\usepackage{amsmath}
\usepackage{amssymb}
\usepackage{amsbsy} 
\usepackage{amsthm}
\usepackage{mathrsfs}
\usepackage{eufrak}
\usepackage{mathrsfs}
\usepackage{color}
\usepackage{verbatim}
\usepackage{graphicx}
\usepackage{bm}
\usepackage{enumerate}
\usepackage{epstopdf} 
\usepackage{natbib}
\usepackage{undertilde}
%\RequirePackage[colorlinks,citecolor=blue,urlcolor=blue]{hyperref}
%\usepackage{subfig}
\usepackage[final]{pdfpages}

\usepackage{algorithm}  %@subhajit
\usepackage{algpseudocode} %@subhajit
\usepackage{algorithmicx}     %@subhajit
\usepackage{undertilde}


\newcommand{\sphere}{{\mathbb{S}}}
\newcommand{\R}{\mathbb{R}}
\newcommand{\LatentV}{V}
\newcommand{\NC}{m}
\newcommand{\Priorf}{f_{prior}}
\newcommand{\FWOne}[2]{{{}_{1}\Psi _{1}\left[{\begin{matrix}(\frac{#1}{2},\frac{1}{2})\\(1,0)\end{matrix}};#2\right]} 
}


\newcommand{\HyPriorMu}{\thetabf}
\newcommand{\HyPriorAlpha}{\alpha}
\newcommand{\HyPriorBeta}{\beta}
\newcommand{\HyPriorK}{\zeta}
\newcommand{\Indicator}[2]{\mathbb{I}_{_{#1}}({#2 })}
\newcommand{\xb}{\bm{x}}
\newcommand{\bx}{\MakeVec{\bm{x}}}
\newcommand{\bX}{\bm{X}}
\newcommand{\by}{\MakeVec{\bm{y}}}
\newcommand{\bZ}{\bm{Z}}
\newcommand{\bF}{\bm{F}}
\newcommand{\btheta}{\MakeVec{{\bm{\theta}}}}
\newcommand{\Bpi}{\MakeVec{\boldsymbol{\pi}}}
\newcommand{\thetabf}{\MakeVec{\boldsymbol{\theta}}}
\newcommand{\Thetabf}{\boldsymbol{\Theta}}
\newcommand{\taubf}{\MakeVec{\boldsymbol{\tau}}}
\newcommand{\Tr}{Tr}


\newcommand{\bM}{\bm{M}}
\newcommand{\bD}{\MakeVec{\bm{D}}}
\newcommand{\bV}{\MakeVec{\bm{V}}}
\newcommand{\loglikmix}{\mathcal{L}_{\bM,\bD,\bV}}
\newcommand{\hypdc}{{}_0F_1\left(\frac{n}{2},\frac{D_c^2}{4}\right)}


\usepackage{xstring}
\usepackage[normalem]{ulem}
\definecolor{ultramarine}{RGB}{38,29,163}
\newcommand\PalDel[1]{{\color{red} {\sout{#1}}}}
\newcommand\Pal[1]{{\color{ultramarine}{#1}}}
\newcommand\PalRp[2]{\PalDel{#1} \Pal{#2}}
\newcommand\PalCmnt[1]{{\color{ultramarine} {[[[***PAL:  #1 ***]]]}}}

\newcommand{\qedwhite}{\hfill \ensuremath{\Box}}
\newcommand{\SpaceD}{\mathcal{S}_p}
\newcommand{\SpaceM}{\widetilde{\mathcal{V}}_{n,p}}
\newcommand{\SpaceV}{\mathcal{V}_{p,p}}
\newcommand{\SpaceF}{\mathbb{R}^{n,p}}
\newcommand{\StiefelS}{\mathcal{V}_{n,p}}
\newcommand{\SpacePi}{\mathbb{S}_{\pi}}
\newcommand{\ML}{{\cal{ML}}}
\newcommand{\ProdSpace}{\boldsymbol{\Theta}}
\newcommand{\ThetaAndPi}{\Xi}
\newcommand{\ClassML}{\mathcal{C}_{\ML}}


\newcommand{\balpha}{\MakeVec{\bm{\alpha}}}
\newcommand{\bbeta}{\MakeVec{\bm{\beta}}}
\newcommand{\bEta}{\bm{\eta}}
\newcommand{\bd}{{\utilde{\bm{d}}}}
\newcommand{\BoEta}{{\utilde{\boldsymbol{\eta}}}}
%\newtheorem{theorem}{Theorem}[section]
%\newtheorem{theorem}{Theorem}
%\newtheorem{lemma}{Lemma}
%\newtheorem{result}{Result}
\newtheorem{defn}{Definition}
\newcommand{\pdv}[2][]{\frac{\partial#1}{\partial#2}}
\newcommand{\pdvtwo}[2][]{\frac{\partial^2#1}{{\partial#2}^2}}


%\newcommand{\mubf}{\boldsymbol{\mu}}
\newcommand{\mubf}{\MakeVec{\mu}}
\newcommand{\sumI}{ \sum_{i=1}^{n}}
\newcommand{\Ybar}{{\overline{Y}}}

\newcommand{\Expectation}[1]{\mathbb{E}{[#1]}}
\newcommand{\priorXzero}{\Psi}
\newcommand{\iMat}{\mathbf{I}_{p}}

% 
% \newtheorem{thm}{Theorem}[section]
% \newtheorem{cor}[thm]{Corollary}
% \newtheorem{lem}[thm]{Lemma}
%\newtheorem{proposition}{Proposition}

%\newtheorem{theorem}{Theorem}[chapter]%To link the theorem to each chapter uncomment the chapter option
%\newtheorem{lemma}{Lemma}%[theorem]% To link each lemma to a theorem uncomment the theorem option
%\newtheorem{corollary}{Corollary}%[theorem]% To link each corollary to a theorem uncomment the theorem option
% to link a corollary to a chapter change the theorem option to chapter
%\newtheorem{definition}{Definition}%[chapter] %the same is true for both definitions and assumptions
\newtheorem{assumption}{Assumption}%[chapter] %
%\newtheorem{proposition}{Proposition}[chapter]
%\newtheorem{fact}{Fact} %%% added by @subho
\newcommand{\StrongNBD}[2]{S_{#1}{#2}}
\newcommand{\bpi} {\boldsymbol{\pi}}
\newcommand{\bphi} {\boldsymbol{\phi}}
\newcommand{\bb}[1]{\boldsymbol{#1}}
% Definitions of handy macros can go here

\newcommand{\normtwo}[1]{{\left\lVert#1\right\rVert}_2}

\newcommand{\dataset}{{\cal D}}
\newcommand{\fracpartial}[2]{\frac{\partial #1}{\partial  #2}}
\newcommand{\Lesbegue}[1]{\mu_{\btheta_{#1},\bpi_{#1}}}
\newcommand{\fthetapi}[1]{f_{\btheta_{#1},\bpi_{#1}}}
% Heading arguments are {volume}{year}{pages}{submitted}{published}{author-full-names}
\newcommand{\doublehat}[1]{%
    \settoheight{\dhatheight}{\ensuremath{\widehat{#1}}}%
    \addtolength{\dhatheight}{-0.35ex}%
    \widehat{\vphantom{\rule{2pt}{\dhatheight}}%
    \smash{\hspace{-0.5mm}\widehat{#1}}}}

\newcommand{\hyp}{{}_0F_1\left(\frac{n}{2},\frac{D^2}{4}\right)}
\newcommand{\hypinline}{{}_0F_1\left({n}/{2},{D^2}/{4}\right)}

\newcommand{\partialhyp}[1]{\frac{\partial}{\partial\,{d_{#1}}}\,\left[\hyp\right]}

\newcommand{\fracProbZ}[1]{\frac{\langle Z_{ic} \rangle \, #1}{\sum_{i=1}^{N} \langle Z_{ic}\rangle  } }
\newcommand{\EmVar}[1]{\widetilde{#1}^{(c)}}

\newcommand{\IMDY}{{\it{CCPD}}}
\newcommand{\JMDY}{{\it{JCPD}}}

\newcommand{\DYlang}{\frac{\exp(\nu\,\bEta^T\bd)}{{\left[{}_0F_1\left(\frac{n}{2},\frac{D^2}{4}\right)\right]}^{\nu}}}

\newcommand{\logDYlang}{\nu\,\bEta^T\bd - \nu\,\log\left({}_0F_1\left(\frac{n}{2},\frac{D^2}{4}\right)\right)}

\newcommand{\lhyp}{\log\left({}_0F_1\left(\frac{n}{2},\frac{D^2}{4}\right)\right)}

%\jmlrheading{1}{2000}{1-48}{4/00}{10/00}{SS \& JH \& AB}

% Short headings should be running head and authors last names

%\ShortHeadings{BDP and cIBP}{SS \& JH \& AB}
%\firstpageno{1}

\newcommand{\diam}[1]{{{#1}^{\ast}}}

%%% coloring option %%%
\definecolor{auburn}{rgb}{0.53, 0.1, 0.5}
\newcommand{\sss}{\color{auburn}}  %%% for Subhajit
\newcommand{\sse}{\color{black}}
\newcommand{\attn}{\color{red}}
\newcommand{\rms}{\color{magenta}}  %%% for Riten
\newcommand{\rme}{\color{black}}
\newcommand{\MLDensity}{f_{\ML}}
\setlength{\parindent}{0cm}
\newcommand{\posterior}

\newcommand{\variableX}{\bd}
\newcommand{\funch}{\mathfrak{h}}
\newcommand{\IndVzero}[1]{\mathbb{I}({X\in \mathcal{V}^{#1}_0})}
\newcommand{\Rnp}{\mathbb{R}^{n \times p}}
\newcommand{\Rpp}{\mathbb{R}^{p \times p}}
\newcommand{\vecnorm}[1]{\lVert #1\rVert}

\newcommand{\etapsiD}{\eta_{\priorXzero}}
\newcommand{\BoEtapsiD}{\BoEta_{\priorXzero}}

\newcommand{\DMp}{\mathcal{D}^{p \times p}}
\newcommand{\Rplus}{\mathbb{R}_{+}}
\newcommand{\prodMeasure}{\Upsilon}

\newcommand{\m}{{\bf m_{\BoEta}}} 
\newcommand{\SetWithMode}{\mathcal{S}}
\newcommand{\SetWithModePrime}{\mathcal{S}}
\newcommand{\TargetComp}{\mathcal{S}^{\star}}

\newcommand{\ConstCondDen}{K_{\nu, \BoEta}} 

\newcommand{\hyparam}[2]{
    \IfEqCase{#1}{
        {M}{\xi^{#2}_c}
        {V}{\gamma^{#2}_c}%
        
    }
  }
\newcommand{\threepartdef}[6]
{
	\left\{
		\begin{array}{lll}
			#1 & \mbox{if } #2 \\
			#3 & \mbox{if } #4 \\
			#5 & \mbox{if } #6
		\end{array}
	\right.
}

\def\bv{\color{blue}}
\def\ev{\color{black}}
\newcommand{\bch}{\bv }
\newcommand{\ech}{\ev\normalsize}
%\newcommand{\MakeVec}[1]{{\utilde{\bf #1}}}
\newcommand \Measure[2][]{%
  \ifstrempty{#1}{
  \IfEqCase{#2}{
        {M}{\mu}%
        {D}{\mu_1}%
        {V}{\mu_2}
        {X}{\mu}
   }  
  }{
  \IfEqCase{#1}{
  {1}{
   \IfEqCase{#2}{
        {M}{d\mu(M)}%
        {D}{d\mu_1(\bd)}%
        {V}{d\mu_2(V)}
        {X}{d\mu(X)}
        {Y}{d\mu(Y)}
        {MDV} {d\mu(M)\; d\mu_1(\bd) \;d\mu_2(V) }
        }
   } 
   {2}{
   \IfEqCase{#2}{
         {M}{d\mu(M^{\ast})}%
        {D}{d\mu_1(\bd^{\ast})}%
        {V}{d\mu_2(V^{\ast})}
        {X}{d\mu(X^{\ast})}
        }
   }
   {3}{
   \IfEqCase{#2}{
         {M}{\mu(dM^{\star})}%
        {D}{\mu_2(d\bd^{\star})}%
        {V}{\mu_1(dV^{\star})}
        {X}{\mu(X^{\star})}
        }
   }   
   
   } 
  }%
}
  \newcommand{\VONF}{\text{VonMisesFisher}}
\newcommand{\MPGalpha}{\alpha}
\newcommand{\MPGnu}{\nu}
\newcommand{\MPG}{MPG }
\newcommand{\ybin}{y^{(\text{bin})}}


%\newcommand{\abs}[1]{\left \vert  #1  \right\vert  }
\usepackage{caption}
\usepackage{subcaption}

%%%%%%%%%%%%%%%%%%%%%%%%%%%
\newcommand{\IEHC}{\text{IEHC}}







\newcommand \Th[1]{%
  \IfEqCase{#1}{
        {1}{ 1^{\text{st}}}%
        {2}{2^{\text{nd}}}%
        {3}{3^{\text{rd}}}%
  }[{#1}^{\text{th}}]
}
  
  
   \newcommand{\augV}{\text{aux}}
  
  
  
  
  \newcommand{\CDE}{\text{PL}}
\newcommand{\CDEsigma}{\sigma}
\newcommand{\CDEepsilon}{\SVepsilon}
\newcommand{\CDEmu}{\mu}
 % \newcommand{\SVepsilon}{\varepsilon}
  \newcommand{\SVepsilon}{\delta}
 \newcommand{\abs}[1]{\left\lvert{#1}\right \rvert }
 
 
\newcommand{\CPDX }{\text{CPDX}}
\newcommand{\CPDXPar}{\vartheta}
\newcommand{\K}{\mathcal{K}}



\newcommand{\lossFunctionOne}[1]{ \left\{ \abs{ ( \abs{#1}-\SVepsilon)}  + ( \abs{#1}-\SVepsilon)\right\} }

\newcommand{\lossFunctionAlt}[1]{ \abs{  #1-\SVepsilon}  + \abs{ #1+\SVepsilon}-2\SVepsilon }

\newcommand{\lossFunctionAltOne}[1]{   \lossFunctionAlt{ \frac{\left(#1\right)}{\sigma}}}

\newcommand{\lossFunction}[1]{ \left\{ \abs{ \left( \frac{\abs{#1}}{\sigma}-\SVepsilon\right)}  + \left( \frac{\abs{#1}}{\sigma}-\SVepsilon\right)\right\} }
\newcommand{  \Likelihood}{\mathcal{L}}
%\newcommand{\Onebf}{\bf 1}
\newcommand{\Onebf}{{\bf \utilde{1}_{n}}}





\newcommand{\InvGamma}{\text{InvGamma}}
\newcommand{\PriorSigmaAlpha}{a}
\newcommand{\PriorSigmaBeta}{b}
\newcommand{\PriorBetaMean}{\mubf_{_{\bbeta}}}
\newcommand{\PriorBetaVar}{\Sigma_{_{\bbeta}}  }
\newcommand{\mvnormPdf}[4]{\frac{1}{ \left({2\pi}\right)^{\frac{#4}{2}} \sqrt{\vert{#3}\vert}}{\exp\left[ - \frac{1}{2}(#1- #2)^T {#3}^{-1} (#1- #2)\right]}      }
\newcommand{\InvGammaPdf}[3]{ \frac{(#1)^{-#2+1}}{\Gamma\left( #2\right) } \exp\left[ -\frac{{#3}}{{#1}} \right] }

 \newcommand{\byTilde}{\tilde{\by}}
 
 \newcommand{\TrfSigma}{\varsigma}
 \newcommand{ \Normal}{\text{Normal}}
 \newcommand{\GlobalPar}{\tau}
\newcommand{\LocalPar}{\psi}
\newcommand{\Not}[1]{{\overline{#1}}}
\newcommand{\st}{:}

\newcommand{\define}[2]{ \begin{definition}[#1]  #2  \end{definition}  }

\newcommand{\Exmpl}[2]{\qBrd[0.75in]{#1}{Example #2:}}
\newcommand{\Qn}{\HLTW{Question:} }


\newcommand{\pmf}{p}
\newcommand{\cdf}{F}
%\NewDocumentCommand{\support}{O{ }}{{\mathcal{S}}_{_{#1}}}
\NewDocumentCommand{\support}{O{ }}{{\mathbb{S}}_{_{#1}}}
%\newcommand{\SampleS}{\mathcal{S}}
\newcommand{\SampleS}{\mathscr{S}}
\usepackage{xcolor}
\usepackage{xparse}
\definecolor{lightGray}{gray}{0.95}
\definecolor{lightGrayOne}{gray}{0.9}
\definecolor{lightBlueOne}{RGB}{204, 255, 255}
\definecolor{lightBlueTwo}{RGB}{204, 238, 255}
\definecolor{lightBlueThree}{RGB}{204, 204, 255}
\definecolor{AltBlue}{RGB}{119,14,161}
\definecolor{Orchid}{RGB}{186,85,211}

\definecolor{BGBlue}{RGB}{220,221,252}
\definecolor{BGBlueOne}{RGB}{204,229,255}

\definecolor{DarkGreenOne}{RGB}{34,139,34}

\definecolor{BGGreen}{RGB}{240,243,245}
\definecolor{lightGreenOne}{RGB}{179, 255, 179}
\definecolor{lightGreenTwo}{RGB}{198, 255, 179}
\definecolor{lightGreenThree}{RGB}{243, 255, 230}
\definecolor{AltGreen}{RGB}{193, 240, 240}

\definecolor{BOGreen}{RGB}{180,0,0}
\definecolor{BGGreenOne}{RGB}{220,250,220}

\definecolor{lightBrownOne}{RGB}{255, 221, 204}
\definecolor{lightBrownTwo}{RGB}{255, 229, 204}
\definecolor{lightBrownThree}{RGB}{242, 217, 230}


\definecolor{HLTGreen}{RGB}{230,244,215}
\definecolor{ExcBrown}{RGB}{153,0,0}
\definecolor{ExcBGBrown}{RGB}{255,204,204}
\definecolor{BGYellowOne}{RGB}{255,235,208}
\definecolor{BGPink}{RGB}{255,215,240}

\newcommand{\MakeVec}[1]{{\utilde{\bf #1}}}

\NewDocumentCommand{\MCOption}{O{1.75in} m}{
\TextInBoxTwo[BGPink]{ #1 } {\TextInBoxTwo[white]{.1 in }{ \quad}\HLT{#2}}
}

 \NewDocumentCommand{\ThreeChoices}{O{Do not Know}O{Not confident}O{Confident}}{
\MCOption{#1} \MCOption{#2} \MCOption{#3}
}
 
\NewDocumentCommand{\OneBlock}{ O{HLTGreen} m m }{\colorbox{#1}{\begin{minipage}{#2} $ #3$ \end{minipage}}}

\NewDocumentCommand{\HLT}{ O{HLTGreen} m }{\colorbox{#1}{#2}}
%\NewDocumentCommand{\HLTEQ}{ O{HLTGreen} m }{\colorbox{#1}{$#2$}}
\NewDocumentCommand{\HLTEQ}{ O{white} m }{\colorbox{#1}{$#2$}}

%\newcommand{\HLT}[1]{\colorbox{HLTGreen}{#1}}
\newcommand{\DEHLT}[1]{\colorbox{lightGrayOne}{\color{white} #1}}

\newcommand{\TextInBoxOne}[2]{  {\fcolorbox{white}{lightGrayOne}{\begin{minipage}{#1}  #2 \end{minipage}}}}


\NewDocumentCommand{\TextInBoxTwo}{ O{lightGrayOne} m m } {{\fcolorbox{white}{#1}{\begin{minipage}{#2} { #3} \end{minipage}}}}


\newcommand{\TextInBox}[2]{  {\fcolorbox{BGGreen}{BGGreen}{\begin{minipage}{#1}  #2 \end{minipage}}}}
\newcommand{\TextInBoxCol}[2]{
\fcolorbox{BGBlue}{BGBlue}{%
\begin{minipage}{#1}
 {\color{AltBlue} #2}
\end{minipage}}%
}

\NewDocumentCommand{\TxtBnd}{ O{lightBrownOne} m m } {{\fcolorbox{white}{#1}{\begin{minipage}{#2} { #3} \end{minipage}}}}


\newcommand{\BandInTopBox}[2]{
\fcolorbox{AltBlue}{AltBlue}{%
\begin{minipage}{#1}{ {\color{white}  #2 \hspace{.1in}} }
\end{minipage}}%
}


\newcommand{\TextInBoxThm}[2]{
\fcolorbox{AltBlue}{lightGray}{%
\begin{minipage}{#1}
 {\color{black} #2}
\end{minipage}}%
}

\newcommand{\TextInBoxThmOne}[2]{
\fcolorbox{BGBlue}{BGBlueOne}{%
\begin{minipage}{#1}
 {\color{AltBlue} #2}
\end{minipage}}%
}

\newcommand{\TextInBoxLem}[2]{
\fcolorbox{BGBlue}{lightGray}{%
\begin{minipage}{#1}
 {\color{black} #2}
\end{minipage}}%
}



\newcommand{\TextInBoxLemOne}[2]{
\vspace{.02 in}
\noindent
\fcolorbox{BGBlue}{BGBlue}{%
\begin{minipage}{#1}
 {\color{AltBlue} #2}
\end{minipage}}%
}



\newcommand{\DefBox}[1]{
%\vspace{.1 in}
\noindent
\TextInBoxLem{4.5 in }{
\BandInTopBox{4.4 in }{}
\TextInBoxLemOne{4.4 in }{
#1
}}}





\newcommand{\DefBoxOne}[2]{
%\vspace{.1 in}
\noindent
\TextInBoxLem{4.5 in }{
\BandInTopBox{4.4 in }{#1}
\TextInBoxLemOne{4.4 in }{
#2
}}}


\newcommand{\ThmBox}[2]{
\noindent
\TextInBoxThm{4.4 in }{
\TextInBoxThmOne{4.4 in }{
#1}
#2}
}

\newcommand{\LemBox}[2]{
\noindent
\TextInBoxLem{4.5 in }{
\TextInBoxLemOne{4.4 in }{
#1}
#2}
}

\newcommand{\PropBox}[2]{
\vspace{.1 in}
\noindent
\TextInBoxLem{4.5 in }{
\TextInBoxLemOne{4.4 in }{
#1}
#2}
}




\newcommand{\TextInBoxExc}[2]{
\noindent
\fcolorbox{white}{BGGreenOne}{%
\begin{minipage}{#1}
 {\color{black} #2}
\end{minipage}}%
}


\newcommand{\TextInBoxExample}[2]{
\noindent
\fcolorbox{white}{BGPink}{%
\begin{minipage}{#1}
 {\color{black} #2}
\end{minipage}}%
}


\newcommand{\ExerciseBox}[1]{
\noindent
%\TextInBoxLem{6 in }{
\TextInBoxExc{6 in }{
#1}
%#2}
}


\newcommand{\ExampleBox}[1]{
\noindent
%\TextInBoxLem{6 in }{
\TextInBoxExample{6 in }{
#1}
%#2}
}

\NewDocumentCommand{\CommentBox}{ O{BGBlue}  m }{
\TextInBoxLem{5.5in }{
{\bf Comment:}\\
\TextInBoxLemOne{5.4 in }{
#2}}
}



\newcommand{\HLTY}[1]{\HLTEQ[yellow]{#1}}
\newcommand{\HLTW}[1]{\HLTEQ[white]{#1}}



\newcommand{\qBox}[1]{
  \begin{tikzpicture}
\node[draw=none,shade,
      top color=lightGrayOne,
      bottom color=lightGray,
      rounded corners=2pt,
      blur shadow={shadow blur steps=5}
    ] at (0,0) {    \noindent 
\fcolorbox{white}{BGBlue}{%
\begin{minipage}{4.55 in}
 {\color{black} {
 #1}}
\end{minipage}  }%
 };
 
    \end{tikzpicture}
}
 
 


 

\newcommand{\qBoxCol}[2]{
  \begin{tikzpicture}
\node[draw=none,shade,
      top color=lightGrayOne,
      bottom color=lightGray,
      rounded corners=2pt,
      blur shadow={shadow blur steps=5}
    ] at (0,0) {    \noindent
\fcolorbox{white}{#1}{%
%\begin{minipage}{4.55 in}
\begin{minipage}{4.55 in}
 {
 \color{black} {
 #2}}
\end{minipage}  }%
 };
 
    \end{tikzpicture}
}
  
  
  
  
  
  

\NewDocumentCommand{\qBrd}{O{4.55 in} m m}{
  \begin{tikzpicture}
\node[draw=none,shade,
      top color=#2,
      bottom color=#2,
      rounded corners=2pt,
      blur shadow={shadow blur steps=5}
    ] at (0,0) {    \begin{minipage}{#1}
 {
 \color{black} {
 #3}}
\end{minipage} 

 };
 
    \end{tikzpicture}
}
    
  
  
  
  
\NewDocumentCommand{\qbx}{O{4.55 in} m m}{
  \begin{tikzpicture}
\node[draw=none,shade,
      top color=lightGrayOne,
      bottom color=lightGray,
      rounded corners=2pt,
      blur shadow={shadow blur steps=5}
    ] at (0,0) {    \noindent
\fcolorbox{white}{#2}{%
%\begin{minipage}{4.55 in}
\begin{minipage}{#1}
 {
 \color{black} {
 #3}}
\end{minipage}  }%
 };
 
    \end{tikzpicture}
}
  
 
 
 \newcommand{\CurlyBox}[1]{
\begin{center}
  \begin{tikzpicture}
    \node[tape,draw=none,shade,
      top color=blue!40,
      bottom color=blue!5,
      rounded corners=1pt,
      blur shadow={shadow blur steps=5,shadow blur extra rounding=1.3pt}
    ] at (2,0){\sffamily\bfseries\large #1};
  \end{tikzpicture}
\end{center} 
 }


\newcommand{\CmntBnd}{\BandInTopBox{4.5in}{Comment:}}
\NewDocumentCommand{\TopBand}{O{Comment:} m}{ \BandInTopBox{4.5in}{#2}}

\newcommand{\DBX}[1]{
 	\HLTEQ[AltBlue]{
 				\HLTEQ[BGBlue]{  #1  }
 	}
 }



\NewDocumentCommand{\TransitionFrame}{O{slateblue}m}{
\begin{frame}{ }
\qBoxCol{#1!40}{\vspace{.8in}  \begin{center}\qBrd[2in]{#1!70}{ \begin{center} \vspace{.1in}
  #2 \\
 \vspace{.1in}
\end{center}}\end{center}
\vspace{.7in}
}

\end{frame}

}


\newcommand \rbind[1]{%
    \saveexpandmode\expandarg
    \StrSubstitute{\noexpand#1}{,}{&}[\fooo]%
    %\StrSubstitute{\fooo}{,}{&}[\fooo]%
    \StrSubstitute{\fooo}{;}{\noexpand\\}[\fooo]%
    \StrSubstitute{\fooo}{:}{\noexpand\\}[\fooo]%
    \restoreexpandmode
   \left[ \begin{matrix}\fooo\end{matrix}\right]
    }
    
    
    
   \newcommand \ColVec[1]{%
    \saveexpandmode\expandarg
    \StrSubstitute{\noexpand#1}{,}{\noexpand\\}[\fooo]%
    %\StrSubstitute{\fooo}{,}{&}[\fooo]%
    \StrSubstitute{\fooo}{;}{\noexpand\\}[\fooo]%
    \StrSubstitute{\fooo}{:}{\noexpand\\}[\fooo]%
    \restoreexpandmode
   \left[ \begin{matrix}\fooo\end{matrix}\right]
    }
     \newcommand \RowVec[1]{%
    \saveexpandmode\expandarg
    \StrSubstitute{\noexpand#1}{,}{&}[\fooo]%
    %\StrSubstitute{\fooo}{,}{&}[\fooo]%
    \StrSubstitute{\fooo}{;}{&}[\fooo]%
    \StrSubstitute{\fooo}{:}{&}[\fooo]%
    \restoreexpandmode
   \left[ \begin{matrix}\fooo\end{matrix}\right]
    }



  \newcommand \Row[1]{%
    \saveexpandmode\expandarg
    \StrSubstitute{\noexpand#1}{,}{&}[\fooo]%
    %\StrSubstitute{\fooo}{,}{&}[\fooo]%
    \StrSubstitute{\fooo}{;}{&}[\fooo]%
    \StrSubstitute{\fooo}{:}{&}[\fooo]%
    \restoreexpandmode
    \begin{matrix}\fooo\end{matrix}
    }
        
    
    
    
    \newcommand \Col[1]{%
    \saveexpandmode\expandarg
    \StrSubstitute{\noexpand#1}{,}{\noexpand\\}[\fooo]%
    %\StrSubstitute{\fooo}{,}{&}[\fooo]%
    \StrSubstitute{\fooo}{;}{\noexpand\\}[\fooo]%
    \StrSubstitute{\fooo}{:}{\noexpand\\}[\fooo]%
    \restoreexpandmode
    \begin{matrix}\fooo\end{matrix}
    }

%%%%%%%%%%%%%%%%%%%%% Experimental %%%%%%%%%%%%%%%%%


\ExplSyntaxOn
\DeclareExpandableDocumentCommand{\replicate}{O{}mm}
 {
  \int_compare:nT { #2 > 0 }
   {
    {#3}\prg_replicate:nn {#2 - 1} { #1#3 }
   }
 }
\ExplSyntaxOff


\ExplSyntaxOn
\DeclareExpandableDocumentCommand{\repdiag}{O{}mm}
 {
  \int_compare:nT { #2 > 0 }
   {
    {\prg_replicate:nn {#2}{#3#1}}{#3}
   }
 }
\ExplSyntaxOff


\newcommand \StrRowDiag[1]{%
    \saveexpandmode\expandarg
    \StrSubstitute{\noexpand#1}{,}{&}[\fooo]%
    %\StrSubstitute{\fooo}{,}{&}[\fooo]%
    \StrSubstitute{\fooo}{;}{&}[\fooo]%
    \StrSubstitute{\fooo}{:}{&}[\fooo]%
    \StrCount{\fooo}{&}[\countfooo]
    \restoreexpandmode
    \repdiag[0]{\countfooo+1}{{,}}
   %\left[ \begin{matrix}\fooo\end{matrix}\right]
    }


\newcommand \DiagStrOne[2]{%
    \saveexpandmode\expandarg
    \StrSubstitute{\noexpand#1}{,}{\noexpand#2}[\fooo]%
    \restoreexpandmode
   %\left[ \begin{matrix}\fooo\end{matrix}\right]
   \fooo
    }
    
    \newcommand \DiagStr[1]{%
    \DiagStrOne{#1}{{\StrRowDiag{#1}}}
    }


%$\rbind{\replicate[,]{10}{\Col{\replicate[;]{7}{0}}}}$

%$\Col{1,2,3}$
%$\ColVec{\replicate[;]{5}{B}}$
%$ \StrRowDiag{1,2} $

%$\DiagStr{1,2,3}$

%\repdiag[-]{3}{A}
\ExplSyntaxOn
\NewDocumentCommand{\Split}{ m m o }
 {
  \tarass_split:nn { #1 } { #2 }
  \IfNoValueTF { #3 } { \tl_use:N } { \tl_set_eq:NN #3 } \l_tarass_string_tl
 }

\tl_new:N \l_tarass_string_tl

\cs_new_protected:Npn \tarass_split:nn #1 #2
 {
  \tl_set:Nn \l_tarass_string_tl { #2 }
  % we need to start from the end, so we reverse the string
  \tl_reverse:N \l_tarass_string_tl
  % add a comma after any group of #1 tokens
  \regex_replace_all:nnN { (.{#1}) } { \1\, } \l_tarass_string_tl
  % if the length of the string is a multiple of #1 a trailing comma is added
  % so we remove it
  \regex_replace_once:nnN { \,\Z } { } \l_tarass_string_tl
  % reverse back
  \tl_reverse:N \l_tarass_string_tl
 }
\ExplSyntaxOff

%%%%%%%%%%%%%%%%%%%%%%%%%%%%%%%%

\newcommand{\ShowRowMatrix}[3]{ \left[ {\begin{array}{ccc}
  \line(1,0){22} &{#1} &  \line(1,0){22} \\
     & \vdots& \\
  \line(1,0){22}  &{#2}& \line(1,0){22} \\
   &  \vdots & \\
    \line(1,0){22} &{#3}& \line(1,0){22}  \\
    \end{array}
   } \right]}
 


\newcommand{\ShowColMatrix}[3]{ \left[ {\begin{array}{ccccc}
  \line(0,1){25} & &\line(0,1){25} &  &  \line(0,1){25} \\
  {#1}  & \ldots & {#2} &\ldots   &{#3} \\
 \line(0,1){25} &  & \line(0,1){25}  &  &  \line(0,1){25} \\
    \end{array}
   } \right]}
   
   
   
   
\newcommand{\ShowRowVector}[1]{ \left[ {\begin{array}{ccc}
  \line(1,0){25} &{#1} &  \line(1,0){25} 
    \end{array}
   } \right]}   
   
   
\newcommand{\ShowColVector}[1]{ \left[ {\begin{array}{c}
  \line(0,1){25} \\    {#1} \\   \line(0,1){25}     \end{array}  } \right]}
  
\newcommand{\ColVector}[3]{ \left[ {\begin{array}{c}
  {#1}\\ \vdots \\    {#2}\\ \vdots\\{#3}  \end{array}  } \right]}
  
  
  
  
  
\newcommand{\EqSetThree}[3]{ \left\{ {\begin{array}{c}
  {#1}\\ \vdots \\    {#2}\\ \vdots\\{#3}  \end{array}  } \right.}  
  



\newcommand{\MatrixTypeA}[3]{ \left[ {\begin{array}{ccc}
 {#1}_{1,1} & \cdots & {#1}_{1,{#3}}   \\
  {#1}_{2,1} & \cdots & {#1}_{2,{#3}}   \\
    \vdots  & \ddots& \vdots  \\
     {#1}_{{#2},1} & \cdots & {#1}_{{#2},{#3}}   \\
    \end{array}
   } \right]}
 
\newcommand{\MatrixTypeAKronecker}[4]{ \left[ {\begin{array}{ccc}
 {#1}_{11}{#4} & \cdots & {#1}_{1{#3}}{#4}   \\
  {#1}_{21} {#4} & \cdots & {#1}_{2{#3}} {#4}   \\
    \vdots  & \ddots& \vdots  \\
     {#1}_{{#2}1} {#4} & \cdots & {#1}_{{#2}{#3}} {#4}   \\
    \end{array}
   } \right]}
 



\newcommand{\ShowIMat}{ {\begin{array}{cccc}
 1&  &  &    \\
  & 1 &  &  \\
    &  & \ddots &    \\
     & & & 1   \\
    \end{array}
   } }
 
\newcommand{\ShowVecOne}{
\begin{array}{c}
 1\\ 1 \\    1  
\end{array}
}

 
\newcommand{\ShowUnitVecOne}{
\begin{array}{c}
 1\\ 0 \\   0  
\end{array}
}


\newcommand{\ShowUnitVecTwo}{
\begin{array}{c}
 0\\ 1 \\   0  
\end{array}
}


\newcommand{\ShowUnitVecThree}{
\begin{array}{c}
 0\\ 0\\   1  
\end{array}
}

\newcommand{\ShowZeroThree}{
\begin{array}{c}
 0\\ 0\\   0 
\end{array}
}


\newcommand{\TwoBlockMatrix}[2]{\left[  {\begin{array}{c;{2pt/2pt}c}
   {#1} &  {#2}
   \end{array} }\right]}
   
   \newcommand{\TwoBlockMatrixH}[2]{\left[  {\begin{array}{c}
   {#1} \\
   \hdashline[2pt/2pt]
    {#2}
   \end{array} }\right]}
   
   \newcommand{\TwoBlockH}[2]{ {\begin{array}{c}
   {#1} \\
   \hdashline[2pt/2pt]
    {#2}
   \end{array} }}
   
   
\newcommand{\TwoBlock}[2]{ {\begin{array}{c;{2pt/2pt}c}
   {#1} &  {#2}
   \end{array} }}
   

      
   
   
   
 \newcommand{\ThreeBlockColVec}[3]{
   \left[ {\begin{array}{c}
  #1\\
  \hdashline[2pt/2pt]\\
   \vdots\\
  \hdashline[2pt/2pt]\\
  #2\\
  \hdashline[2pt/2pt]\\
   \vdots\\
  \hdashline[2pt/2pt]\\
   #3\\
    \end{array}
   } \right]
   }



\NewDocumentCommand{\ColDyn}{>{\SplitList{;}}m}
   {
     \left[\begin{array}{c}
       \ProcessList{#1}{ \inserColtitem }
     \end{array}\right]
   }
   \newcommand \inserColtitem[1]{ #1 \\}


\makeatletter
\newcommand{\ColDynAlt}[2][r]{%
  \gdef\@VORNE{1}
  \left[\hskip-\arraycolsep%
    \begin{array}{#1}\vekSp@lten{#2}\end{array}%
  \hskip-\arraycolsep\right]}

\def\vekSp@lten#1{\xvekSp@lten#1;vekL@stLine;}
\def\vekL@stLine{vekL@stLine}
\def\xvekSp@lten#1;{\def\temp{#1}%
  \ifx\temp\vekL@stLine
  \else
    \ifnum\@VORNE=1\gdef\@VORNE{0}
    \else\@arraycr\fi%
    #1%
    \expandafter\xvekSp@lten
  \fi}
\makeatother


\NewDocumentCommand{\eVec}{m O{}}{\MakeVec{e}_{#1, (#2)}}

\NewDocumentCommand{\Ones}{O{3}}{\Col{\replicate[,]{#1}{1}}}
\NewDocumentCommand{\Zeros}{O{3}}{\Col{\replicate[,]{#1}{0}}}











\title{  STAT 320: Principles of Probability\\ {\color{black}  Unit 3: Introduction to Probability}}

\author[UAEU]
{United Arab Emirates University}
\institute[UAEU] % (optional, but mostly needed)
{
  \inst{Department of Statistics}%
  %Indian Institute of Management,  Udaipur\\
  \vspace{0.1in}

  
}

\date{}


\newcommand{\Xnew}{ \HLTEQ[orange]{X_{_{\text{i}}}} }
\newcommand{\Ynew}{ \HLTEQ[orange]{Y_{_{\text{i}}}} }

%\date{\today}

\AtBeginSection[]
{
  \begin{frame}{Inhalt}
 % \begin{multicols}{1}
	\frametitle{Outline}
    \tableofcontents[currentsection]
  %  \end{multicols}
  \end{frame}
}

\begin{document}
\maketitle

%\begin{frame}{Outline}
%%\begin{multicols}{}
%  \tableofcontents
%%\end{multicols}
%\end{frame}

%\section{Introduction to DSBA 2023}
%
%
%\begin{frame}
%\qBoxCol{blue!30}{
%\begin{center} Course  Website \end{center}
%\qbx[4.2in]{teal!40}{\sqBullet{teal} \color{blue} $ \href{https://sites.google.com/iimu.ac.in/dsba2023e/home}{https://sites.google.com/iimu.ac.in/dsba2023e/home}$
%}\\
%\qbx[3.0in]{green!40}{ \sqBullet{green} Regular Announcements.
%}\\
%\qbx[3.0in]{olive!40}{\sqBullet{olive}  Slides and other materials.
%}
%}
%
%\pause
%\qBoxCol{blue!30}{
%\sqBullet{blue}
%You can contact the instructor at {\it subhadip.pal@iimu.ac.in} and schedule for office hours.  
%}
%\pause
%\qBoxCol{olive!30}{
%\sqBullet{olive}
%Mr. Praveen Kumar has been assigned as Teaching Assistant (TA) for this course.  His email I'd is:  {\it praveen.kumar@iimu.ac. }
%}
%
%
%\end{frame}
%


%
%\begin{frame}{Course Outline}
%\hspace{-.1in}\qBoxCol{blue!35}{
%% Please add the following required packages to your document preamble:
%% \usepackage{booktabs}
%\begin{table}[]
%\begin{tabular}{@{}lll@{}}
%\toprule
%         & Topics                                                & Dataset or Case                                    \\ \midrule \midrule
%\rowcolor{blue!20}     \multicolumn{1}{|l|}{1-2}   & \multicolumn{1}{l|}{Overview of Data Science}        & \multicolumn{1}{l|}{Household Data}                \\ \midrule
%\rowcolor{purple!20} 
%\multicolumn{1}{|l|}{3-5}   & \multicolumn{1}{l|}{Data Visualization}              & \multicolumn{1}{l|}{Global Super Store }       \\ \midrule
%\rowcolor{blue!20} 
%\multicolumn{1}{|l|}{6}     & \multicolumn{1}{l|}{Introduction to R/ JMP}          & \multicolumn{1}{l|}{}                              \\ \midrule
%\rowcolor{purple!20} 
%\multicolumn{1}{|l|}{7}     & \multicolumn{1}{l|}{Regression Analysis}             & \multicolumn{1}{l|}{Display \& Liquor Sales} \\ \midrule
%\rowcolor{blue!20} 
%\multicolumn{1}{|l|}{8}     & \multicolumn{1}{l|}{Multiple Regression}             & \multicolumn{1}{l|}{}                              \\ \midrule
%\rowcolor{purple!20} 
%\multicolumn{1}{|l|}{9}     & \multicolumn{1}{l|}{Dealing with Nominal Covariates} & \multicolumn{1}{l|}{Gender Divide}                 \\ \midrule
%\rowcolor{blue!20} 
%\multicolumn{1}{|l|}{10}    & \multicolumn{1}{l|}{Regression Diagonistics}         & \multicolumn{1}{l|}{}                              \\ \midrule
%\rowcolor{purple!20} 
%\multicolumn{1}{|l|}{11-12} & \multicolumn{1}{l|}{Project Presentations}            &\multicolumn{1}{l|}{}          \\\midrule \bottomrule
%\end{tabular}
%\end{table}
%}
%\end{frame}


%\begin{frame}{Case Study }
%\qBoxCol{teal!40}{\vspace{1in}\begin{center}\sqBullet{teal} \Large Case: Liquor sales and display space \end{center}
%\vspace{1in}
%}\\
%\end{frame}







\section{ Sample Space \& Events}

\TransitionFrame[antiquefuchsia]{\Large Sample Space \& Events  }






\begin{frame}
	\frametitle{ Sample Space \& Events}
	\begin{center}
	\begin{definition}[Random Experiment]
A process of observation whose outcome is not known in
advance with certainty.\\
\vspace{.1in}
\end{definition}
\vspace{-.2in}
\qBrd[4in]{teal!40}{
\HLTW{Experiment:}Single throw of a 6-sided die.
}
\vspace{.5in}
\pause

\begin{definition}[Outcome]
An outcome is defined as any possible result of a random
experiment.\\
\vspace{.2in}
\end{definition}
\vspace{-.2in}
\qBrd[4in]{applegreen!40}{
\HLTW{Experiment:}Single throw of a 6-sided die.  \\\HLTW{\text{An Outcome:}}The number 5 appear in the die-throw example.
}
	\vspace{2in}
	\end{center}
\end{frame}





\begin{frame}
	\frametitle{ Sample Space}
		\begin{center}
	\begin{definition}[Sample Space]
The set, S, of all possible outcomes of a particular experiment is called the sample space for the experiment. \\
\vspace{.2in}
\end{definition}
\vspace{-.2in}
\qBrd[4in]{applegreen!40}{
\HLTW{Experiment:}Single throw of a 6-sided die. 
}\\
\vspace{.01in}
\pause
\qBrd[4in]{amethyst!40}{
\HLTW{\text{Sample Space:}} $S= \{1, 2, 3, 4, 5, 6\}$
}
\vspace{3in}

	\end{center}
\end{frame}






\begin{frame}
	\frametitle{  Events}
		\begin{center}
	
\begin{definition}[Events]
An event is any collection of possible outcomes of a particular experiment, that is, any subset of $S$ (including  $\emptyset$ and $S$  itself  ). \\
\vspace{.2in}
\end{definition}
\vspace{-.2in}
\qBrd[4in]{applegreen!40}{
\HLTW{Experiment:}Single throw of a 6-sided die. \\\HLTW{\text{Sample Space:}} $S= \{1, 2, 3, 4, 5, 6\}$
}\\
\vspace{.01in}
\pause
\qBrd[4in]{brightpink!40}{
\HLTW{\text{Example of Events:}} $\HLTY{A=\{2,4,6\}},  \HLTY{B=\{3\}}$
}\\
{\tiny All Possible Events?}
\vspace{3in}
	\end{center}
\end{frame}


\begin{frame}
	\frametitle{Example }
	\begin{center}
\qbx[4.5in]{apricot!40}{
\Exmpl{apricot}{}
\HLTW{\text{Experiment:}} Determination of and recording of the sex of a newborn child.
}
\vspace{.1in}
\pause
\qBrd[4.5in]{amethyst!40}{
\HLTW{\text{Sample Space:}} $S=\{B, G\} $
where the outcome $G$ means that the child is a girl and $B$ that
it is a boy.}

\end{center}
\vspace{2in}

\end{frame}




\begin{frame}
	\frametitle{Example }
	\begin{center}
\qbx[4.5in]{antiquefuchsia!40}{
\Exmpl{antiquefuchsia}{}  Consider a context of horse race where 7 horses have participated the race. They are marked as $1, 2, \ldots, 7$. \\\HLTW{\text{Experiment:}  }Recording the order of the horse numbers of 7 horses according to their completion time. The  positions for the horses can be  1, 2, 3, 4, 5, 6, and 7.
}
\vspace{.1in}
\pause
\qBrd[4.5in]{babyblue!40}{
\HLTW{\text{Sample Space:}} $S =$ All 7! permutations of (1, 2, 3, 4, 5,6,7)
}
\end{center}
{\tiny 
An outcome (2,3, 1, 6, 5, 4, 7) means, for instance, that the
number 2 horse comes in first, then the number 3 horse, then
the number 1 horse, and so on.

}

\qBrd[4.5in]{darksalmon!70}{ \Qn Let $\HLTY{A}$ be 
the event that horse 3 wins the race.   Write down the explicit description of A.   }

\vspace{2.5in}

\end{frame}




\begin{frame}
	\frametitle{Example }
	\begin{center}
	
\qbx[4.5in]{atomictangerine!40}{ \Exmpl{atomictangerine}{}Consider the {\bf single flip } of a coin. 
\\\HLTW{\text{Experiment:}  } Recording the  outcome after flipping a  coin
}
\vspace{.1in}
\pause
\qBrd[4.5in]{bananamania!80}{
\HLTW{\text{Sample Space:}} $S =\{H, T\}$
}



\end{center}
\vspace{2.5in}

\end{frame}


\begin{frame}
	\frametitle{Example }
	\begin{center}
	\vspace{-.1in}
\qbx[4.5in]{atomictangerine!60}{ \Exmpl{atomictangerine}{}Consider the  flipping of {\bf  two coins.} 
\\\HLTW{\text{Experiment:}  } Recording the  outcome after flipping {\bf two coins}.
}
\vspace{.01in}
\pause
\qBrd[4.5in]{bananamania!90}{
\HLTW{\text{Sample Space:}}\\ $S ={\tiny\{H, T\}\times \{H, T\}}=\HLTY{ \{(H,H), (H,T), (T, H), (T,T)\}}$
}\\
\pause
\vspace{.01in}
\qBrd[4.5in]{applegreen!90}{
As there is no ambiguity in this case,   for brevity of notation we often/will use the following notation:\\
\HLTW{\text{Sample Space:}} $\HLTY{S = \{HH, HT, TH, TT\}}$
}\\
\vspace{.1in}
\qBrd[4.5in]{darksalmon!70}{ \Qn Let $\HLTY{B}$ be 
the event that  the  Head appears on the first coin.Write down the explicit description of B. }

\end{center}
\vspace{2.5in}

\end{frame}





\begin{frame}
	\frametitle{Example }
	\begin{center}
	
\qbx[4.5in]{amethyst!40}{ \Exmpl{amethyst}{}Consider the  rolling of  a dice {\bf two times  } 
\\\HLTW{\text{Experiment:}  } Recording the  outcome after rolling  a dice {\bf two times}.
}
\vspace{.1in}
\pause
\qBrd[4.5in]{ceil!40}{
\HLTW{\text{Sample Space:}}The sample space consists of the 36 points\\
 {\tiny S =\{1,2,3,4,5,6\}$\times $\{1,2,3,4,5,6\}}\\
 $S=\HLTY{ \{(i, j):i,j=1,2,3,4,5,6 \}}$
 {\tiny where the outcome $(i, j)$ is said to occur if $i$ appears on the first through and  j on the second.
other die.}
}
\\
\vspace{.1in}
\qBrd[4.5in]{darksalmon!70}{ \Qn Let $\HLTY{E}$ be 
the event that the sum of the dice equals 7.Write down the explicit description of E. }


\end{center}
\vspace{2.5in}

\end{frame}






\begin{frame}
	\frametitle{Example }
	\begin{center}
	
\qbx[4.5in]{applegreen!40}{ \Exmpl{applegreen}{} Consider the experiment in which we  measure (in hours) the lifetime of a transistor.
\\\HLTW{\text{Experiment:}  } (in hours) the lifetime of a transistor.
}
\vspace{.1in}
\pause
\qBrd[4.5in]{ceil!40}{
\HLTW{\text{Sample Space:}} The sample space consists of
all non-negative real numbers; that is,
 $S=\HLTY{ \{x\in \R: x>0   \}}=\HLTY{\R_{+}}.$
}
\\
\pause
\vspace{.1in}
\qBrd[4.5in]{darksalmon!70}{ \Qn Let $\HLTY{A}$ be 
the event that  the transistor does not last longer than 5 hours.  Write down the event A in the notation of set theory.  }


\end{center}
\vspace{2.5in}

\end{frame}




\TransitionFrame[antiquefuchsia]{\Large Reminder:  Disjoint Events and Partition }


\begin{frame}\frametitle{ Disjoint Events \& Partition}


\begin{definition}[Pairwise Disjoint Events]
Two events $A$ and $B$ are disjoint (or mutually exclusive) if $A\cap B=\emptyset$.  A collection of events  $\{A_i\}_{i=1}^{n}$ are pairwise disjoint (or mutually exclusive) if $A_i \cap A_j=\emptyset$ for all $i \neq j$.
\end{definition}
\pause
\define{Partition}{ $A_1,A_2, \ldots,  A_n$ are called partition of the sample space $S$ if $\{A_i\}_{i=1}^{n}$ is pairwise disjoint and $\HLTY{ \displaystyle \bigcup_{i=1}^{n} A_i=S.}$}
\pause
\qBrd[4.7in]{teal!40}{\HLTW{
Comment:} Any set $A$ and  it's complement,  $\Not{A}$,  creates  a partition of $S$.}
\pause
\vspace{-.15in}
\qBrd[4.7in]{amethyst!40}{\HLTW{
Comment:}In the above definition, we may replace $n$ by $\infty$ and the definition naturally extends. }

\end{frame}




\section{ The Notion of Probability}
\TransitionFrame[bittersweet]{\Large The Notion of Probability  }


\begin{frame}\frametitle{Basic Notion of Probability}
\begin{itemize}
\item {\bf Probability} refers to the chance that a particular event will
occur.
\vspace{.1in}
\item The probability of an event is the proportion of times the
event is expected to occur in repeated experiments.
\vspace{.1in}
\item If we denote by $n(E)$ the number of times in the first $n$
repetitions of the experiment that the event $E$ occurs, the
probability of the event $E$ is defined by $$P(E) =\lim_{n\mapsto1}
\frac{n(E)}{n}.$$
\end{itemize}
\end{frame}



\begin{frame}\frametitle{Axiomatic Definition of Probability}

\define{Probability}{
Consider an experiment whose sample space is $S$. For each
event $E$ of the sample space $S$, we assume that a number
$P(E)$ is defined and satisfies the following three axioms
\begin{enumerate}
\item $0\leq P(E)\leq 1$
\item $ P(S)= 1$
\item For any sequence of mutually exclusive events $E_1, E_2, E_3, \ldots ,  $ ((that is, events for which $E_i \cap E_j= \emptyset$ fpr all $i\neq j$)
$$ P\left(\bigcup_{i=1}^{\infty}E_i \right) = \sum_{i=1}^{\infty}P\left(E_i \right)   $$
\end{enumerate}
We refer to $P(E)$ as the probability of the event $ E$.
}
\vspace{1.5in}

\end{frame}


\begin{frame}
\qbx[4.6in]{teal!40}{
\HLTW{Comment: } \\
\vspace{.05in}
\qBrd[4.4in]{applegreen!40}{{\bf Axiom 1} states that the probability of an event E is always  between 0 and 1. }\\
\vspace{.05in}
\qBrd[4.4in]{olive!40}{{\bf Axiom 2} Probability of the entire sample space is 1. }\\
\vspace{.05in}
\qBrd[4.4in]{lime!40}{{\bf Axiom 3} states that, for any sequence
of mutually exclusive events, the probability of at least one of these
events occurring is just the sum of their respective probabilities.
}}
\end{frame}

\section{ A Few Properties of Probability}



\begin{frame}{A Few Properties of Probability}
Let $(S , P )$ be a sample space along with the Probability measure. Let $A, B$ are two events.  Then 
\begin{itemize}
\item $P(\emptyset)=0$ where $\emptyset$ denotes the Null set. 
\item $P(A)\leq 1$. 
\item $P(\Not{A})=1-P(A)$.
\item If $A\subset B $ then $P(A)\leq P(B) $.
\item  $P(A\cup B)= P(A)+P(B)-P(A\cap B)$. 
\end{itemize}

\vspace{2in}

\end{frame}


\begin{frame}
\vspace{-.2in}
\qBox{ \Qn: Represent the probability of the following events using $P(A), P(B)$ and $P(A\cap B)$. 
\begin{itemize}
\item $P(A- B)= P(A\cap \Not{B})=?$
\item $P(B- A)=?$
\item If $A, B$ are disjoint, i.e. $A\cap B=\emptyset$ then what is $P(A\cup B) = ?$
\end{itemize}
 }
 \vspace{.5in}
 
 \qBox{\Qn:
Let $A, B$ are two events such that  $P(A) = \frac{1}{3}$ and $P(\Not{B}) = \frac{1}{4}$.  Can A and B be disjoint? Explain.}
\end{frame}


\section{Examples}
\TransitionFrame[bittersweet]{\Large Examples  }

\begin{frame}{ Examples}

\qbx[4.5in]{bronze!40}{
\Exmpl{bronze}{} A smoke detector system uses two devices, A and B. If smoke is present, the probability that
it will be detected by device A is 0.95; by device B,0.90; and by both devices, 0.88.
\begin{enumerate}
\item If smoke is present, find the probability that the smoke will be detected by either device A or B or both devices.
\item Find the probability that the smoke will be undetected.
\end{enumerate}
 }

\vspace{1in}

{\tiny
Solution: A: \{device A detects smoke\}   B: \{device B detects smoke\}

$P(A\cup B)= P(A)+P(B)=P(A\cap B)= 0.95+0.90-0.88=.97$

$P(\text{Smoke is undetected })=1- P(\text{Smoke is detected by atleast one devices })  =  1-P(A\cap B)=1-0.97=0.03$

}

\end{frame}



%__________________________________________
\begin{frame}{Inclusion Exclusion Principle: Specific Case  $n=3$}
\qBox{
Let $A_1, A_2, A_3$ are three events.  Then 
\begin{eqnarray*}
 P\left(A_1\cup A_2 \cup A_3\right) 
 & =&  \left\{ \HLTEQ{P(A_1)+P(A_2)+P(A_3)}\right\}  \nonumber\\
 & &-  \left\{\HLTEQ[lightBlueOne]{ P(A_1\cap A_2) + P(A_1\cap A_3) + P(A_2\cap A_3) }  \right\}\\
 & & + \left\{ \HLTEQ{P(A_1\cap A_2 \cap A_3)}     \right\}  
\end{eqnarray*}  
}


\end{frame}
%__________________________________________
\begin{frame}{Inclusion Exclusion Principle}

\begin{lemma}
Let  $\{A_i\}_{i=1}^{n}$ be a sequence of events, for all $i=1, 2, 3, \ldots, n $. Then 
$$  P\left(\bigcup_{i=1}^{n}A_i \right)= \sum_{\HLTY{k}=1}^{n } \sum_{\HLTEQ[white]{\HLTEQ[lightBlueTwo]{(i_1,i_2,\ldots,{\HLTY{i_k}})\in \mathbb{Q}_{n,k}}}} (-1)^{\HLTY{k+1}}P\left(\bigcap_{m=1}^{k} A_{i_m} \right) ,  $$
where   $ \HLTEQ[white]{\HLTEQ[lightBlueTwo]{\mathbb{Q}_{n,k}:= \left\{(i_1, i_2\ldots, i_k)\in \mathbb{Z}_{+}^k: 1\leq i_1<i_2<\ldots<i_{k}\leq n \right\} }}.$ 
\end{lemma}
$\mathbb{Q}_{4, 2}=\left\{   \HLTEQ{(1, 2), (1, 3), (1, 4)},  \HLTEQ[lightBlueOne]{(2, 3), (2, 4)},  \HLTEQ[lightGreenTwo]{(3, 4)}   \right\},$
$\mathbb{Q}_{4, 3}=\left\{   \HLTEQ{(1, 2, 3), (1, 3, 4)},  \HLTEQ[lightGreenTwo]{(2, 3, 4)}\right\},$\\
$$\HLTEQ[lightBrownOne]{ \HLTEQ[white]{  \mathbb{Q}_{3, 1}=?,\mathbb{Q}_{3, 2}=?}}$$



\end{frame}





\TransitionFrame[bittersweet]{\Large Examples  }


\begin{frame}\frametitle{Finite Sample Spaces with Equally Likely Outcomes}

\qBrd[4.6in]{amethyst!40}{\sqBullet{amethyst} In many experiments, it is natural to assume that all outcomes in the sample space are equally likely to occur. }\\
\vspace{.05in}
\qBrd[4.6in]{olive!40}{\sqBullet{olive}  That is, consider an experiment whose sample space S is a finite set, say,  $S=\{1, 2, 3, \ldots, N\}.$ Then it is often natural to assume that  $P(\{1\})=P(\{2\})=\ldots =P(\{N\})$  which implies, from
Axioms 2 and 3, that $P(\{i\})=\frac{1}{N}$ for all $i=1,\ldots N$. }\\
\vspace{.05in}
\qBrd[4.7in]{babyblue!40}{\sqBullet{babyblue}  In equally-likely setup, it follows from Axiom 3 that, for any event $E$,  $$P(E)=\frac{\text{Number of outcomes in E}}{\text{Number of outcomes in S}}.$$
\vspace{-.2in}
}
\vspace{.05in}
\qBrd[4.6in]{applegreen!40}{\sqBullet{applegreen}  In words, if we assume that all outcomes of an experiment are equally likely to occur, then the probability of any event $E$ equals the proportion of outcomes in the sample space that are contained in $E$.
}
\end{frame}



\begin{frame}\frametitle{Example}
\qbx[4.5in]{amethyst!40}{
\Exmpl{amethyst}{} If two dice are rolled, what is the probability that the sum of the upturned faces will equal 7?
}\\
\pause
\vspace{2in}
{\tiny Solution: We shall solve this problem under the assumption that all of the 36 possible outcomes are equally likely.  Since there are 6 possible outcomes namely,  (1; 6); (2; 5); (3; 4), (4; 3); (5; 2), and (6; 1)that result in the sum of the dice being equal to 7, the desired probability is$ \frac{6}{36} =\frac{1}{6}.$}
\vspace{2in}

\end{frame}








\begin{frame}\frametitle{Example}
\qbx[4.5in]{applegreen!40}{
\Exmpl{applegreen}{} If 3 balls are randomly drawn from a bowl
containing 6 white and 5 black balls, what is the probability
that one of the balls is white and the other two black?
}\\
\pause
\vspace{1.5in}
{\tiny Solution: $n=11 \times 10\times 9 = 990$ and $n(E)=?$ if
$E=\{\text{one of the balls is white and the other two black}\}$
For the order WBB we have $6 \times 5 \times 4=120$
possibilities, for BWB, we have $5 \times 6 \times 4=120$ possibilities and for BBW we have $ 5\times 4 \times 6=120$
possibilities. Then $P(E)=\frac{n(E)}{n}=\frac{ 120+120+120}{990} = \frac{4}{11}$.}
\vspace{2in}

\end{frame}








\begin{frame}\frametitle{Example}
\qbx[4.5in]{darksalmon!40}{
\Exmpl{darksalmon}{} A committee of 5 is to be selected from a group of 6 men and 9 women.  If the selection is made randomly, what is the probability that the committee consists of 3 men and 2 women?
}\\
\pause
\vspace{1.5in}
{\tiny Solution:  Because each of the ${15 \choose 5}$ possible committees is equally likely to be selected, the desired
probability is $\frac{{6 \choose 3}\times {9 \choose 2}}{{15 \choose 5}}=0.24$}
\vspace{2in}

\end{frame}




\begin{frame}\frametitle{Example}
\qbx[4.5in]{darkraspberry!40}{
\Exmpl{darkraspberry}{} Suppose 5 people are to be randomly selected from a group of 20 individuals consisting of 10 married couples, and we want to determine $P(N)$, the probability that the 5 chosen are all unrelated. (That is, no two are married to each other.)
}\\
\pause
\vspace{1.5in}
{\tiny 
Solution: 
$P(N)=\frac{{10 \choose 5}\times 2^5}{{20 \choose 5}}\implies P(N)=\frac{20 \times 18\times 16\times 14\times 12}{20\times 19\times 18\times 17\times 16}$
}
\vspace{2in}

\end{frame}




\begin{frame}\frametitle{Example}
\qbx[4.5in]{darktangerine!40}{
\Exmpl{darktangerine}{} If $n$ people are present in a room, what is the probability that no two of them celebrate their birthday on the same day of the year? How large need n be so that this probability is less than $\frac{1}{2}$?
}\\
\pause
\vspace{1.5in}
{\tiny 
Solution: 
$\frac{^365P_n}{(365)^n}=\frac{(365)\times (364) \times \cdots \times (365-n+1)}{(365)^n}$
}
\vspace{2in}

\end{frame}




\begin{frame}\frametitle{Example}
\qbx[4.5in]{cherryblossompink!40}{
\Exmpl{cherryblossompink}{}A poker hand consists of 5 cards. If the cards have distinct consecutive values and are not all of the same suit, we say that the hand is a straight. For instance, a hand consisting of the five of spades, six of spades, seven of spades, eight of spades, and nine of hearts is a straight.  What is the probability that one is dealt a straight?
}\\
\pause
\vspace{1.5in}
{\tiny 
Solution: 
?
}
\vspace{2in}

\end{frame}



\begin{frame}\frametitle{Example}
\qbx[4.5in]{capri!40}{
\Exmpl{capri}{}In the game of bridge, the entire deck of 52 cards is
dealt out to 4 players. What is the probability that
\begin{enumerate}[a).]
\item one of the players receives all 13 spades?
\item each player receives 1 ace?
\end{enumerate}
}\\
\pause
\vspace{1.5in}
{\tiny 
Solution: 
?
}
\vspace{2in}

\end{frame}




\begin{frame}{ Examples: On Sharing a Birthday  }

\qBox{\Qn:
There are 15 students registered in the course STAT320.  What is the probability that at least two of the students will share their Birthday? (Ignore the leap year and assume there is 365 days in a year) 
}


\pause
\vspace{1.5in}
{\tiny Solution: 
$P(\{\text{At least two students have same birthday}\})=1-P(\text{None of the students have same birthday})=1-\frac{365\times 364\times \ldots \times 351}{365^{15}}\approx 1-0.7470987=0.25  $
}


\end{frame}




\begin{frame}{Application of Inclusion Exclusion Principle}

\qBox{\Qn: Suppose we turn over cards simultaneously from two well shuffled decks of ordinary playing cards. We say we obtain an exact match on a particular turn if the same card appears from each deck; for example, the queen of spades against the
queen of spades.  Let $C_i$ denotes the event that there is an exact match at the $i^{\text{th}}$ turn.   
\begin{enumerate}
\item Argue that   $P(C_i)=\frac{51!}{52!}.$ 
\item  Argue that  for $i\neq j $,  $P(C_i \cap C_j)= \frac{50!}{52!}$. 
\item  Argue that  for $i\neq j \neq k $,  $P(C_i \cap C_j \cap C_k)= \frac{49!}{52!}$. 
\item Let $p_M$ equal the probability of at least one exact match. Show that 
$p_M:= 1-\frac{1}{2!}+\frac{1}{3!}-\cdots -\frac{1}{52!}$.
\end{enumerate}
}
%\qBox{\Qn:
%Find the number of nonnative integer solutions of $X_1, X_2, X_3, $  such that $X_1+X_2+X_3=14$. \\
%\HLT[yellow]{Ans: 120}
%}
%
%\qBox{\Qn:
% Consider an Experiment,  where a  dice is thrown 4 times.
%Calculate the probability that the sum of all the through is 17?
%}
\end{frame}
\TransitionFrame[antiquefuchsia]{\Large Questions?  }
 
 
\end{document}
