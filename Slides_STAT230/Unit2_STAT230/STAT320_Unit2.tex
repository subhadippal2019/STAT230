\documentclass[compress]{beamer}
\mode<presentation>
\setbeamercovered{transparent}
\usetheme{Warsaw}
%\useoutertheme{smoothtree}
\usepackage{multirow}
\usepackage[english]{babel}
\usepackage[latin1]{inputenc}
\usepackage{times}
\usepackage[T1]{fontenc}
\usepackage{xmpmulti}
\usepackage{multicol}
\usepackage{colortbl}



%\setbeamersize{text margin left=.25 in,text margin right=.25 in}
\setbeamersize{text margin left=.15 in,text margin right=.15 in}
\usepackage[authoryear]{natbib}


\usepackage{epstopdf}
\usepackage{xcolor}
\usepackage{latexcolors}
%\usepackage[dvipsnames]{xcolor}
\definecolor{antiquebrass}{rgb}{0.8, 0.58, 0.46}
\definecolor{babyblueeyes}{rgb}{0.63, 0.79, 0.95}
\definecolor{babyblue}{rgb}{0.54, 0.81, 0.94}
\definecolor{bistre}{rgb}{0.24, 0.17, 0.12}
\definecolor{brightlavender}{rgb}{0.75, 0.58, 0.89}
\definecolor{bulgarianrose}{rgb}{0.28, 0.02, 0.03}
\definecolor{slateblue}{rgb}{0.56, 0.74, 0.56}
\definecolor{cordovan}{rgb}{0.54, 0.25, 0.27}
\definecolor{darkbyzantium}{rgb}{0.36, 0.22, 0.33}

\setbeamercolor{structure}{fg=bittersweet!90, bg= black!60}







\usepackage{tikz}
\usetikzlibrary{shadows,calc}
\usetikzlibrary{shadows.blur}
\usetikzlibrary{shapes.symbols}
\usepackage{hyperref}
\usepackage{booktabs}
\usepackage{colortbl}
\usepackage{multirow}
%%%%%%%%% shaddow image %%%%%
% some parameters for customization
\def\shadowshift{3pt,-3pt}
\def\shadowradius{6pt}
\colorlet{innercolor}{black!60}
\colorlet{outercolor}{gray!05}
% this draws a shadow under a rectangle node
\newcommand\drawshadow[1]{
\begin{pgfonlayer}{shadow}
    \shade[outercolor,inner color=innercolor,outer color=outercolor] ($(#1.south west)+(\shadowshift)+(\shadowradius/2,\shadowradius/2)$) circle (\shadowradius);
    \shade[outercolor,inner color=innercolor,outer color=outercolor] ($(#1.north west)+(\shadowshift)+(\shadowradius/2,-\shadowradius/2)$) circle (\shadowradius);
    \shade[outercolor,inner color=innercolor,outer color=outercolor] ($(#1.south east)+(\shadowshift)+(-\shadowradius/2,\shadowradius/2)$) circle (\shadowradius);
    \shade[outercolor,inner color=innercolor,outer color=outercolor] ($(#1.north east)+(\shadowshift)+(-\shadowradius/2,-\shadowradius/2)$) circle (\shadowradius);
    \shade[top color=innercolor,bottom color=outercolor] ($(#1.south west)+(\shadowshift)+(\shadowradius/2,-\shadowradius/2)$) rectangle ($(#1.south east)+(\shadowshift)+(-\shadowradius/2,\shadowradius/2)$);
    \shade[left color=innercolor,right color=outercolor] ($(#1.south east)+(\shadowshift)+(-\shadowradius/2,\shadowradius/2)$) rectangle ($(#1.north east)+(\shadowshift)+(\shadowradius/2,-\shadowradius/2)$);
    \shade[bottom color=innercolor,top color=outercolor] ($(#1.north west)+(\shadowshift)+(\shadowradius/2,-\shadowradius/2)$) rectangle ($(#1.north east)+(\shadowshift)+(-\shadowradius/2,\shadowradius/2)$);
    \shade[outercolor,right color=innercolor,left color=outercolor] ($(#1.south west)+(\shadowshift)+(-\shadowradius/2,\shadowradius/2)$) rectangle ($(#1.north west)+(\shadowshift)+(\shadowradius/2,-\shadowradius/2)$);
    \shade[outercolor,right color=innercolor,left color=innercolor] ($(#1.north west)+(-\shadowradius/12,\shadowradius/12)$) rectangle ($(#1.south east)+(\shadowradius/12,-\shadowradius/12)$);%Frame
    \filldraw ($(#1.south west)+(\shadowshift)+(\shadowradius/2,\shadowradius/2)$) rectangle ($(#1.north east)+(\shadowshift)-(\shadowradius/2,\shadowradius/2)$);
\end{pgfonlayer}
}
% create a shadow layer, so that we don't need to worry about overdrawing other things
\pgfdeclarelayer{shadow} 
\pgfsetlayers{shadow,main}
% Define image shadow command
\newcommand\shadowimage[2][]{%
\begin{tikzpicture}
\node[anchor=south west,inner sep=0] (image) at (0,0) {\includegraphics[#1]{#2}};
\drawshadow{image}
\end{tikzpicture}}
\usepackage{calligra}

\DeclareMathOperator*{\argmax}{Arg\,max}
\DeclareMathOperator*{\argmin}{Arg\,min}
\newcommand{\norm}[1]{\left\Vert #1 \right\Vert }
\newcommand{\bbetaHat}{ \widehat{\bbeta}}
\newcommand{\bbetaLSE}{ \widehat{\bbeta}_{_{\text{LSE}}}}
\newcommand{\bbetaMLE}{ \widehat{\bbeta}_{_{\text{MLE}}}}
\newcommand{\sqBullet}[1]{  {\tiny \tiny \tiny \qBoxCol{#1!60}{ }} }
%***************
%\newtheorem{thm}{Theorem}
%\documentclass[noinfoline]{imsart}
%\usepackage{amsmath,amstext,amssymb}
%%\usepackage[top=1.5in, bottom=1.5in, left=1.2in, right=1.2in]{geometry}
%% settings
%%\pubyear{2005}
%%\volume{0}
%%\issue{0}
%%\firstpage{1}
%%\lastpage{8}
%\arxiv{arXiv:0000.0000}
\usepackage{subcaption}
%\startlocaldefs
%\numberwithin{equation}{section}
%\theoremstyle{plain}
%\newtheorem{thm}{Theorem}
%\endlocaldefs
\usepackage{lipsum} 
\usepackage{amsmath}
\usepackage{amssymb}
\usepackage{amsbsy} 
\usepackage{amsthm}
\usepackage{mathrsfs}
%\usepackage{eufrak}
\usepackage{mathrsfs}
\usepackage{color}
\usepackage{verbatim}
\usepackage{graphicx}
\usepackage{bm}
\usepackage{enumerate}
\usepackage{epstopdf} 
\usepackage{natbib}
\usepackage{undertilde}

\usepackage{tfrupee}

\usepackage{tikz}
\usetikzlibrary{shadows,calc}
\usetikzlibrary{shadows.blur}
\usetikzlibrary{shapes.symbols}
%%%%%%%%% shaddow image %%%%%
\usepackage{calligra}

%\newcommand{\logLik}{\text{\calligra l}\,}
%\usepackage{calligra,amsmath,amssymb}

\usepackage{mathrsfs}
\DeclareMathAlphabet{\mathpzc}{OT1}{pzc}{m}{it} 
% \newcommand{\logLik}{ \mathpzc{l}}
 \newcommand{\logLik}{ \mathbb{\ell}_{_n}}
  \newcommand{\Lik}{ \mathcal{L}_{_n}}
  \newcommand{\score}{\mathpzc{S}_{_n}}
  %\newcommand{\Finfo}{1}{ \mathpzc{I}_{#1}}
  \NewDocumentCommand{\Finfo}{O{ }}{ \mathcal{I}_{_{#1}}}
\newcommand{\Bias}[1]{  \text{Bias}\left(#1\right)   }
\newcommand{\Var}[1]{  \text{Var}\left(#1\right)  }
\newcommand{\Mse}[1]{  \text{Mse}\left(#1\right)}   

\newcommand{\gCalli}{\text{\calligra g}\,}
% some parameters for customization
\def\shadowshift{3pt,-3pt}
\def\shadowradius{6pt}

\colorlet{innercolor}{black!60}
\colorlet{outercolor}{gray!05}

% this draws a shadow under a rectangle node
\newcommand\drawshadow[1]{
    \begin{pgfonlayer}{shadow}
        \shade[outercolor,inner color=innercolor,outer color=outercolor] ($(#1.south west)+(\shadowshift)+(\shadowradius/2,\shadowradius/2)$) circle (\shadowradius);
        \shade[outercolor,inner color=innercolor,outer color=outercolor] ($(#1.north west)+(\shadowshift)+(\shadowradius/2,-\shadowradius/2)$) circle (\shadowradius);
        \shade[outercolor,inner color=innercolor,outer color=outercolor] ($(#1.south east)+(\shadowshift)+(-\shadowradius/2,\shadowradius/2)$) circle (\shadowradius);
        \shade[outercolor,inner color=innercolor,outer color=outercolor] ($(#1.north east)+(\shadowshift)+(-\shadowradius/2,-\shadowradius/2)$) circle (\shadowradius);
        \shade[top color=innercolor,bottom color=outercolor] ($(#1.south west)+(\shadowshift)+(\shadowradius/2,-\shadowradius/2)$) rectangle ($(#1.south east)+(\shadowshift)+(-\shadowradius/2,\shadowradius/2)$);
        \shade[left color=innercolor,right color=outercolor] ($(#1.south east)+(\shadowshift)+(-\shadowradius/2,\shadowradius/2)$) rectangle ($(#1.north east)+(\shadowshift)+(\shadowradius/2,-\shadowradius/2)$);
        \shade[bottom color=innercolor,top color=outercolor] ($(#1.north west)+(\shadowshift)+(\shadowradius/2,-\shadowradius/2)$) rectangle ($(#1.north east)+(\shadowshift)+(-\shadowradius/2,\shadowradius/2)$);
        \shade[outercolor,right color=innercolor,left color=outercolor] ($(#1.south west)+(\shadowshift)+(-\shadowradius/2,\shadowradius/2)$) rectangle ($(#1.north west)+(\shadowshift)+(\shadowradius/2,-\shadowradius/2)$);
        \filldraw ($(#1.south west)+(\shadowshift)+(\shadowradius/2,\shadowradius/2)$) rectangle ($(#1.north east)+(\shadowshift)-(\shadowradius/2,\shadowradius/2)$);
    \end{pgfonlayer}
}

% create a shadow layer, so that we don't need to worry about overdrawing other things
\pgfdeclarelayer{shadow} 
\pgfsetlayers{shadow,main}

\newsavebox\mybox
\newlength\mylen

\newcommand\shadowimage[2][]{%
\setbox0=\hbox{\includegraphics[#1]{#2}}
\setlength\mylen{\wd0}
\ifnum\mylen<\ht0
\setlength\mylen{\ht0}
\fi
\divide \mylen by 120
\def\shadowshift{\mylen,-\mylen}
\def\shadowradius{\the\dimexpr\mylen+\mylen+\mylen\relax}
\begin{tikzpicture}
\node[anchor=south west,inner sep=0] (image) at (0,0) {\includegraphics[#1]{#2}};
\drawshadow{image}
\end{tikzpicture}}

%\begin{document}
%
%\noindent\shadowimage[width=6cm]{image}\par\bigskip

%%%%%%%%%%%%%%%%%%%%%%%



%\RequirePackage[colorlinks,citecolor=blue,urlcolor=blue]{hyperref}
%\usepackage{subfig}
\usepackage[final]{pdfpages}

\usepackage{algorithm}  %@subhajit
\usepackage{algpseudocode} %@subhajit
\usepackage{algorithmicx}     %@subhajit
\usepackage{undertilde}


\newcommand{\sphere}{{\mathbb{S}}}
\newcommand{\R}{\mathbb{R}}
\newcommand{\LatentV}{V}
\newcommand{\NC}{m}
\newcommand{\Priorf}{f_{prior}}
\newcommand{\FWOne}[2]{{{}_{1}\Psi _{1}\left[{\begin{matrix}(\frac{#1}{2},\frac{1}{2})\\(1,0)\end{matrix}};#2\right]} 
}


\newcommand{\HyPriorMu}{\thetabf}
\newcommand{\HyPriorAlpha}{\alpha}
\newcommand{\HyPriorBeta}{\beta}
\newcommand{\HyPriorK}{\zeta}
\newcommand{\Indicator}[1]{\mathbb{I}({#1 })}
\newcommand{\IndicatorA}[2]{\mathbb{I}_{#2}({#1 })}
\newcommand{\xb}{\bm{x}}
\newcommand{\bx}{\bm{x}}



\newcommand{\bX}{\bm{X}}
\newcommand{\by}{\bm{y}}
\newcommand{\bZ}{\bm{Z}}
\newcommand{\bF}{\bm{F}}
\newcommand{\btheta}{\bm{\theta}}
\newcommand{\Bpi}{\boldsymbol{\pi}}
\newcommand{\thetabf}{\boldsymbol{\theta}}
\newcommand{\Thetabf}{\boldsymbol{\Theta}}
\newcommand{\taubf}{\boldsymbol{\tau}}
\newcommand{\Tr}{Tr}
\newcommand{\HaarMu}{\mu}
\newcommand{\RestMu}{\mu_{\delta}}
\newcommand{\ConstOne}{K}

\newcommand{\bM}{\bm{M}}
\newcommand{\bD}{\utilde{\bm{D}}}
\newcommand{\bV}{\bm{V}}
\newcommand{\loglikmix}{\mathcal{L}_{\bM,\bD,\bV}}
\newcommand{\hypdc}{{}_0F_1\left(\frac{n}{2},\frac{D_c^2}{4}\right)}


\usepackage{xstring}
\usepackage[normalem]{ulem}
\definecolor{ultramarine}{RGB}{38,29,163}
\newcommand\PalDel[1]{{\color{red} {\sout{#1}}}}
\newcommand\Pal[1]{{\color{ultramarine}{#1}}}
\newcommand\PalRp[2]{\PalDel{#1} \Pal{#2}}
\newcommand\PalCmnt[1]{{\color{ultramarine} {[[[***PAL:  #1 ***]]]}}}

\newcommand{\qedwhite}{\hfill \ensuremath{\Box}}
\newcommand{\SpaceD}{\mathcal{S}_p}
\newcommand{\SpaceM}{\widetilde{\mathcal{V}}_{n,p}}
\newcommand{\SpaceV}{\mathcal{V}_{p,p}}
\newcommand{\SpaceF}{\mathbb{R}^{n,p}}
\newcommand{\StiefelS}{\mathcal{V}_{n,p}}
\newcommand{\SpacePi}{\mathbb{S}_{\pi}}
\newcommand{\ML}{{\cal{ML}}}
\newcommand{\ProdSpace}{\boldsymbol{\Theta}}
\newcommand{\ThetaAndPi}{\Xi}
\newcommand{\ClassML}{\mathcal{C}_{\ML}}
\newcommand{\balpha}{\bm{\alpha}}
\newcommand{\bbeta}{\bm{\beta}}
\newcommand{\bEta}{\bm{\eta}}
\newcommand{\bd}{{\utilde{\bm{d}}}}
\newcommand{\BoEta}{{\utilde{\boldsymbol{\eta}}}}
%\newtheorem{theorem}{Theorem}[section]
%\newtheorem{theorem}{Theorem}
%\newtheorem{lemma}{Lemma}
%\newtheorem{result}{Result}
\newtheorem{defn}{Definition}

\newcommand{\define}[2]{ \begin{definition}[#1]  #2  \end{definition}  }

\newcommand{\pdv}[2]{\frac{\partial#1}{\partial#2}}
\newcommand{\pdvtwo}[2]{\frac{\partial^2#1}{{\partial#2}^2}}


\newcommand{\mubf}{\boldsymbol{\mu}}
\newcommand{\sumI}{ \sum_{i=1}^{n}}
\newcommand{\Ybar}{{\overline{Y}}}

\newcommand{\Expectation}[1]{\mathbb{E}{[#1]}}
\newcommand{\priorXzero}{\Psi}
\newcommand{\iMat}{\mathbf{I}_{p}}

% 
% \newtheorem{thm}{Theorem}[section]
% \newtheorem{cor}[thm]{Corollary}
% \newtheorem{lem}[thm]{Lemma}
%\newtheorem{proposition}{Proposition}

%\newtheorem{theorem}{Theorem}[chapter]%To link the theorem to each chapter uncomment the chapter option
%\newtheorem{lemma}{Lemma}%[theorem]% To link each lemma to a theorem uncomment the theorem option
%\newtheorem{corollary}{Corollary}%[theorem]% To link each corollary to a theorem uncomment the theorem option
% to link a corollary to a chapter change the theorem option to chapter
%\newtheorem{definition}{Definition}%[chapter] %the same is true for both definitions and assumptions
\newtheorem{assumption}{Assumption}%[chapter] %
%\newtheorem{proposition}{Proposition}[chapter]
%\newtheorem{fact}{Fact} %%% added by @subho
\newcommand{\StrongNBD}[2]{S_{#1}{#2}}
\newcommand{\bpi} {\boldsymbol{\pi}}
\newcommand{\bphi} {\boldsymbol{\phi}}
\newcommand{\bb}[1]{\boldsymbol{#1}}
% Definitions of handy macros can go here

\newcommand{\normtwo}[1]{{\left\lVert#1\right\rVert}_2}

\newcommand{\dataset}{{\cal D}}
\newcommand{\fracpartial}[2]{\frac{\partial #1}{\partial  #2}}
\newcommand{\Lesbegue}[1]{\mu_{\btheta_{#1},\bpi_{#1}}}
\newcommand{\fthetapi}[1]{f_{\btheta_{#1},\bpi_{#1}}}
% Heading arguments are {volume}{year}{pages}{submitted}{published}{author-full-names}
\newcommand{\doublehat}[1]{%
    \settoheight{\dhatheight}{\ensuremath{\widehat{#1}}}%
    \addtolength{\dhatheight}{-0.35ex}%
    \widehat{\vphantom{\rule{2pt}{\dhatheight}}%
    \smash{\hspace{-0.5mm}\widehat{#1}}}}

\newcommand{\hyp}{{}_0F_1\left(\frac{n}{2},\frac{D^2}{4}\right)}
\newcommand{\hypinline}{{}_0F_1\left({n}/{2},{D^2}/{4}\right)}

\newcommand{\partialhyp}[1]{\frac{\partial}{\partial\,{d_{#1}}}\,\left[\hyp\right]}

\newcommand{\fracProbZ}[1]{\frac{\langle Z_{ic} \rangle \, #1}{\sum_{i=1}^{N} \langle Z_{ic}\rangle  } }
\newcommand{\EmVar}[1]{\widetilde{#1}^{(c)}}

\newcommand{\IMDY}{{\it{CCPD}}}
\newcommand{\JMDY}{{\it{JCPD}}}

\newcommand{\DYlang}{\frac{\exp(\nu\,\bEta^T\bd)}{{\left[{}_0F_1\left(\frac{n}{2},\frac{D^2}{4}\right)\right]}^{\nu}}}

\newcommand{\logDYlang}{\nu\,\bEta^T\bd - \nu\,\log\left({}_0F_1\left(\frac{n}{2},\frac{D^2}{4}\right)\right)}

\newcommand{\lhyp}{\log\left({}_0F_1\left(\frac{n}{2},\frac{D^2}{4}\right)\right)}

%\jmlrheading{1}{2000}{1-48}{4/00}{10/00}{SS \& JH \& AB}

% Short headings should be running head and authors last names

%\ShortHeadings{BDP and cIBP}{SS \& JH \& AB}
%\firstpageno{1}

\newcommand{\diam}[1]{{{#1}^{\ast}}}

%%% coloring option %%%
\definecolor{auburn}{rgb}{0.53, 0.1, 0.5}
\newcommand{\sss}{\color{auburn}}  %%% for Subhajit
\newcommand{\sse}{\color{black}}
\newcommand{\attn}{\color{red}}
\newcommand{\rms}{\color{magenta}}  %%% for Riten
\newcommand{\rme}{\color{black}}
\newcommand{\MLDensity}{f_{\ML}}
\setlength{\parindent}{0cm}
\newcommand{\posterior}

\newcommand{\variableX}{\bd}
\newcommand{\funch}{\mathfrak{h}}
\newcommand{\IndVzero}[1]{\mathbb{I}({X\in \mathcal{V}^{#1}_0})}
\newcommand{\Rnp}{\mathbb{R}^{n \times p}}
\newcommand{\Rpp}{\mathbb{R}^{p \times p}}
\newcommand{\vecnorm}[1]{\lVert #1\rVert}

\newcommand{\etapsiD}{\eta_{\priorXzero}}
\newcommand{\BoEtapsiD}{\BoEta_{\priorXzero}}

\newcommand{\DMp}{\mathcal{D}^{p \times p}}
\newcommand{\Rplus}{\mathbb{R}_{+}}
\newcommand{\prodMeasure}{\Upsilon}

\newcommand{\m}{{\bf m_{\BoEta}}} 
\newcommand{\SetWithMode}{\mathcal{S}}
\newcommand{\SetWithModePrime}{\mathcal{S}}
\newcommand{\TargetComp}{\mathcal{S}^{\star}}

\newcommand{\ConstCondDen}{K_{\nu, \BoEta}} 

\newcommand{\hyparam}[2]{
    \IfEqCase{#1}{
        {M}{\xi^{#2}_c}
        {V}{\gamma^{#2}_c}%
        
    }
  }
\newcommand{\threepartdef}[6]
{
	\left\{
		\begin{array}{lll}
			#1 & \mbox{if } #2 \\
			#3 & \mbox{if } #4 \\
			#5 & \mbox{if } #6
		\end{array}
	\right.
}

\def\bv{\color{blue}}
\def\ev{\color{black}}
\newcommand{\bch}{\bv }
\newcommand{\ech}{\ev\normalsize}
%\newcommand{\MakeVec}[1]{{\utilde{\bf #1}}}
\newcommand \Measure[2][]{%
  \ifstrempty{#1}{
  \IfEqCase{#2}{
        {M}{\mu}%
        {D}{\mu_1}%
        {V}{\mu_2}
        {X}{\mu}
   }  
  }{
  \IfEqCase{#1}{
  {1}{
   \IfEqCase{#2}{
        {M}{d\mu(M)}%
        {D}{d\mu_1(\bd)}%
        {V}{d\mu_2(V)}
        {X}{d\mu(X)}
        {Y}{d\mu(Y)}
        {MDV} {d\mu(M)\; d\mu_1(\bd) \;d\mu_2(V) }
        }
   } 
   {2}{
   \IfEqCase{#2}{
         {M}{d\mu(M^{\ast})}%
        {D}{d\mu_1(\bd^{\ast})}%
        {V}{d\mu_2(V^{\ast})}
        {X}{d\mu(X^{\ast})}
        }
   }
   {3}{
   \IfEqCase{#2}{
         {M}{\mu(dM^{\star})}%
        {D}{\mu_2(d\bd^{\star})}%
        {V}{\mu_1(dV^{\star})}
        {X}{\mu(X^{\star})}
        }
   }   
   
   } 
  }%
}
  \newcommand{\VONF}{\text{VonMisesFisher}}
\newcommand{\MPGalpha}{\alpha}
\newcommand{\MPGnu}{\nu}
\newcommand{\MPG}{MPG }
\newcommand{\ybin}{y^{(\text{bin})}}


\usepackage{caption}
\usepackage{subcaption}


\newcommand{\nullSet}{\Phi}
\newcommand{\SP}{S}
\newcommand{\B}{ \mathcal{B}}
\newcommand{\prob}[1]{P\left( #1 \right)}
\newcommand{\Qn}{{\bf Question:}}
\newcommand{\Cmt}{{\bf Comment:}}




\newcommand{\support}{\mathcal{S}}
\newcommand{\tht}{\text{th}}
\newcommand{\abs}[1]{ \left\vert  #1 \right\vert }
\newcommand{\var}{\text{Var}}

\newcommand{\TwoColFunction}[2]{
\left\{
\begin{array}{ll}
#1 & \text{ if } #2\\
0 & \text{ otherwise. }
\end{array}
\right.
}
%%%%%%%%%%%%%%%%%%%%%%%%%%%
\newcommand{\vnsp}{\vspace{-.2in}}
\newcommand{\Cmnt}{{\bf Comment}}
\newcommand{\Eqn}[1]{ \vspace{-.15in} $$ {\HLTEQ{ \displaystyle  #1 }}\vspace{-.1in}$$   }


\newcommand{\sampleX}[1]{X_1, X_2, \ldots , X_{#1}}
\newcommand{\sampleY}[1]{Y_1, Y_2, \ldots , Y_{#1}}
\newcommand{\sampleZ}[1]{Z_1, Z_2, \ldots , Z_{#1}}
\newcommand{\sampleGen}[2]{{#2}_1, {#2}_2, \ldots , {#2}_{#1}}

\newcommand{\Xbar}{\overline{X}}
\newcommand{\Zbar}{\overline{Z}}
\newcommand{\Ubar}{\overline{U}}
\newcommand{\Vbar}{\overline{V}}
\newcommand{\Wbar}{\overline{W}}


\renewcommand{\bX}{\MakeVec{\bf X}}
\newcommand{\bY}{\MakeVec{\bf Y}}
\renewcommand{\bx}{\MakeVec{\bf x}}
\renewcommand{\by}{\MakeVec{\bf y}}


\newcommand{\pHat}{\widehat{p}}
\newcommand{\qHat}{\widehat{q}}
%\usepackage{xcolor}
\usepackage{xcolor}
\usepackage{xparse}
\definecolor{lightGray}{gray}{0.95}
\definecolor{lightGrayOne}{gray}{0.9}
\definecolor{lightBlueOne}{RGB}{204, 255, 255}
\definecolor{lightBlueTwo}{RGB}{204, 238, 255}
\definecolor{lightBlueThree}{RGB}{204, 204, 255}
\definecolor{AltBlue}{RGB}{119,14,161}


\definecolor{BGBlue}{RGB}{220,221,252}
\definecolor{BGBlueOne}{RGB}{204,229,255}



\definecolor{BGGreen}{RGB}{240,243,245}
\definecolor{lightGreenOne}{RGB}{179, 255, 179}
\definecolor{lightGreenTwo}{RGB}{198, 255, 179}
\definecolor{lightGreenThree}{RGB}{243, 255, 230}
\definecolor{AltGreen}{RGB}{193, 240, 240}

\definecolor{BOGreen}{RGB}{180,0,0}
\definecolor{BGGreenOne}{RGB}{220,250,220}

\definecolor{lightBrownOne}{RGB}{255, 221, 204}
\definecolor{lightBrownTwo}{RGB}{255, 229, 204}
\definecolor{lightBrownThree}{RGB}{242, 217, 230}


\definecolor{HLTGreen}{RGB}{230,244,215}
\definecolor{ExcBrown}{RGB}{153,0,0}
\definecolor{ExcBGBrown}{RGB}{255,204,204}
\definecolor{BGYellowOne}{RGB}{255,235,208}
\definecolor{BGPink}{RGB}{255,215,240}



\NewDocumentCommand{\HLT}{ O{HLTGreen} m }{\colorbox{#1}{#2}}
\NewDocumentCommand{\HLTEQ}{ O{HLTGreen} m }{\colorbox{#1}{$#2$}}

%\newcommand{\HLT}[1]{\colorbox{HLTGreen}{#1}}
\newcommand{\DEHLT}[1]{\colorbox{lightGrayOne}{\color{white} #1}}

\newcommand{\TextInBoxOne}[2]{  {\fcolorbox{lightGrayOne}{white}{\begin{minipage}{#1}  #2 \end{minipage}}}}

\newcommand{\TextInBoxOneQ}[2]{  {\fcolorbox{white}{lightGrayOne}{\begin{minipage}{#1}  #2 \end{minipage}}}}

\newcommand{\TextInBoxOneEQ}[2]{  {\fcolorbox{white}{lightBlueTwo}{\begin{minipage}{#1}  #2 \end{minipage}}}}

\newcommand{\QuizQuestion}[3]{  {\fcolorbox{black}{white}{\begin{minipage}{5.6 in}
\TextInBoxOneEQ{5.5in}{ #1 }\\
{\large \HLTEQ{\hspace{4.61in}\frac{\text{Score: \;\;\;\;}}{\text{#3}}}}\\
\vspace{.01in}#2 \end{minipage}}}}

\newcommand{\QuizQ}[3]{  {\fcolorbox{black}{lightGrayOne}{\begin{minipage}{5.6 in}
\TextInBoxOne{5.5in}{ #1 }\\
\vspace{.01in}#2 \end{minipage}}}}



\newcommand{\ExamQuestion}[3]{  {\fcolorbox{lightBlueTwo}{lightBlueTwo}{\begin{minipage}{5.85 in}
\TextInBoxOne{5.8in}{ #1 }\\
{\large \HLTEQ[lightBlueTwo]{\hspace{5.01in}\frac{\text{Score: \;\;\;\;}}{\text{#3}}}}\\
\end{minipage} }
#2 }}


\NewDocumentCommand{\MCOption}{O{1.75 in}m}{
\TextInBoxTwo[BGPink]{ #1 } {\TextInBoxTwo[white]{.1 in }{ \quad}\HLT{#2}}
}




\NewDocumentCommand{\MCOptionSelected}{m}{
\TextInBoxTwo[BGPink]{ 1.75 in } {\TextInBoxTwo[white]{.1 in }{{\huge $\bullet$}}\HLT{#1}}
}


%
%\NewDocumentCommand{\MCOption}{m}{
%\TextInBoxTwo[white]{.1 in }{ \quad}\HLT{#1}}







\NewDocumentCommand{\TextInBoxTwo}{ O{lightGrayOne} m m } {{\fcolorbox{white}{#1}{\begin{minipage}{#2} { #3} \end{minipage}}}}


\newcommand{\TextInBox}[2]{  {\fcolorbox{BGGreen}{BGGreen}{\begin{minipage}{#1}  #2 \end{minipage}}}}
\newcommand{\TextInBoxCol}[2]{
\fcolorbox{BGBlue}{BGBlue}{%
\begin{minipage}{#1}
 {\color{AltBlue} #2}
\end{minipage}}%
}




\newcommand{\BandInTopBox}[2]{
\fcolorbox{AltBlue}{AltBlue}{%
\begin{minipage}{#1}{ {\color{white}  #2 \hspace{.1in}} }
\end{minipage}}%
}


\newcommand{\TextInBoxThm}[2]{
\fcolorbox{AltBlue}{lightGray}{%
\begin{minipage}{#1}
 {\color{black} #2}
\end{minipage}}%
}

\newcommand{\TextInBoxThmOne}[2]{
\fcolorbox{BGBlue}{BGBlueOne}{%
\begin{minipage}{#1}
 {\color{AltBlue} #2}
\end{minipage}}%
}

\newcommand{\TextInBoxLem}[2]{
\fcolorbox{BGBlue}{lightGray}{%
\begin{minipage}{#1}
 {\color{black} #2}
\end{minipage}}%
}



\newcommand{\TextInBoxLemOne}[2]{
\vspace{.02 in}
\noindent
\fcolorbox{BGBlue}{BGBlue}{%
\begin{minipage}{#1}
 {\color{AltBlue} #2}
\end{minipage}}%
}


\newcommand{\CmntBox}[1]{
\noindent
\TextInBoxLem{5.3 in }{
\TextInBoxLemOne{5.2 in }{
#1
}}

}

\newcommand{\DefBox}[1]{
%\vspace{.1 in}
\noindent
\TextInBoxLem{6 in }{
\BandInTopBox{5.9 in }{}
\TextInBoxLemOne{5.9 in }{
#1
}}}


\newcommand{\DefBoxL}[1]{
%\vspace{.1 in}
\noindent
\TextInBoxLem{8 in }{
\BandInTopBox{7.9 in }{}
\TextInBoxLemOne{7.9 in }{
#1
}}}




%Old measurements
%\newcommand{\DefBoxOne}[2]{
%%\vspace{.1 in}
%\noindent
%\TextInBoxLem{6 in }{
%\BandInTopBox{5.9 in }{#1}
%\TextInBoxLemOne{5.9 in }{
%#2
%}}}
%

\newcommand{\DefBoxOne}[2]{
%\vspace{.1 in}
\noindent
\TextInBoxLem{6.8 in }{
\BandInTopBox{6.7 in }{#1}
\TextInBoxLemOne{6.7 in }{
#2
}}}


\newcommand{\ThmBox}[2]{
\noindent
\TextInBoxThm{6.8 in }{
\TextInBoxThmOne{6.7 in }{
#1}
#2}
}

\newcommand{\LemBox}[2]{
\noindent
\TextInBoxLem{6.8 in }{
\TextInBoxLemOne{6.7 in }{
#1}
#2}
}

\newcommand{\PropBox}[2]{
\vspace{.1 in}
\noindent
\TextInBoxLem{6.8 in }{
\TextInBoxLemOne{6.7 in }{
#1}
#2}
}




\newcommand{\TextInBoxExc}[2]{
\noindent
\fcolorbox{white}{BGGreenOne}{%
\begin{minipage}{#1}
 {\color{black} #2}
\end{minipage}}%
}


\newcommand{\TextInBoxExample}[2]{
\noindent
\fcolorbox{white}{BGPink}{%
\begin{minipage}{#1}
 {\color{black} #2}
\end{minipage}}%
}


\newcommand{\ExerciseBox}[1]{
\noindent
%\TextInBoxLem{6 in }{
\TextInBoxExc{6 in }{
#1}
%#2}
}


\newcommand{\ExampleBox}[1]{
\noindent
%\TextInBoxLem{6 in }{
\TextInBoxExample{6 in }{
#1}
%#2}
}


\newcommand{\IndicatorA}[2]{\mathbb{I}_{#2}({#1 })}


 


\newcommand \rbind[1]{%
    \saveexpandmode\expandarg
    \StrSubstitute{\noexpand#1}{,}{&}[\fooo]%
    %\StrSubstitute{\fooo}{,}{&}[\fooo]%
    \StrSubstitute{\fooo}{;}{\noexpand\\}[\fooo]%
    \StrSubstitute{\fooo}{:}{\noexpand\\}[\fooo]%
    \restoreexpandmode
   \left[ \begin{matrix}\fooo\end{matrix}\right]
    }
    
    
    
   \newcommand \ColVec[1]{%
    \saveexpandmode\expandarg
    \StrSubstitute{\noexpand#1}{,}{\noexpand\\}[\fooo]%
    %\StrSubstitute{\fooo}{,}{&}[\fooo]%
    \StrSubstitute{\fooo}{;}{\noexpand\\}[\fooo]%
    \StrSubstitute{\fooo}{:}{\noexpand\\}[\fooo]%
    \restoreexpandmode
   \left[ \begin{matrix}\fooo\end{matrix}\right]
    }
     \newcommand \RowVec[1]{%
    \saveexpandmode\expandarg
    \StrSubstitute{\noexpand#1}{,}{&}[\fooo]%
    %\StrSubstitute{\fooo}{,}{&}[\fooo]%
    \StrSubstitute{\fooo}{;}{&}[\fooo]%
    \StrSubstitute{\fooo}{:}{&}[\fooo]%
    \restoreexpandmode
   \left[ \begin{matrix}\fooo\end{matrix}\right]
    }



  \newcommand \Row[1]{%
    \saveexpandmode\expandarg
    \StrSubstitute{\noexpand#1}{,}{&}[\fooo]%
    %\StrSubstitute{\fooo}{,}{&}[\fooo]%
    \StrSubstitute{\fooo}{;}{&}[\fooo]%
    \StrSubstitute{\fooo}{:}{&}[\fooo]%
    \restoreexpandmode
    \begin{matrix}\fooo\end{matrix}
    }
        
    
    
    
    \newcommand \Col[1]{%
    \saveexpandmode\expandarg
    \StrSubstitute{\noexpand#1}{,}{\noexpand\\}[\fooo]%
    %\StrSubstitute{\fooo}{,}{&}[\fooo]%
    \StrSubstitute{\fooo}{;}{\noexpand\\}[\fooo]%
    \StrSubstitute{\fooo}{:}{\noexpand\\}[\fooo]%
    \restoreexpandmode
    \begin{matrix}\fooo\end{matrix}
    }

%%%%%%%%%%%%%%%%%%%%% Experimental %%%%%%%%%%%%%%%%%


\ExplSyntaxOn
\DeclareExpandableDocumentCommand{\replicate}{O{}mm}
 {
  \int_compare:nT { #2 > 0 }
   {
    {#3}\prg_replicate:nn {#2 - 1} { #1#3 }
   }
 }
\ExplSyntaxOff


\ExplSyntaxOn
\DeclareExpandableDocumentCommand{\repdiag}{O{}mm}
 {
  \int_compare:nT { #2 > 0 }
   {
    {\prg_replicate:nn {#2}{#3#1}}{#3}
   }
 }
\ExplSyntaxOff


\newcommand \StrRowDiag[1]{%
    \saveexpandmode\expandarg
    \StrSubstitute{\noexpand#1}{,}{&}[\fooo]%
    %\StrSubstitute{\fooo}{,}{&}[\fooo]%
    \StrSubstitute{\fooo}{;}{&}[\fooo]%
    \StrSubstitute{\fooo}{:}{&}[\fooo]%
    \StrCount{\fooo}{&}[\countfooo]
    \restoreexpandmode
    \repdiag[0]{\countfooo+1}{{,}}
   %\left[ \begin{matrix}\fooo\end{matrix}\right]
    }


\newcommand \DiagStrOne[2]{%
    \saveexpandmode\expandarg
    \StrSubstitute{\noexpand#1}{,}{\noexpand#2}[\fooo]%
    \restoreexpandmode
   %\left[ \begin{matrix}\fooo\end{matrix}\right]
   \fooo
    }
    
    \newcommand \DiagStr[1]{%
    \DiagStrOne{#1}{{\StrRowDiag{#1}}}
    }


%$\rbind{\replicate[,]{10}{\Col{\replicate[;]{7}{0}}}}$

%$\Col{1,2,3}$
%$\ColVec{\replicate[;]{5}{B}}$
%$ \StrRowDiag{1,2} $

%$\DiagStr{1,2,3}$

%\repdiag[-]{3}{A}
\ExplSyntaxOn
\NewDocumentCommand{\Split}{ m m o }
 {
  \tarass_split:nn { #1 } { #2 }
  \IfNoValueTF { #3 } { \tl_use:N } { \tl_set_eq:NN #3 } \l_tarass_string_tl
 }

\tl_new:N \l_tarass_string_tl

\cs_new_protected:Npn \tarass_split:nn #1 #2
 {
  \tl_set:Nn \l_tarass_string_tl { #2 }
  % we need to start from the end, so we reverse the string
  \tl_reverse:N \l_tarass_string_tl
  % add a comma after any group of #1 tokens
  \regex_replace_all:nnN { (.{#1}) } { \1\, } \l_tarass_string_tl
  % if the length of the string is a multiple of #1 a trailing comma is added
  % so we remove it
  \regex_replace_once:nnN { \,\Z } { } \l_tarass_string_tl
  % reverse back
  \tl_reverse:N \l_tarass_string_tl
 }
\ExplSyntaxOff

%%%%%%%%%%%%%%%%%%%%%%%%%%%%%%%%

\newcommand{\ShowRowMatrix}[3]{ \left[ {\begin{array}{ccc}
  \line(1,0){22} &{#1} &  \line(1,0){22} \\
     & \vdots& \\
  \line(1,0){22}  &{#2}& \line(1,0){22} \\
   &  \vdots & \\
    \line(1,0){22} &{#3}& \line(1,0){22}  \\
    \end{array}
   } \right]}
 


\newcommand{\ShowColMatrix}[3]{ \left[ {\begin{array}{ccccc}
  \line(0,1){25} & &\line(0,1){25} &  &  \line(0,1){25} \\
  {#1}  & \ldots & {#2} &\ldots   &{#3} \\
 \line(0,1){25} &  & \line(0,1){25}  &  &  \line(0,1){25} \\
    \end{array}
   } \right]}
   
   
   
   
\newcommand{\ShowRowVector}[1]{ \left[ {\begin{array}{ccc}
  \line(1,0){25} &{#1} &  \line(1,0){25} 
    \end{array}
   } \right]}   
   
   
\newcommand{\ShowColVector}[1]{ \left[ {\begin{array}{c}
  \line(0,1){25} \\    {#1} \\   \line(0,1){25}     \end{array}  } \right]}
  
\newcommand{\ColVector}[3]{ \left[ {\begin{array}{c}
  {#1}\\ \vdots \\    {#2}\\ \vdots\\{#3}  \end{array}  } \right]}
  
  
  
  
  
\newcommand{\EqSetThree}[3]{ \left\{ {\begin{array}{c}
  {#1}\\ \vdots \\    {#2}\\ \vdots\\{#3}  \end{array}  } \right.}  
  



\newcommand{\MatrixTypeA}[3]{ \left[ {\begin{array}{ccc}
 {#1}_{1,1} & \cdots & {#1}_{1,{#3}}   \\
  {#1}_{2,1} & \cdots & {#1}_{2,{#3}}   \\
    \vdots  & \ddots& \vdots  \\
     {#1}_{{#2},1} & \cdots & {#1}_{{#2},{#3}}   \\
    \end{array}
   } \right]}
 
\newcommand{\MatrixTypeAKronecker}[4]{ \left[ {\begin{array}{ccc}
 {#1}_{11}{#4} & \cdots & {#1}_{1{#3}}{#4}   \\
  {#1}_{21} {#4} & \cdots & {#1}_{2{#3}} {#4}   \\
    \vdots  & \ddots& \vdots  \\
     {#1}_{{#2}1} {#4} & \cdots & {#1}_{{#2}{#3}} {#4}   \\
    \end{array}
   } \right]}
 



\newcommand{\ShowIMat}{ {\begin{array}{cccc}
 1&  &  &    \\
  & 1 &  &  \\
    &  & \ddots &    \\
     & & & 1   \\
    \end{array}
   } }
 
\newcommand{\ShowVecOne}{
\begin{array}{c}
 1\\ 1 \\    1  
\end{array}
}

 
\newcommand{\ShowUnitVecOne}{
\begin{array}{c}
 1\\ 0 \\   0  
\end{array}
}


\newcommand{\ShowUnitVecTwo}{
\begin{array}{c}
 0\\ 1 \\   0  
\end{array}
}


\newcommand{\ShowUnitVecThree}{
\begin{array}{c}
 0\\ 0\\   1  
\end{array}
}

\newcommand{\ShowZeroThree}{
\begin{array}{c}
 0\\ 0\\   0 
\end{array}
}


\newcommand{\TwoBlockMatrix}[2]{\left[  {\begin{array}{c;{2pt/2pt}c}
   {#1} &  {#2}
   \end{array} }\right]}
   
   \newcommand{\TwoBlockMatrixH}[2]{\left[  {\begin{array}{c}
   {#1} \\
   \hdashline[2pt/2pt]
    {#2}
   \end{array} }\right]}
   
   \newcommand{\TwoBlockH}[2]{ {\begin{array}{c}
   {#1} \\
   \hdashline[2pt/2pt]
    {#2}
   \end{array} }}
   
   
\newcommand{\TwoBlock}[2]{ {\begin{array}{c;{2pt/2pt}c}
   {#1} &  {#2}
   \end{array} }}
   

      
   
   
   
 \newcommand{\ThreeBlockColVec}[3]{
   \left[ {\begin{array}{c}
  #1\\
  \hdashline[2pt/2pt]\\
   \vdots\\
  \hdashline[2pt/2pt]\\
  #2\\
  \hdashline[2pt/2pt]\\
   \vdots\\
  \hdashline[2pt/2pt]\\
   #3\\
    \end{array}
   } \right]
   }



\NewDocumentCommand{\ColDyn}{>{\SplitList{;}}m}
   {
     \left[\begin{array}{c}
       \ProcessList{#1}{ \inserColtitem }
     \end{array}\right]
   }
   \newcommand \inserColtitem[1]{ #1 \\}


\makeatletter
\newcommand{\ColDynAlt}[2][r]{%
  \gdef\@VORNE{1}
  \left[\hskip-\arraycolsep%
    \begin{array}{#1}\vekSp@lten{#2}\end{array}%
  \hskip-\arraycolsep\right]}

\def\vekSp@lten#1{\xvekSp@lten#1;vekL@stLine;}
\def\vekL@stLine{vekL@stLine}
\def\xvekSp@lten#1;{\def\temp{#1}%
  \ifx\temp\vekL@stLine
  \else
    \ifnum\@VORNE=1\gdef\@VORNE{0}
    \else\@arraycr\fi%
    #1%
    \expandafter\xvekSp@lten
  \fi}
\makeatother


\NewDocumentCommand{\eVec}{m O{}}{\MakeVec{e}_{#1, (#2)}}

\NewDocumentCommand{\Ones}{O{3}}{\Col{\replicate[,]{#1}{1}}}
\NewDocumentCommand{\Zeros}{O{3}}{\Col{\replicate[,]{#1}{0}}}











\title{  STAT 320: Principles of Probability\\ {\color{black}  Unit 2: A Few Counting Principles \& and Their Applications}}

\author[UAEU]
{United Arab Emirates University}
\institute[UAEU] % (optional, but mostly needed)
{
  \inst{Department of Statistics}%
  %Indian Institute of Management,  Udaipur\\
  \vspace{0.1in}

  
}

\date{}


\newcommand{\Xnew}{ \HLTEQ[orange]{X_{_{\text{i}}}} }
\newcommand{\Ynew}{ \HLTEQ[orange]{Y_{_{\text{i}}}} }

%\date{\today}

\AtBeginSection[]
{
  \begin{frame}{Inhalt}
 % \begin{multicols}{1}
	\frametitle{Outline}
    \tableofcontents[currentsection]
  %  \end{multicols}
  \end{frame}
}

\begin{document}
\maketitle

%\begin{frame}{Outline}
%%\begin{multicols}{}
%  \tableofcontents
%%\end{multicols}
%\end{frame}

%\section{Introduction to DSBA 2023}
%
%
%\begin{frame}
%\qBoxCol{blue!30}{
%\begin{center} Course  Website \end{center}
%\qbx[4.2in]{teal!40}{\sqBullet{teal} \color{blue} $ \href{https://sites.google.com/iimu.ac.in/dsba2023e/home}{https://sites.google.com/iimu.ac.in/dsba2023e/home}$
%}\\
%\qbx[3.0in]{green!40}{ \sqBullet{green} Regular Announcements.
%}\\
%\qbx[3.0in]{olive!40}{\sqBullet{olive}  Slides and other materials.
%}
%}
%
%\pause
%\qBoxCol{blue!30}{
%\sqBullet{blue}
%You can contact the instructor at {\it subhadip.pal@iimu.ac.in} and schedule for office hours.  
%}
%\pause
%\qBoxCol{olive!30}{
%\sqBullet{olive}
%Mr. Praveen Kumar has been assigned as Teaching Assistant (TA) for this course.  His email I'd is:  {\it praveen.kumar@iimu.ac. }
%}
%
%
%\end{frame}
%


%
%\begin{frame}{Course Outline}
%\hspace{-.1in}\qBoxCol{blue!35}{
%% Please add the following required packages to your document preamble:
%% \usepackage{booktabs}
%\begin{table}[]
%\begin{tabular}{@{}lll@{}}
%\toprule
%         & Topics                                                & Dataset or Case                                    \\ \midrule \midrule
%\rowcolor{blue!20}     \multicolumn{1}{|l|}{1-2}   & \multicolumn{1}{l|}{Overview of Data Science}        & \multicolumn{1}{l|}{Household Data}                \\ \midrule
%\rowcolor{purple!20} 
%\multicolumn{1}{|l|}{3-5}   & \multicolumn{1}{l|}{Data Visualization}              & \multicolumn{1}{l|}{Global Super Store }       \\ \midrule
%\rowcolor{blue!20} 
%\multicolumn{1}{|l|}{6}     & \multicolumn{1}{l|}{Introduction to R/ JMP}          & \multicolumn{1}{l|}{}                              \\ \midrule
%\rowcolor{purple!20} 
%\multicolumn{1}{|l|}{7}     & \multicolumn{1}{l|}{Regression Analysis}             & \multicolumn{1}{l|}{Display \& Liquor Sales} \\ \midrule
%\rowcolor{blue!20} 
%\multicolumn{1}{|l|}{8}     & \multicolumn{1}{l|}{Multiple Regression}             & \multicolumn{1}{l|}{}                              \\ \midrule
%\rowcolor{purple!20} 
%\multicolumn{1}{|l|}{9}     & \multicolumn{1}{l|}{Dealing with Nominal Covariates} & \multicolumn{1}{l|}{Gender Divide}                 \\ \midrule
%\rowcolor{blue!20} 
%\multicolumn{1}{|l|}{10}    & \multicolumn{1}{l|}{Regression Diagonistics}         & \multicolumn{1}{l|}{}                              \\ \midrule
%\rowcolor{purple!20} 
%\multicolumn{1}{|l|}{11-12} & \multicolumn{1}{l|}{Project Presentations}            &\multicolumn{1}{l|}{}          \\\midrule \bottomrule
%\end{tabular}
%\end{table}
%}
%\end{frame}


%\begin{frame}{Case Study }
%\qBoxCol{teal!40}{\vspace{1in}\begin{center}\sqBullet{teal} \Large Case: Liquor sales and display space \end{center}
%\vspace{1in}
%}\\
%\end{frame}





\begin{frame}\frametitle{Counting Principles: Objective}
\qBrd[4.5in]{olive!40}{
\sqBullet{olive} In this unit, we will consider a few basic concepts of the Combinatorial analysis and Counting principles.    }\\
\vspace{.1in}
\qBrd[4.5in]{teal!40}{
\sqBullet{teal} In Probability and Statistics, when dealing with finite sample space,   counting principles provides an efficient way in obtaining probabilities of events.  }\\
\vspace{.1in}
\qBrd[4.5in]{amber!40}{
\sqBullet{amber}  Combinatorics is also used in various other disciplines of science and engineering including Graph Theory,  Computer science, etc.   }\\
\vspace{.1in}
\qBrd[4.5in]{babyblue!40}{
\sqBullet{babyblue} As it is widely used and a nontrivial topic,  we will keep our focus on the examples that are aligned with our discussion on Probability.   }



\end{frame}

\section{Multiplication Principle }
\TransitionFrame[bittersweet]{\Large Multiplication Principle  }

\begin{frame}
	\frametitle{Overview of Counting Principals  }
	
\begin{defn}[Multiplication Principle of counting]
Suppose that two experiments are to be performed.  If experiment 1 can result
in any one of the $\HLTY{m}$ possible outcomes and if, for each outcome of experiment 1, there are $\HLTY{\HLTW{n}}$ possible outcomes of experiment 2, then altogether there are a total of $\HLTY{m\times \HLTW{n}}$ possible outcomes if we consider the  two experiments together.
\end{defn}	
	\vspace{.5in}
	
	\qBrd[4.5in]{applegreen!40}{
\sqBullet{olive} If a task involves two steps.  The
first step can be completed in $m$ ways and the second step in
$n$ ways, then there are $m \times n$ ways to complete the task.}
	
	
	
\end{frame}


%\begin{frame}
%	\frametitle{Example }
%\qbx[4.5in]{babyblue!40}{
%\Exmpl{babyblue}{1} Enrollment in the course Principles of probability consists of: 28 statistics majors, and
% 53 math major students.  If  2 students are selected at random.  In how many ways can we select one math and one statistics student?
%}
%\pause
%
%{\small Solution: $53 \times 28 = 1484$.}
%\vspace{2in}
%
%\end{frame}





\begin{frame}
	\frametitle{Example }



\qbx[4.5in]{teal!40}{
\Exmpl{teal}{1} An airline has four flights (4 flight numbers) from New-York to California and five flights  (5 different flight numbers ) from California to
Hawaii per day.  If the flights are to be made on separate days, how many different flight
arrangements can the airline offer from New-York to Hawaii?}\\
%
%\define{Factorial}{
%For a positive integer $n$, $n!$ (read $n$ factorial) is the product of
%all of the positive integers less than or equal to $n$. That is,
%$n! = n x (n - 1) x (n - 2) \times  \ldots  \times  3 \times  2 \times  1.$
%Furthermore, we define $0! = 1$.}
\vspace{1.5in}
\pause
{\tiny Solution: $5\times 4=20$ }
\vspace{1in}
\end{frame}



%
%
%\begin{frame}
%	\frametitle{Example }
%\qbx[4.5in]{amethyst!40}{
%\Exmpl{amethyst}{2} Say the only clean clothes you've got are 2 t-shirts and 4 jeans.  How many different combinations can you choose? 
%}
%\pause
%
%{\small Solution: $2 \times 4 = 8$.}
%\vspace{2in}
%
%\end{frame}
%



\begin{frame}
	\frametitle{Example }
\qbx[4.5in]{babyblue!40}{
\Exmpl{babyblue}{2} According the standard medical convention, a person's {\bf blood type} consists of two symbols.   The first symbol is  a letter code that is either `A', `B', `AB', or `O'.  The second code shows the person's Rhesus (Rh) factor.  It can either be  `+' (i.e.   Rh factor +)  or `-' (i,.e.  Rh factor- ).  In a medical facility,  a technician records a person's blood type once a blood test is done.  What is the total number of different {\bf blood types} that are possible?
}
\pause
\vspace{1in}

{\tiny Solution: $4 \times 2 =8$.}
\vspace{2.5in}

\end{frame}




\begin{frame}
\begin{defn}[Generalized Multiplication Principle of Counting]
If there are $k$ steps in an operation of which first can be done in $\HLTY{n_1}$ ways,
for each of these second can be done in $\HLTY{n_2}$ ways, for each of
the first two the third step can be done in $\HLTY{n_3}$ ways, and so
forth, then the whole operation can be done in
$\HLTW{\HLTY{n_1} \times \HLTY{n_2} \times \HLTY{ n_3} \times \cdots  \times  \HLTY{n_k}}$ ways.
\end{defn}	
	\vspace{.5in}
	
%	\qBrd[4.5in]{applegreen!40}{
%\sqBullet{olive} If a task involves two steps and the
%first step can be completed in $m$ ways and the second step in
%$n$ ways, then there are $m \times n$ ways to complete the task.}
%	
	
	\vspace{2in}
\end{frame}




\begin{frame}
	\frametitle{Example }
\qbx[4.5in]{babyblue!40}{
\Exmpl{babyblue}{3} A college planning committee consists of
\begin{itemize}
\item \HLTW{\text{three}} $1^{\text{st}}$-Year,
\item \HLTW{\text{four}} $2^{\text{nd}}$-Year, 
\item\HLTW{\text{five}} $3^{\text{rd}}$- Year, and 
\item \HLTW{\text{two}} Final-Year Students
\end{itemize} 
 A subcommittee of \HLTY{4}, consisting of 1 person from each class, is to be chosen.  How many
different subcommittees are possible?
}\\
\pause
\vspace{.5in}
{\tiny {\bf Solution:} We may regard the choice of a subcommittee as the combined outcome of the four separate
experiments of choosing a single representative from each of the classes. It then follows from the generalized version of the basic principle that there are $3\times 4\times 5\times 2 = 120$ possible subcommittees.}
\vspace{2in}

\end{frame}









%
%
%\begin{frame}
%	\frametitle{Example }
%\qbx[4.5in]{babyblue!40}{
%\Exmpl{babyblue}{3} A college planning committee consists of 3 freshmen, 4 sophomores, 5 juniors, and 2 seniors.  A subcommittee of 4, consisting of 1 person from each class, is to be chosen.  How many
%different subcommittees are possible?
%}\\
%\pause
%\vspace{1.2in}
%{\tiny {\bf Solution:} We may regard the choice of a subcommittee as the combined outcome of the four separate
%experiments of choosing a single representative from each of the classes. It then follows from the generalized version of the basic principle that there are $3\times 4\times 5\times 2 = 120$ possible subcommittees.}
%\vspace{2in}
%
%\end{frame}
%
%




\begin{frame}
	\frametitle{Example }
	\vspace{-.25in}
\qbx[4.5in]{amethyst!40}{
\Exmpl{amethyst}{4} A car manufacturer provides cars with the following different variations:
\begin{itemize}
\item Manual or automatic transmission
%\item  With or without air conditioning
\item  Three different stereo systems
\item Four possible exterior colors
\end{itemize}
How many different types of car the manufacturer sells?
}\\
\includegraphics[scale=.23]{figs/cars.png} \\
\vspace{.5in}
\pause
{\tiny Solution: $2\times 3\times 4=24$ }


\end{frame}





\section{Two Important Terminologies:  }





\begin{frame}

\qBrd[4.68in]{blue!10}{
Terminology:  \HLTY{\text{Order in Arrangements}} is\\
$\Col{{\Row{\MCQ[2.1in]{
Order is Important 
} , \MCQ[2.1in]{
Order is NOT Important
}}}}$
  }

\vspace{.5in}

\qBrd[4.7in]{blue!10}{
Terminology:  \\
$\Col{{\Row{\MCQ[2.1in][babyblue!60]{ 
WITH Replacement
},\MCQ[2.1in][babyblue!60]{
WITHOUT Replacement
}}}}$ }



\end{frame}


\begin{frame}\frametitle{Example}
Rearranging the letters of a word: ``EARTH"\\
{\tiny Rearranging the letters of a word: ``TEN",  
Rearranging the letters of a word: ``SILENT"  }
\vspace{1.6in}


$\Col{{\Row{\MCQ[2.1in]{
Order is Important 
} , \MCQ[2.1in]{
Order is NOT Important
}}},{\Row{\MCQ[2.1in][babyblue!60]{ 
WITH Replacement
},\MCQ[2.1in][babyblue!60]{
WITHOUT Replacement
}}}}$


\end{frame}



\begin{frame}\frametitle{Example}
\vspace{-.1in}
How many 4 letter different `words' ({\tiny sequences of 4 letters}) can be constructed using the letters of   ``EARTH" ( {\tiny you may use same letters on multiple occasions })\\
\vspace{1.5in}


$\Col{{\Row{\MCQ[2.1in]{
Order is Important 
} , \MCQ[2.1in]{
Order is NOT Important
}}},{\Row{\MCQ[2.1in][babyblue!60]{ 
WITH Replacement
},\MCQ[2.1in][babyblue!60]{
WITHOUT Replacement
}}}}$


\end{frame}



\begin{frame}\frametitle{Example}
\vspace{-.1in}
How many different 4 digit numbers can be constructed using the digits of the number ``12345"
\vspace{1.5in}


$\Col{{\Row{\MCQ[2.1in]{
Order is Important 
} , \MCQ[2.1in]{
Order is NOT Important
}}},{\Row{\MCQ[2.1in][babyblue!60]{ 
WITH Replacement
},\MCQ[2.1in][babyblue!60]{
WITHOUT Replacement
}}}}$

\end{frame}


\begin{frame}\frametitle{Example}
How many different 4 digit numbers can be constructed?
\vspace{1.5in}


$\Col{{\Row{\MCQ[2.1in]{
Order is Important 
} , \MCQ[2.1in]{
Order is NOT Important
}}},{\Row{\MCQ[2.1in][babyblue!60]{ 
WITH Replacement
},\MCQ[2.1in][babyblue!60]{
WITHOUT Replacement
}}}}$


\end{frame}




\begin{frame}\frametitle{Example}
How many different ways one can arrange ( permute) the following 5 distinct/ nonidentical balls in a linear manner?\\
\includegraphics[scale=.35]{figs/5DifferentBalls.png}
\vspace{1in}


$\Col{{\Row{\MCQ[2.1in]{
Order is Important 
} , \MCQ[2.1in]{
Order is NOT Important
}}},{\Row{\MCQ[2.1in][babyblue!60]{ 
WITH Replacement
},\MCQ[2.1in][babyblue!60]{
WITHOUT Replacement
}}}}$


\end{frame}




\begin{frame}\frametitle{Example}
How many different ways one select 3 balls from a bunch of 5 {\bf distinct/ Non identical} balls?\\
\includegraphics[scale=.35]{figs/5BallsInBox.png}
\vspace{.1in}


$\Col{{\Row{\MCQ[2.1in]{
Order is Important 
} , \MCQ[2.1in]{
Order is NOT Important
}}},{\Row{\MCQ[2.1in][babyblue!60]{ 
WITH Replacement
},\MCQ[2.1in][babyblue!60]{
WITHOUT Replacement
}}}}$

\end{frame}





\begin{frame}\frametitle{}
\vspace{-.1in}
How many different ways one can distribute the following 6 distinct balls so that each of the kids gets two balls each?\\
\includegraphics[scale=.25]{figs/6Balls2each.png}
%\vspace{.1in}

\pause
$\Col{{\Row{\MCQ[2in]{\tiny
Order of the balls a person receives is Important
} , \MCQ[2.5in]{
\tiny 
Order of the two balls a person receives is  NOT Important
}}},{\Row{{\MCQ[2in]{\tiny 
Order of the individuals is  Important
} }, \MCQ[2.5in]{\tiny Order of the Individuals are NOT Important
}}},{\Row{\MCQ[2in][babyblue!60]{ 
WITH Replacement
},\MCQ[2.5in][babyblue!60]{
WITHOUT Replacement
}}}}$

\end{frame}






\begin{frame}\frametitle{}
How many different ways one can distribute the following 6 distinct balls so that each of the kids gets two balls each?\\
\includegraphics[scale=.35]{figs/6PurpleBass3kids.png}


$\Col{{\Row{\MCQ[2in]{\tiny
Order of the balls a person receives is Important
} , \MCQ[2.5in]{
\tiny 
Order of the two balls a person receives is  NOT Important
}}},{\Row{{\MCQ[2in]{\tiny 
Order of the individuals is  Important
} }, \MCQ[2.5in]{\tiny Order of the Individuals are NOT Important
}}},{\Row{\MCQ[2in][babyblue!60]{ 
WITH Replacement
},\MCQ[2.5in][babyblue!60]{
WITHOUT Replacement
}}}}$

\end{frame}




\begin{frame}\frametitle{}
How many different ways one can distribute the following 6 distinct balls ({\tiny consider the scenario when a kid may obtain NONE,One or More balls})  ?\\
\includegraphics[scale=.35]{figs/6PurpleBass3kids.png}



$\Col{{\Row{\MCQ[2in]{\tiny
Order of the balls a person receives is Important
} , \MCQ[2.5in]{
\tiny 
Order of the two balls a person receives is  NOT Important
}}},{\Row{{\MCQ[2in]{\tiny 
Order of the individuals is  Important
} }, \MCQ[2.5in]{\tiny Order of the Individuals are NOT Important
}}},{\Row{\MCQ[2in][babyblue!60]{ 
WITH Replacement
},\MCQ[2.5in][babyblue!60]{
WITHOUT Replacement
}}}}$

\end{frame}




\section{{\bf Ordered}  Arrangements {\bf with} Replacements }



\TransitionFrame[bittersweet]{\Large {\bf Ordered}  Arrangements of {\bf distinct} objects {\bf with} Replacements   }


\begin{frame}{Ordered Arrangements of Distinct Objects  {\bf with} Replacements }
\begin{center}
\define{Ordered Arrangements with Replacement}{
Let $\HLTW{n, r}$ be positive  integers.  Number of different ways a sequence of $\HLTW{r}$ symbols can be created from $\HLTW{n}$ distinct symbols (when repetition of the symbols are allowed) is $\HLTEQ[yellow]{n^r}$. \\
\vspace{.1in}}
\vspace{-.2in}
\qBrd[4.3in]{amethyst!30}{ \tiny You may also view the procedure as the following: Number of different ways $r-$tuples (a vector with $r$ coordinates) can be constructed where elements of each coordinates are chosen arbitrary from a set containing $n$ distinct objects.  }\\
\vspace{1.2in}


\end{center}
\end{frame}



\begin{frame}
\vspace{-.2in}
\qBrd[4.6in]{babyblue!60}{ \Exmpl{babyblue!30}{} How many different numbers can be constructed using a 32 bits binary digits? 
} 
\vspace{-.25in}
\begin{center}
\includegraphics[scale=.4]{figs/bits.png} \\
\end{center}

\vspace{1.15in}
\pause
{\tiny Solution:  \HLT[yellow]{$2^{32}$}}

\end{frame}


\begin{frame}
	\frametitle{Example }
\qbx[4.5in]{cherryblossompink!50}{
\Exmpl{cherryblossompink}{} A typical lisence plate number in Abu-Dhabi consists of 5 digits.  How many different lisence plate numbers (excluding `00000') are possible ? 
}\\
\includegraphics[scale=.2]{figs/L_Plate.png} \\
\vspace{1in}
\pause
{\tiny Solution: By the generalized version of the basic principle, the answer is $10^5-1=9999$}

\vspace{1in}
\end{frame}



\begin{frame}
\vspace{-.1in}
\qBrd[4.5in]{applegreen!30}{ \Exmpl{applegreen!50}{} In a popular  lottery game,  a person
may pick any six numbers from \{1, 2, \ldots ,44\} for her ticket.  How many different groups of six numbers can be chosen from the forty-four {\bf if the repeated selection of the numbers is allowed and the ``order of the selected are important" ?
} }

\includegraphics[scale=.25]{figs/lottery.png} \\
\vspace{.8in}
\pause
{\tiny Solution:  \HLT[yellow]{$44^6$}}

\end{frame}




\section{Permutations:{\small {\bf Ordered} Arrangements  of {\bf distinct} objects  {\bf Without} Replacements}    }


\TransitionFrame[bittersweet]{\Large Permutations:{\small {\bf Ordered} Arrangements {\bf distinct} objects {\bf Without} Replacements}    }

\begin{frame}\frametitle{Reminder: Factorial}


%\define{Factorial}{
%For a positive integer $n$, $n!$ (read $n$ factorial) is the product of
%all of the positive integers less than or equal to $n$. That is,
%$\HLTY{\HLTW{n!} = n \times  (n - 1) \times (n - 2) \times  \ldots  \times  3 \times  2 \times  1.}$
%Furthermore, we define $\HLTEQ[purple!30]{\HLTW{0!} = 1}$.}

\define{Factorial}{
Let $n$ be a {\bf non-negative integer,} then the {\bf factorial of n}, denoted as $\HLTY{n!}$ is defined to be
\begin{enumerate}
\item $\HLTEQ[purple!30]{\HLTW{0!} = 1}$,
\item $\HLTEQ[amethyst!40]{\HLTW{1!}=\HLTY{1}}$, 
\item $\HLTEQ[applegreen!40]{\HLTW{n!}= \HLTY{n\times(n-1)\times \cdots \times 1}}$ for $n\geq 2$
\end{enumerate} 
}

\vspace{.1in}
\pause
\qBrd[4.5in]{olive!30}{Example: $4!=4\times 3\times 2\times 1=24.$}
\qBrd[4.5in]{babyblue!30}{Example: $6!=6\times 5 \times 4\times 3\times 2\times 1=720.$}\\
\vspace{.05in}
\qBrd[4.5in]{amethyst!40}{\sqBullet{amethyst}Note that $\HLTW{n!}=n\times \{\HLTW{(n-1)!}\}$, for $n$ integer and  $n\geq 1$.}








\end{frame}



\begin{frame}
\vspace{-.1in}
\begin{center}
\define{Permutations (Ordered, without replacement)}{ Let $r$, and $n$ be two positive integers such that $r\leq n$.  An ordered arrangement of $r$ distinct objects is called a permutation.  The number of ways of ordering $n$ distinct objects taken $r$  at a time, denoted by the symbol $^nP_r$,  is given as   $$\HLTEQ[applegreen!50]{ \HLTW{\displaystyle  ^nP_r}=  \HLTEQ[white]{ \displaystyle n(n - 1)(n- 2)(n - r + 1) }= \HLTW{ \displaystyle \frac{n!}{(n - r )!}  }   }.$$
\vspace{.2in}
}
\vspace{-.3in}
\qBrd[4.3in]{teal!40}{Also, we may envision $^nP_r$  to be the number of different ways $r$ distinct objects can be selected from n distinct objects when the order of the choice is considered to be important.  However, the repeated selection of the same objects are not allowed. }
\end{center}
\pause
\qbx[4.5in]{ceil!50}{\sqBullet{ceil}The number of possible permutations of n distinct objects is
given by: $\HLTW{\displaystyle ^nP_n= n! =n\times (n-1) \times \cdots \times 2\times 1}$. }

\vspace{1in}
\end{frame}





\begin{frame}
\vspace{-.2in}
\qBrd[4.5in]{iceberg!30}{ \Exmpl{iceberg!70}{} Consider a lottery game where a participant selects any 6 numbers from \{1, 2, \ldots ,44\}.  How many different possibilities are there for the six numbers to if {\bf the repeated selection of the numbers are {\it NOT allowed}  while the order at which the numbers are selected is considered to be important.}
}\\
\includegraphics[scale=.25]{figs/lottery.png} \\
\pause 
\vspace{.8in}
{\tiny
Solution: \HLT[yellow]{$ ^{44}P_{6}= 44\time 43 \times 42\times 41\times 40 \times 39 $}}
\end{frame}




\begin{frame}
\qbx[4.5in]{bluebell!40}{\Exmpl{bluebell}{}
How many different batting orders are possible for a
baseball team consisting of 9 players?}\\
\vspace{2.5in}
\pause
{\tiny
 Solution: There are $9! = 362880$ possible batting orders. }
\end{frame}


\begin{frame}
\qbx[4.5in]{cambridgeblue!50}{\Exmpl{cambridgeblue}{}
How many different ``words" ({\tiny sequence of three letters that may or may not be dictionary words}) can you make with the letters
\HLTW{\text{"CAN"}}?}\\
\vspace{2.5in}
\pause
{\tiny
 Solution:  $3! = 6$ }
\end{frame}




\begin{frame}
\vspace{-.1in}
\qbx[4.5in]{carnationpink!50}{\Exmpl{carnationpink}{}
A class in probability theory consists of \HLTW{ \text{6 men}} and \HLTW{\text{4 women}}.  An examination is given, and the students are ranked
according to their performance. Assume that no two students
obtain the same score.
\begin{enumerate}[(a)]
\item How many different rankings are possible?
\item  If the men are ranked just among themselves and the women
just among themselves, how many different rankings are possible?
\end{enumerate}}\\
\vspace{.7in}
\pause
{\tiny
 Solution:
 \begin{enumerate}[(a)]
 \item  Because each ranking corresponds to a particular
ordered arrangement of the 10 people, the answer to this part is
$10! = 3,628,800$.
\item  Since there are 6! possible rankings of the men among
themselves and 4! possible rankings of the women among
themselves, it follows from the basic principle that there are
$(6!)(4!) = (720)(24) = 17280$ possible rankings in this case.
\end{enumerate}  
}
\end{frame}







\section{Combinations: {\small Arrangements of {\bf distinct} objects {\bf without replacement} when {\bf order is not important}}   }



\TransitionFrame[bittersweet]{\Large Combinations: {\small Arrangements of {\bf distinct} objects {\bf without replacement} when {\bf order is not important}}  }




\begin{frame}{Combinations (Un-ordered, without replacement)}
\begin{center}
\vspace{-.1in}
\define{Combinations}{ Let $n\geq r $ be two non-negative integers.  The number of different ways to select (/choose) $\HLTY{r}$ distinct objects from a list of $\HLTY{n}$ distinct (non-identical) objects  is given as( {\tiny or $^nC_r$)},
 \vspace{-.1in} $$\HLTEQ[babyblue!70]{\HLTW{ \displaystyle {n \choose r }}= \HLTW{ \displaystyle\frac{n!}{r! (n-r)!}}} $$
}
\vspace{-.2in}
\qBrd[4.1in]{amethyst!40}{
Also, we may envision the number ${n \choose r }$ to be  the different ways $r$ distinct objects can be chosen from $n$ distinct objects  when the order of the selection is not important, but, there repeated selection of an object is not allowed. }
\end{center}
\vspace{1in}


\end{frame}


\begin{frame}
\vspace{-.2in}
\qbx[4.6in]{amethyst!40}{\Exmpl{amethyst}{}
\small A motherboard has 8 slots, where we want to insert 4
{\bf identical} memory-cards.  How many unique possible ways one can insert the 4 memory-cards?}\\
\includegraphics[scale=.25]{figs/motherboard.png}
\includegraphics[scale=.2]{figs/slotCard5.png}\\
\pause
\vspace{1in}
 \HLT[yellow]{ \tiny Solution ${8 \choose 4 }= \frac{8!}{(4!)(8-4)! }$}  
\end{frame}




\begin{frame}
\vspace{-.2in}
\qBrd[4.65in]{iceberg!30}{ \Exmpl{iceberg!70}{} In a game of Lottery,  a person may pick any six numbers from  \{1, 2, \ldots ,44\}.  How many different groups of six numbers can be chosen {\bf if the repeated selection of a numbers is {\it NOT allowed}  and the order at which the numbers are selected is  NOT considered to be important.}
}\\
\includegraphics[scale=.25]{figs/lottery.png} \\
\pause 
\vspace{1in}
{\tiny
Solution: \HLT[yellow]{${ {44}\choose 6}= \frac{44\time 43 \times 42\times 41\times 40 \times 39}{6\times 5\times 4\times 3\times 2\times 1} $}}
\end{frame}
%
%
%\begin{frame}
%\qbx[4.5in]{applegreen!40}{\Exmpl{applegreen}{}
%From a group of 5 women and 7 men, how many
%different committees consisting of 2 women and 3 men can be
%formed? What if 2 of the men are feuding and refuse to serve on
%the committee together? }\\
%\vspace{1.5in}
%\pause
% {\tiny Solution: ${5 \choose 2}\times {7 \choose 3}= 350$\\
% Now a total of ${2 \choose 2}\times {5 \choose 1}= 5 $ out of ${7 \choose 3}=35$ possible groups of 3
%men contain both of the feuding men, it follows that there are 35 -5 = 30 groups that do not contain
%both of the feuding men. Then, the possible number of committees becomes $30 \times 10 =300$
% }  
%\end{frame}

\begin{frame}
\define{Multinomial Probability}{
The number of ways of partitioning $n$ distinct objects into $k$ distinct groups
containing $n_1$, $n_2$, . . . , $n_k$ objects, respectively, where each object appears exactly in  one group and $\HLTW{\sum_{i=1}^{k}n_i=n}$, is  
 $$\HLTEQ[yellow]{\large \displaystyle { n\choose {n_1, n_2, \ldots , n_k}}:= \frac{n!}{(n_1!)(n_2!)\ldots (n_k!)}}$$
}
\begin{center}
\qBrd[4.5in]{amber!40}{The above procedure is also utilized to determine the number of permutations of a set
of $n$ objects when one or multiple group of objects are indistinguishable from each other.}
\end{center}
\vspace{.5in}
\end{frame}


\begin{frame}
\qbx[4.5in]{teal!40}{\Exmpl{teal}{}
How many different `words' ({\tiny sequences of letters which may or may not be a dictionary word }) can be constructed by permuting the letters of the word  \HLTW{\text{"PEPPER"}}?}\\
\vspace{2.5in}
\pause
{\tiny
 Solution:  $\frac{6!}{(3!)(2!)(1!)}$ }
\end{frame}




%\section{Arrangements {\bf With Replacement} When {\bf Order is Not Important} }
%
%\TransitionFrame[bittersweet]{ Arrangements {\bf With Replacement} When {\bf Order is Not Important}  }


\section{Arrangements of {\bf identical} Objects in Multiple Groups}
\vspace{-.2in}
\TransitionFrame[bittersweet]{Arrangements of {\bf identical} objects in Multiple Groups ({\tiny without replacement and order of objects in each selected group is not important. }) }

\begin{frame}{Unordered Arrangements of Identical Objects}
\begin{center}
\define{Unordered Arrangements of Identical Objects}{
Number of ways $n$ indistinguishable/identical objects can be organized into r different (ordered) groups is $$\HLTW{\displaystyle \frac{(n+r-1)!}{n! (r-1)!}={{n+r-1} \choose {n}}}.$$
}
\vspace{-.2in}
\qBrd[4.1in]{guppiegreen!30}{ The processes can viewed as:  How many different ways $\HLTW{n}$ identical  balls can be placed in $\HLTW{r}$ different urns.  Ordering of the urns are considered to be important.  }\\
%
%\qBox{\Qn: How many different solutions  $(X_1, X_2, \ldots, X_r)$ are there satisfying the equation $X_1+X_2+\ldots + X_r=n$ where we assume $X_i $ to be non-negative integer for all $i=1, \ldots, r$.}
\vspace{.1in}
%\qBrd[4.5in]{amethyst!50}{\sqBullet{amethyst} It is also referred to as the occupancy problem }

\vspace{.1in}
\qBrd[4.5in]{amber!40}{ \sqBullet{amber} \small In the textbook it is referred to as: ``THE NUMBER OF INTEGER SOLUTIONS OF EQUATIONS''}
\end{center}
\vspace{.5in}

\end{frame}






\begin{frame}
\vspace{-0.2in}
\qbx[4.6in]{apricot!40}{\Exmpl{apricot}{}
How many different solutions are there for the following equation?
$\HLTW{x_1+x_2+x_3+x_4+x_5=10}$,  if $x_1, x_2, x_3, x_4,x_5$ are all non-negative integers?
}\\
\vspace{2in}
\pause
{\tiny Solution:  ${14 \choose 4}   =  1001$  }
\end{frame}



%
%\begin{frame}
%\qBrd[4.5in]{iceberg!30}{ \Exmpl{iceberg!70}{} From the numbers \{1, 2, \ldots ,44\}, a person
%may pick any six for her ticket.  How many different groups of six numbers can be chosen from the forty-four {\bf if the repeated selection of the numbers are allowed while the order at which the numbers are selected is not important.}
%}\\
%\includegraphics[scale=.25]{figs/lottery.png} \\
%\pause 
%\vspace{.8in}
%{\tiny
%Solution: \HLT[yellow]{$\frac{(6+44-1)!}{(43!) (6!)}={{49} \choose {43}}$}}
%\end{frame}



\section{Miscellaneous Problems On Counting}

\TransitionFrame[amethyst]{ Miscellaneous Problems On Counting }


\begin{frame}
\qbx[4.6in]{brightube!40}{\Exmpl{brightube}{}
Consider 6 tosses of a coin.  For example two typical sequences of outcomes that are considered different is  `HTTTTT', and `THTTTT'.
\begin{enumerate}[a).]
\item What is the total number of different outcomes from the experiment?
\item  How many different ways there can be exactly 2 heads?
\end{enumerate} 
}\\
\includegraphics[scale=.2]{figs/coin.png} \\
\vspace{.7in}
\pause
{\tiny
 Solution: a)  ${2^6}= 64$,  b) ${6 \choose 2}$  }
\end{frame}




\begin{frame}
\qBrd[4.5in]{purple!30}{ \Exmpl{purple!50}{} If a dice is thrown 4 times and all the the 4 numbers (4-tuple) that appear are recorded.  
\begin{enumerate}[a).]
\qBrd[4in]{camel!40}{ \item What is the total number of  distinct possibilities of the 4 numbers (distinct 4-tuples) that can occur?}
\qBrd[4in]{amber!40}{ \item In how many such 4-tuples, there will be exactly two 5s?}
\qBrd[4in]{olive!40}{ \item In how many such 4-tuples, there will be no 6?}
\qBrd[4in]{babyblue!40}{ \item In how many such 4-tuples, there will be at least one  6?}
\end{enumerate}
\vspace{.05in}
} 

\includegraphics[scale=.15]{figs/dice.png} \\
\vspace{.3in}
\pause
{\tiny Solution: 
a)$ \HLTY{6^4}$, b) $\HLTY{{4\choose 2} 5^2}$, c) $\HLTY{5^4}$ d) $\HLTY{6^4-5^4}$
}

\end{frame}




\begin{frame}
\qbx[4.5in]{amber(sae/ece)!40}{\Exmpl{amber(sae/ece)}{} {\tiny Passwords are a context where it is interesting to count and compare the total number of possible ways when the length is relatively small and as restrictions are introduced or relaxed. } \\
A typical ATM pin consists of 4 digits.  
\begin{enumerate}
\item How many different (unique) 4 digit passwords can be constructed if digits are restricted to be only numeric? 
\item How many different (unique) 4 digit passwords can be constructed if the First digit must be an English alphabet while rest are numeric? 
\item How many different (unique) 4 digit passwords can be constructed if the digits can either be any number or English alphabet? 
\end{enumerate}  }\\
\includegraphics[scale=.25]{figs/ATMPIN.png} \\
\pause 
\vspace{1.7in}
%\pause
%{\tiny
 %Solution:  $\frac{10!}{(4!)(3!)(2!)(1!)}$ }
\end{frame}








\begin{frame}
\qbx[4.5in]{rose!40}{\Exmpl{rose!70}{} \\
What is the number of  unique permutations for the letters in the word `BANANA'? }\\
\vspace{1.5in}
%\pause
%{\tiny
 %Solution:  $\frac{10!}{(4!)(3!)(2!)(1!)}$ }
 \vspace{2in}
\end{frame}




\begin{frame}
\qbx[4.5in]{olive!40}{\Exmpl{olive}{}
Consider a Set $A=\{a_1, a_2, a_3 , a_4\}$ containnig $4$ elements.  
\begin{enumerate}[a).]
\item What is the total number of different subsets of the set $A$ that has exactly $2$ elements in it?
\item What is the total number of different subsets of the set $A$?
\item What is the total number of different non empty proper subsets of the set $A$?
\end{enumerate} 
}\\
\vspace{1.5in}
\pause
{\tiny
 Solution: a) ${4 \choose 2}$ b)  ${2^4}= 16$,  c) $2^4-2=14$ }
 \vspace{.11in}
\end{frame}


%
%\begin{frame}
%\qbx[4.5in]{applegreen!40}{\Exmpl{applegreen}{}
%Ms. Jones has 10 books that she is going to put on
%her bookshelf. Of these, 4 are mathematics books, 3 are chemistry
%books, 2 are history books, and 1 is a language book. Ms. Jones
%wants to arrange her books so that all the books dealing with the
%same subject are together on the shelf. How many different
%arrangements are possible?}\\
%\vspace{1.5in}
%\pause
%{\tiny
% Solution: There are $4! 3! 2! 1!$ arrangements such that the
%mathematics books are first in line, then the chemistry books, then
%the history books, and then the language book. Similarly, for each
%possible ordering of the subjects, there are $4! 3! 2! 1!$ possible
%arrangements. Hence, as there are $4!$ possible orderings of the
%subjects, the desired answer is $4! (4! 3! 2! 1!) = 6912$. }
%\end{frame}


\begin{frame}
\qbx[4.5in]{teal!40}{\Exmpl{teal}{}
One of the sections in the class STAT230 has 8 students.  Assume the birthday of a student can be any one of days out of 365 days in a year.  We say that the two students share a same birth day if they are born on the same day and same month {\tiny ( for example two students are born on $10^{\text{th}}$ January).}  Answer the follwoing questions. 
\begin{enumerate}[a).]
\item What is the total number of different possibilities for the birthday of the 8 students?
\item How many different possibilities are there such that  exactly two students have the same birthday?
\item How many different possibilities are there if exactly four of students have the same birthday?
\item How many different possibilities are there so that all the birthdays are on different day of the year. 
\end{enumerate} 
}\\
%\vspace{.1in}
\pause
{\tiny
 Solution: a)$ 365^8$ b)  $365\times  {8 \choose 2} 364\times \cdots \times 359$ c)  $365\times {8 \choose 4} \times 364\times 363\times 362 \times 361$  c) $^{365}P_{8}$ }
 \vspace{.4in}
\end{frame}




%
%
%\begin{frame}
%\vspace{-.5in}
%\qbx[4.5in]{amethyst!40}{\Exmpl{amethyst}{}
%A chess tournament has 10 competitors, of which 4
%are Russian, 3 are from the United States, 2 are from Great Britain,
%and 1 is from Brazil. If the tournament result lists just the
%nationalities of the players in the order in which they placed, how
%many outcomes are possible?}\\
%\vspace{1.1in}
%\pause
%{\tiny
% Solution:  $\frac{10!}{(4!)(3!)(2!)(1!)}$ }
%\end{frame}
%



%
%
\begin{frame}
	\frametitle{Example }
\qbx[4.5in]{cherryblossompink!40}{
\Exmpl{cherryblossompink}{} How many different 7-place license plates are possible if the first 3 places are to be occupied by letters and the final 4 by numbers?
}\\
\includegraphics[scale=.1]{figs/NewYork_Plate.png} \\

\vspace{.9in}
\pause
{\tiny Solution: By the generalized version of the basic principle, the answer is $26\times 26\times 26 \times 10\times 10\times 10\times 10 = 175, 760,000.$}

\vspace{1.5in}
\end{frame}






\begin{frame}
%\vspace{-.5in}
\qbx[4.6in]{teal!40}{\Exmpl{teal}{}
Suppose there are a total of 9 flags including 4  identical white  flags, 3  identical red flags, and 2  identical blue flags.
How many different ways the 9 flags can hoisted if we assume all the flagpoles to be arranged in a linear manner. }\\
\begin{center}
\includegraphics[scale=.23]{figs/9Flags.png}
\end{center}
\vspace{1.2in}
\pause
{\tiny
 Solution:  $\frac{9!}{(4!)(3!)(2!)}$ }
\end{frame}



\begin{frame}
%\vspace{-.5in}
\qbx[4.6in]{babyblue!40}{\Exmpl{babyblue}{}
How many different ways the 9 flags containing 4 white  flags, 3 red  flags, and 2 blue  flags can be placed on a 9 linearly arranged flag-poles? Assume that the same color flags can also be identified uniquely as there are different numbers written on them.}\\
\begin{center}
\includegraphics[scale=.23]{figs/9FlagNumber.png}
\end{center}
\vspace{1.2in}
\pause
{\tiny
 Solution:  $9!$ }
\end{frame}




\begin{frame}
\qbx[4.5in]{antiquefuchsia!40}{\Exmpl{antiquefuchsia}{}
Consider a group of 20 people. If everyone shakes
hands with everyone else, how many handshakes
take place?}\\
\includegraphics[scale=.2]{figs/HandShake.png} 

\vspace{1.5in}
\pause
{\tiny
 Solution:  ${20 \choose 2}$ }
\end{frame}





\begin{frame}
\qbx[4.5in]{apricot!40}{\Exmpl{apricot}{}
An investor has 20 thousand dollars to invest among 4 possible investments.  Each investment must be in units of a thousand dollars. If the total 20 thousand is to be invested, 
\begin{enumerate}
\item how many different investment strategies are possible?
\item  how many different investment strategies are possible  if not all the money need be invested?
\end{enumerate}    
}\\
\vspace{1.7in}
\pause
{\tiny
 Solution: a) ${23 \choose 3}= 1771$ , b)  ${24 \choose 4}= 10,626$ }
\end{frame}



\begin{frame}
\vspace{-.35in}
\begin{minipage}{.6\textwidth} %
\vspace{-.1in}
\qBrd[3.1in]{airforceblue!60}{\;
\includegraphics[scale=.33]{figs/UAEU_FLAG_GCC.png}
}
\end{minipage}
\;\;\;\;
\begin{minipage}{.32\textwidth} %
\qbx[1.45in]{amethyst!40}{\Exmpl{amethyst}{}
\small Consider that there are a total of 5 different flags.  How many different ways one can arrange  4 distinct flags in  4 different locations ? Assume that each of the locations are considered to have exactly one flag to put.  
}
\vspace{1.95in}
\end{minipage} %



{\tiny
 Solution:  }
\end{frame}





\begin{frame}
\begin{minipage}{.6\textwidth} %
\vspace{-.3in}
\qBrd[3.1in]{airforceblue!60}{\;
\includegraphics[scale=.3]{figs/UAEU_FLAG_UAE.png}
}
\end{minipage}
\;\;\;\;
\begin{minipage}{.32\textwidth} %
\qbx[1.45in]{lime!40}{\Exmpl{lime}{}
\small How many different ways one can arrange  10 identical flags in  4 different locations? Assume that,  a location may have zero, one or  more than one flags.
}
\vspace{1.95in}
\end{minipage} %



{\tiny
 Solution:  }
\end{frame}





\begin{frame}
\qbx[4.5in]{babyblue!40}{\Exmpl{babyblue}{}
A police department in a small city consists of 10
officers. If the department policy is to have 5 of the officers
patrolling the streets, 2 of the officers working full time at the
station, and 3 of the officers on reserve at the station, how many
different divisions of the 10 officers into the 3 groups are possible? }\\
\vspace{1.8in}
\pause
{\tiny
 Solution:  $\frac{10!}{(5!)(2!)(3!)}= 2520$ }
\end{frame}




\begin{frame}
\qbx[4.5in]{green!40}{\Exmpl{green}{}
Ten children are to be divided into an A team and a
B team of 5 each.  How many different ways we can construct team A and team B?}\\
\vspace{2in}
\pause
{\tiny
 Solution:  ${10 \choose 5}=\frac{10!}{(5!)(5!)}= 252$ }
\end{frame}


%
%
%\TransitionFrame[bittersweet]{\Large Binomial and Multinomial Coefficient  }



\begin{frame}\frametitle{Example}
How many different ways one can arrange ( permute) the following 5 distinct/ nonidentical balls in a linear manner?\\
\includegraphics[scale=.35]{figs/5DifferentBalls.png}
\vspace{.8in}


$\Col{{\Row{\MCQchk[2.1in]{
Order is Important 
} , \MCQ[2.1in]{
Order is NOT Important
}}},{\Row{\MCQ[2.1in][babyblue!60]{ 
WITH Replacement
},\MCQchk[2.1in][babyblue!60]{
WITHOUT Replacement
}}}}$

\vspace{.1in}
{\tiny Solution: $^5P_5=5!$}
\end{frame}


\begin{frame}\frametitle{}
\vspace{-.1in}
How many different ways one can distribute the following 6 distinct balls so that each of the kids gets two balls each?\\
\includegraphics[scale=.25]{figs/6Balls2each.png}
%\vspace{.1in}

$\Col{{\Row{\MCQ[2in]{\tiny
Order of the balls a person receives is Important
} , \MCQchk[2.5in]{
\tiny 
Order of the two balls a person receives is  NOT Important
}}},{\Row{{\MCQchk[2in]{\tiny 
Order of the individuals is  Important
} }, \MCQ[2.5in]{\tiny Order of the Individuals are NOT Important
}}},{\Row{\MCQ[2in][babyblue!60]{ 
WITH Replacement
},\MCQchk[2.5in][babyblue!60]{
WITHOUT Replacement
}}}}$
\pause
{\tiny Solution: ${6 \choose 2}{4 \choose 2}{2\choose 2}= \frac{6!}{(2!)(2!)(2!)}$}
\end{frame}


\begin{frame}\frametitle{}
How many different ways one can distribute the following 6 distinct balls ({\tiny consider the scenario when a kid may obtain NONE,One or More balls})  ?\\
\includegraphics[scale=.32]{figs/6PurpleBass3kids.png}


$\Col{{\Row{\MCQ[2in]{\tiny
Order of the balls a person receives is Important
} , \MCQchk[2.5in]{
\tiny 
Order of the two balls a person receives is  NOT Important
}}},{\Row{{\MCQchk[2in]{\tiny 
Order of the individuals is  Important
} }, \MCQ[2.5in]{\tiny Order of the Individuals are NOT Important
}}},{\Row{\MCQ[2in][babyblue!60]{ 
WITH Replacement
},\MCQchk[2.5in][babyblue!60]{
WITHOUT Replacement
}}}}$
\pause
{\tiny Solution: The problem is same as the number of solutions of $x_1+x_2+x_3=6$ where $x_1, x_2, x_3$ are non-negative integers.  Therefore, the solution is   ${{6+3-1} \choose {3-1} }= {8 \choose 2}= \frac{8\times 7}{2\times 1}= 28.$}
\end{frame}



\TransitionFrame[antiquefuchsia]{\Large Questions?  }


\TransitionFrame[antiquefuchsia]{\Large Thank You }
 
 
\end{document}
