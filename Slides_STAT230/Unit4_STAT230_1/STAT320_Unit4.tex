\documentclass[compress]{beamer}
\mode<presentation>
\setbeamercovered{transparent}
\usetheme{Warsaw}
%\useoutertheme{smoothtree}
\usepackage{multirow}
\usepackage[english]{babel}
\usepackage[latin1]{inputenc}
\usepackage{times}
\usepackage[T1]{fontenc}
\usepackage{xmpmulti}
\usepackage{multicol}
\usepackage{colortbl}

%\setbeamersize{text margin left=.25 in,text margin right=.25 in}
\setbeamersize{text margin left=.15 in,text margin right=.15 in}
\usepackage[authoryear]{natbib}


\usepackage{epstopdf}
\usepackage{xcolor}
\usepackage{latexcolors}
%\usepackage[dvipsnames]{xcolor}
\definecolor{antiquebrass}{rgb}{0.8, 0.58, 0.46}
\definecolor{babyblueeyes}{rgb}{0.63, 0.79, 0.95}
\definecolor{babyblue}{rgb}{0.54, 0.81, 0.94}
\definecolor{bistre}{rgb}{0.24, 0.17, 0.12}
\definecolor{brightlavender}{rgb}{0.75, 0.58, 0.89}
\definecolor{bulgarianrose}{rgb}{0.28, 0.02, 0.03}
\definecolor{slateblue}{rgb}{0.56, 0.74, 0.56}
\definecolor{cordovan}{rgb}{0.54, 0.25, 0.27}
\definecolor{darkbyzantium}{rgb}{0.36, 0.22, 0.33}
%byzantium
\setbeamercolor{structure}{fg=calpolypomonagreen!70, bg= black!60}







\usepackage{tikz}
\usetikzlibrary{shadows,calc}
\usetikzlibrary{shadows.blur}
\usetikzlibrary{shapes.symbols}
\usepackage{hyperref}
\usepackage{booktabs}
\usepackage{colortbl}
\usepackage{multirow}
%%%%%%%%% shaddow image %%%%%
% some parameters for customization
\def\shadowshift{3pt,-3pt}
\def\shadowradius{6pt}
\colorlet{innercolor}{black!60}
\colorlet{outercolor}{gray!05}
% this draws a shadow under a rectangle node
\newcommand\drawshadow[1]{
\begin{pgfonlayer}{shadow}
    \shade[outercolor,inner color=innercolor,outer color=outercolor] ($(#1.south west)+(\shadowshift)+(\shadowradius/2,\shadowradius/2)$) circle (\shadowradius);
    \shade[outercolor,inner color=innercolor,outer color=outercolor] ($(#1.north west)+(\shadowshift)+(\shadowradius/2,-\shadowradius/2)$) circle (\shadowradius);
    \shade[outercolor,inner color=innercolor,outer color=outercolor] ($(#1.south east)+(\shadowshift)+(-\shadowradius/2,\shadowradius/2)$) circle (\shadowradius);
    \shade[outercolor,inner color=innercolor,outer color=outercolor] ($(#1.north east)+(\shadowshift)+(-\shadowradius/2,-\shadowradius/2)$) circle (\shadowradius);
    \shade[top color=innercolor,bottom color=outercolor] ($(#1.south west)+(\shadowshift)+(\shadowradius/2,-\shadowradius/2)$) rectangle ($(#1.south east)+(\shadowshift)+(-\shadowradius/2,\shadowradius/2)$);
    \shade[left color=innercolor,right color=outercolor] ($(#1.south east)+(\shadowshift)+(-\shadowradius/2,\shadowradius/2)$) rectangle ($(#1.north east)+(\shadowshift)+(\shadowradius/2,-\shadowradius/2)$);
    \shade[bottom color=innercolor,top color=outercolor] ($(#1.north west)+(\shadowshift)+(\shadowradius/2,-\shadowradius/2)$) rectangle ($(#1.north east)+(\shadowshift)+(-\shadowradius/2,\shadowradius/2)$);
    \shade[outercolor,right color=innercolor,left color=outercolor] ($(#1.south west)+(\shadowshift)+(-\shadowradius/2,\shadowradius/2)$) rectangle ($(#1.north west)+(\shadowshift)+(\shadowradius/2,-\shadowradius/2)$);
    \shade[outercolor,right color=innercolor,left color=innercolor] ($(#1.north west)+(-\shadowradius/12,\shadowradius/12)$) rectangle ($(#1.south east)+(\shadowradius/12,-\shadowradius/12)$);%Frame
    \filldraw ($(#1.south west)+(\shadowshift)+(\shadowradius/2,\shadowradius/2)$) rectangle ($(#1.north east)+(\shadowshift)-(\shadowradius/2,\shadowradius/2)$);
\end{pgfonlayer}
}
% create a shadow layer, so that we don't need to worry about overdrawing other things
\pgfdeclarelayer{shadow} 
\pgfsetlayers{shadow,main}
% Define image shadow command
\newcommand\shadowimage[2][]{%
\begin{tikzpicture}
\node[anchor=south west,inner sep=0] (image) at (0,0) {\includegraphics[#1]{#2}};
\drawshadow{image}
\end{tikzpicture}}
\usepackage{calligra}

\DeclareMathOperator*{\argmax}{Arg\,max}
\DeclareMathOperator*{\argmin}{Arg\,min}
\newcommand{\norm}[1]{\left\Vert #1 \right\Vert }
\newcommand{\bbetaHat}{ \widehat{\bbeta}}
\newcommand{\bbetaLSE}{ \widehat{\bbeta}_{_{\text{LSE}}}}
\newcommand{\bbetaMLE}{ \widehat{\bbeta}_{_{\text{MLE}}}}
\newcommand{\sqBullet}[1]{  {\tiny \tiny \tiny \qBoxCol{#1!60}{ }} }
%***************
%\newtheorem{thm}{Theorem}
%\documentclass[noinfoline]{imsart}
%\usepackage{amsmath,amstext,amssymb}
%%\usepackage[top=1.5in, bottom=1.5in, left=1.2in, right=1.2in]{geometry}
%% settings
%%\pubyear{2005}
%%\volume{0}
%%\issue{0}
%%\firstpage{1}
%%\lastpage{8}
%\arxiv{arXiv:0000.0000}
\usepackage{subcaption}
%\startlocaldefs
%\numberwithin{equation}{section}
%\theoremstyle{plain}
%\newtheorem{thm}{Theorem}
%\endlocaldefs
\usepackage{lipsum} 
\usepackage{amsmath}
\usepackage{amssymb}
\usepackage{amsbsy} 
\usepackage{amsthm}
\usepackage{mathrsfs}
%\usepackage{eufrak}
\usepackage{mathrsfs}
\usepackage{color}
\usepackage{verbatim}
\usepackage{graphicx}
\usepackage{bm}
\usepackage{enumerate}
\usepackage{epstopdf} 
\usepackage{natbib}
\usepackage{undertilde}

\usepackage{tfrupee}

\usepackage{tikz}
\usetikzlibrary{shadows,calc}
\usetikzlibrary{shadows.blur}
\usetikzlibrary{shapes.symbols}
%%%%%%%%% shaddow image %%%%%
\usepackage{calligra}

%\newcommand{\logLik}{\text{\calligra l}\,}
%\usepackage{calligra,amsmath,amssymb}

\usepackage{mathrsfs}
\DeclareMathAlphabet{\mathpzc}{OT1}{pzc}{m}{it} 
% \newcommand{\logLik}{ \mathpzc{l}}
 \newcommand{\logLik}{ \mathbb{\ell}_{_n}}
  \newcommand{\Lik}{ \mathcal{L}_{_n}}
  \newcommand{\score}{\mathpzc{S}_{_n}}
  %\newcommand{\Finfo}{1}{ \mathpzc{I}_{#1}}
  \NewDocumentCommand{\Finfo}{O{ }}{ \mathcal{I}_{_{#1}}}
\newcommand{\Bias}[1]{  \text{Bias}\left(#1\right)   }
\newcommand{\Var}[1]{  \text{Var}\left(#1\right)  }
\newcommand{\Mse}[1]{  \text{Mse}\left(#1\right)}   

\newcommand{\gCalli}{\text{\calligra g}\,}
% some parameters for customization
\def\shadowshift{3pt,-3pt}
\def\shadowradius{6pt}

\colorlet{innercolor}{black!60}
\colorlet{outercolor}{gray!05}

% this draws a shadow under a rectangle node
\newcommand\drawshadow[1]{
    \begin{pgfonlayer}{shadow}
        \shade[outercolor,inner color=innercolor,outer color=outercolor] ($(#1.south west)+(\shadowshift)+(\shadowradius/2,\shadowradius/2)$) circle (\shadowradius);
        \shade[outercolor,inner color=innercolor,outer color=outercolor] ($(#1.north west)+(\shadowshift)+(\shadowradius/2,-\shadowradius/2)$) circle (\shadowradius);
        \shade[outercolor,inner color=innercolor,outer color=outercolor] ($(#1.south east)+(\shadowshift)+(-\shadowradius/2,\shadowradius/2)$) circle (\shadowradius);
        \shade[outercolor,inner color=innercolor,outer color=outercolor] ($(#1.north east)+(\shadowshift)+(-\shadowradius/2,-\shadowradius/2)$) circle (\shadowradius);
        \shade[top color=innercolor,bottom color=outercolor] ($(#1.south west)+(\shadowshift)+(\shadowradius/2,-\shadowradius/2)$) rectangle ($(#1.south east)+(\shadowshift)+(-\shadowradius/2,\shadowradius/2)$);
        \shade[left color=innercolor,right color=outercolor] ($(#1.south east)+(\shadowshift)+(-\shadowradius/2,\shadowradius/2)$) rectangle ($(#1.north east)+(\shadowshift)+(\shadowradius/2,-\shadowradius/2)$);
        \shade[bottom color=innercolor,top color=outercolor] ($(#1.north west)+(\shadowshift)+(\shadowradius/2,-\shadowradius/2)$) rectangle ($(#1.north east)+(\shadowshift)+(-\shadowradius/2,\shadowradius/2)$);
        \shade[outercolor,right color=innercolor,left color=outercolor] ($(#1.south west)+(\shadowshift)+(-\shadowradius/2,\shadowradius/2)$) rectangle ($(#1.north west)+(\shadowshift)+(\shadowradius/2,-\shadowradius/2)$);
        \filldraw ($(#1.south west)+(\shadowshift)+(\shadowradius/2,\shadowradius/2)$) rectangle ($(#1.north east)+(\shadowshift)-(\shadowradius/2,\shadowradius/2)$);
    \end{pgfonlayer}
}

% create a shadow layer, so that we don't need to worry about overdrawing other things
\pgfdeclarelayer{shadow} 
\pgfsetlayers{shadow,main}

\newsavebox\mybox
\newlength\mylen

\newcommand\shadowimage[2][]{%
\setbox0=\hbox{\includegraphics[#1]{#2}}
\setlength\mylen{\wd0}
\ifnum\mylen<\ht0
\setlength\mylen{\ht0}
\fi
\divide \mylen by 120
\def\shadowshift{\mylen,-\mylen}
\def\shadowradius{\the\dimexpr\mylen+\mylen+\mylen\relax}
\begin{tikzpicture}
\node[anchor=south west,inner sep=0] (image) at (0,0) {\includegraphics[#1]{#2}};
\drawshadow{image}
\end{tikzpicture}}

%\begin{document}
%
%\noindent\shadowimage[width=6cm]{image}\par\bigskip

%%%%%%%%%%%%%%%%%%%%%%%



%\RequirePackage[colorlinks,citecolor=blue,urlcolor=blue]{hyperref}
%\usepackage{subfig}
\usepackage[final]{pdfpages}

\usepackage{algorithm}  %@subhajit
\usepackage{algpseudocode} %@subhajit
\usepackage{algorithmicx}     %@subhajit
\usepackage{undertilde}


\newcommand{\sphere}{{\mathbb{S}}}
\newcommand{\R}{\mathbb{R}}
\newcommand{\LatentV}{V}
\newcommand{\NC}{m}
\newcommand{\Priorf}{f_{prior}}
\newcommand{\FWOne}[2]{{{}_{1}\Psi _{1}\left[{\begin{matrix}(\frac{#1}{2},\frac{1}{2})\\(1,0)\end{matrix}};#2\right]} 
}


\newcommand{\HyPriorMu}{\thetabf}
\newcommand{\HyPriorAlpha}{\alpha}
\newcommand{\HyPriorBeta}{\beta}
\newcommand{\HyPriorK}{\zeta}
\newcommand{\Indicator}[1]{\mathbb{I}({#1 })}
\newcommand{\IndicatorA}[2]{\mathbb{I}_{#2}({#1 })}
\newcommand{\xb}{\bm{x}}
\newcommand{\bx}{\bm{x}}



\newcommand{\bX}{\bm{X}}
\newcommand{\by}{\bm{y}}
\newcommand{\bZ}{\bm{Z}}
\newcommand{\bF}{\bm{F}}
\newcommand{\btheta}{\bm{\theta}}
\newcommand{\Bpi}{\boldsymbol{\pi}}
\newcommand{\thetabf}{\boldsymbol{\theta}}
\newcommand{\Thetabf}{\boldsymbol{\Theta}}
\newcommand{\taubf}{\boldsymbol{\tau}}
\newcommand{\Tr}{Tr}
\newcommand{\HaarMu}{\mu}
\newcommand{\RestMu}{\mu_{\delta}}
\newcommand{\ConstOne}{K}

\newcommand{\bM}{\bm{M}}
\newcommand{\bD}{\utilde{\bm{D}}}
\newcommand{\bV}{\bm{V}}
\newcommand{\loglikmix}{\mathcal{L}_{\bM,\bD,\bV}}
\newcommand{\hypdc}{{}_0F_1\left(\frac{n}{2},\frac{D_c^2}{4}\right)}


\usepackage{xstring}
\usepackage[normalem]{ulem}
\definecolor{ultramarine}{RGB}{38,29,163}
\newcommand\PalDel[1]{{\color{red} {\sout{#1}}}}
\newcommand\Pal[1]{{\color{ultramarine}{#1}}}
\newcommand\PalRp[2]{\PalDel{#1} \Pal{#2}}
\newcommand\PalCmnt[1]{{\color{ultramarine} {[[[***PAL:  #1 ***]]]}}}

\newcommand{\qedwhite}{\hfill \ensuremath{\Box}}
\newcommand{\SpaceD}{\mathcal{S}_p}
\newcommand{\SpaceM}{\widetilde{\mathcal{V}}_{n,p}}
\newcommand{\SpaceV}{\mathcal{V}_{p,p}}
\newcommand{\SpaceF}{\mathbb{R}^{n,p}}
\newcommand{\StiefelS}{\mathcal{V}_{n,p}}
\newcommand{\SpacePi}{\mathbb{S}_{\pi}}
\newcommand{\ML}{{\cal{ML}}}
\newcommand{\ProdSpace}{\boldsymbol{\Theta}}
\newcommand{\ThetaAndPi}{\Xi}
\newcommand{\ClassML}{\mathcal{C}_{\ML}}
\newcommand{\balpha}{\bm{\alpha}}
\newcommand{\bbeta}{\bm{\beta}}
\newcommand{\bEta}{\bm{\eta}}
\newcommand{\bd}{{\utilde{\bm{d}}}}
\newcommand{\BoEta}{{\utilde{\boldsymbol{\eta}}}}
%\newtheorem{theorem}{Theorem}[section]
%\newtheorem{theorem}{Theorem}
%\newtheorem{lemma}{Lemma}
%\newtheorem{result}{Result}
\newtheorem{defn}{Definition}

\newcommand{\define}[2]{ \begin{definition}[#1]  #2  \end{definition}  }

\newcommand{\pdv}[2]{\frac{\partial#1}{\partial#2}}
\newcommand{\pdvtwo}[2]{\frac{\partial^2#1}{{\partial#2}^2}}


\newcommand{\mubf}{\boldsymbol{\mu}}
\newcommand{\sumI}{ \sum_{i=1}^{n}}
\newcommand{\Ybar}{{\overline{Y}}}

\newcommand{\Expectation}[1]{\mathbb{E}{[#1]}}
\newcommand{\priorXzero}{\Psi}
\newcommand{\iMat}{\mathbf{I}_{p}}

% 
% \newtheorem{thm}{Theorem}[section]
% \newtheorem{cor}[thm]{Corollary}
% \newtheorem{lem}[thm]{Lemma}
%\newtheorem{proposition}{Proposition}

%\newtheorem{theorem}{Theorem}[chapter]%To link the theorem to each chapter uncomment the chapter option
%\newtheorem{lemma}{Lemma}%[theorem]% To link each lemma to a theorem uncomment the theorem option
%\newtheorem{corollary}{Corollary}%[theorem]% To link each corollary to a theorem uncomment the theorem option
% to link a corollary to a chapter change the theorem option to chapter
%\newtheorem{definition}{Definition}%[chapter] %the same is true for both definitions and assumptions
\newtheorem{assumption}{Assumption}%[chapter] %
%\newtheorem{proposition}{Proposition}[chapter]
%\newtheorem{fact}{Fact} %%% added by @subho
\newcommand{\StrongNBD}[2]{S_{#1}{#2}}
\newcommand{\bpi} {\boldsymbol{\pi}}
\newcommand{\bphi} {\boldsymbol{\phi}}
\newcommand{\bb}[1]{\boldsymbol{#1}}
% Definitions of handy macros can go here

\newcommand{\normtwo}[1]{{\left\lVert#1\right\rVert}_2}

\newcommand{\dataset}{{\cal D}}
\newcommand{\fracpartial}[2]{\frac{\partial #1}{\partial  #2}}
\newcommand{\Lesbegue}[1]{\mu_{\btheta_{#1},\bpi_{#1}}}
\newcommand{\fthetapi}[1]{f_{\btheta_{#1},\bpi_{#1}}}
% Heading arguments are {volume}{year}{pages}{submitted}{published}{author-full-names}
\newcommand{\doublehat}[1]{%
    \settoheight{\dhatheight}{\ensuremath{\widehat{#1}}}%
    \addtolength{\dhatheight}{-0.35ex}%
    \widehat{\vphantom{\rule{2pt}{\dhatheight}}%
    \smash{\hspace{-0.5mm}\widehat{#1}}}}

\newcommand{\hyp}{{}_0F_1\left(\frac{n}{2},\frac{D^2}{4}\right)}
\newcommand{\hypinline}{{}_0F_1\left({n}/{2},{D^2}/{4}\right)}

\newcommand{\partialhyp}[1]{\frac{\partial}{\partial\,{d_{#1}}}\,\left[\hyp\right]}

\newcommand{\fracProbZ}[1]{\frac{\langle Z_{ic} \rangle \, #1}{\sum_{i=1}^{N} \langle Z_{ic}\rangle  } }
\newcommand{\EmVar}[1]{\widetilde{#1}^{(c)}}

\newcommand{\IMDY}{{\it{CCPD}}}
\newcommand{\JMDY}{{\it{JCPD}}}

\newcommand{\DYlang}{\frac{\exp(\nu\,\bEta^T\bd)}{{\left[{}_0F_1\left(\frac{n}{2},\frac{D^2}{4}\right)\right]}^{\nu}}}

\newcommand{\logDYlang}{\nu\,\bEta^T\bd - \nu\,\log\left({}_0F_1\left(\frac{n}{2},\frac{D^2}{4}\right)\right)}

\newcommand{\lhyp}{\log\left({}_0F_1\left(\frac{n}{2},\frac{D^2}{4}\right)\right)}

%\jmlrheading{1}{2000}{1-48}{4/00}{10/00}{SS \& JH \& AB}

% Short headings should be running head and authors last names

%\ShortHeadings{BDP and cIBP}{SS \& JH \& AB}
%\firstpageno{1}

\newcommand{\diam}[1]{{{#1}^{\ast}}}

%%% coloring option %%%
\definecolor{auburn}{rgb}{0.53, 0.1, 0.5}
\newcommand{\sss}{\color{auburn}}  %%% for Subhajit
\newcommand{\sse}{\color{black}}
\newcommand{\attn}{\color{red}}
\newcommand{\rms}{\color{magenta}}  %%% for Riten
\newcommand{\rme}{\color{black}}
\newcommand{\MLDensity}{f_{\ML}}
\setlength{\parindent}{0cm}
\newcommand{\posterior}

\newcommand{\variableX}{\bd}
\newcommand{\funch}{\mathfrak{h}}
\newcommand{\IndVzero}[1]{\mathbb{I}({X\in \mathcal{V}^{#1}_0})}
\newcommand{\Rnp}{\mathbb{R}^{n \times p}}
\newcommand{\Rpp}{\mathbb{R}^{p \times p}}
\newcommand{\vecnorm}[1]{\lVert #1\rVert}

\newcommand{\etapsiD}{\eta_{\priorXzero}}
\newcommand{\BoEtapsiD}{\BoEta_{\priorXzero}}

\newcommand{\DMp}{\mathcal{D}^{p \times p}}
\newcommand{\Rplus}{\mathbb{R}_{+}}
\newcommand{\prodMeasure}{\Upsilon}

\newcommand{\m}{{\bf m_{\BoEta}}} 
\newcommand{\SetWithMode}{\mathcal{S}}
\newcommand{\SetWithModePrime}{\mathcal{S}}
\newcommand{\TargetComp}{\mathcal{S}^{\star}}

\newcommand{\ConstCondDen}{K_{\nu, \BoEta}} 

\newcommand{\hyparam}[2]{
    \IfEqCase{#1}{
        {M}{\xi^{#2}_c}
        {V}{\gamma^{#2}_c}%
        
    }
  }
\newcommand{\threepartdef}[6]
{
	\left\{
		\begin{array}{lll}
			#1 & \mbox{if } #2 \\
			#3 & \mbox{if } #4 \\
			#5 & \mbox{if } #6
		\end{array}
	\right.
}

\def\bv{\color{blue}}
\def\ev{\color{black}}
\newcommand{\bch}{\bv }
\newcommand{\ech}{\ev\normalsize}
%\newcommand{\MakeVec}[1]{{\utilde{\bf #1}}}
\newcommand \Measure[2][]{%
  \ifstrempty{#1}{
  \IfEqCase{#2}{
        {M}{\mu}%
        {D}{\mu_1}%
        {V}{\mu_2}
        {X}{\mu}
   }  
  }{
  \IfEqCase{#1}{
  {1}{
   \IfEqCase{#2}{
        {M}{d\mu(M)}%
        {D}{d\mu_1(\bd)}%
        {V}{d\mu_2(V)}
        {X}{d\mu(X)}
        {Y}{d\mu(Y)}
        {MDV} {d\mu(M)\; d\mu_1(\bd) \;d\mu_2(V) }
        }
   } 
   {2}{
   \IfEqCase{#2}{
         {M}{d\mu(M^{\ast})}%
        {D}{d\mu_1(\bd^{\ast})}%
        {V}{d\mu_2(V^{\ast})}
        {X}{d\mu(X^{\ast})}
        }
   }
   {3}{
   \IfEqCase{#2}{
         {M}{\mu(dM^{\star})}%
        {D}{\mu_2(d\bd^{\star})}%
        {V}{\mu_1(dV^{\star})}
        {X}{\mu(X^{\star})}
        }
   }   
   
   } 
  }%
}
  \newcommand{\VONF}{\text{VonMisesFisher}}
\newcommand{\MPGalpha}{\alpha}
\newcommand{\MPGnu}{\nu}
\newcommand{\MPG}{MPG }
\newcommand{\ybin}{y^{(\text{bin})}}


\usepackage{caption}
\usepackage{subcaption}


\newcommand{\nullSet}{\Phi}
\newcommand{\SP}{S}
\newcommand{\B}{ \mathcal{B}}
\newcommand{\prob}[1]{P\left( #1 \right)}
\newcommand{\Qn}{{\bf Question:}}
\newcommand{\Cmt}{{\bf Comment:}}




\newcommand{\support}{\mathcal{S}}
\newcommand{\tht}{\text{th}}
\newcommand{\abs}[1]{ \left\vert  #1 \right\vert }
\newcommand{\var}{\text{Var}}

\newcommand{\TwoColFunction}[2]{
\left\{
\begin{array}{ll}
#1 & \text{ if } #2\\
0 & \text{ otherwise. }
\end{array}
\right.
}
%%%%%%%%%%%%%%%%%%%%%%%%%%%
\newcommand{\vnsp}{\vspace{-.2in}}
\newcommand{\Cmnt}{{\bf Comment}}
\newcommand{\Eqn}[1]{ \vspace{-.15in} $$ {\HLTEQ{ \displaystyle  #1 }}\vspace{-.1in}$$   }


\newcommand{\sampleX}[1]{X_1, X_2, \ldots , X_{#1}}
\newcommand{\sampleY}[1]{Y_1, Y_2, \ldots , Y_{#1}}
\newcommand{\sampleZ}[1]{Z_1, Z_2, \ldots , Z_{#1}}
\newcommand{\sampleGen}[2]{{#2}_1, {#2}_2, \ldots , {#2}_{#1}}

\newcommand{\Xbar}{\overline{X}}
\newcommand{\Zbar}{\overline{Z}}
\newcommand{\Ubar}{\overline{U}}
\newcommand{\Vbar}{\overline{V}}
\newcommand{\Wbar}{\overline{W}}


\renewcommand{\bX}{\MakeVec{\bf X}}
\newcommand{\bY}{\MakeVec{\bf Y}}
\renewcommand{\bx}{\MakeVec{\bf x}}
\renewcommand{\by}{\MakeVec{\bf y}}


\newcommand{\pHat}{\widehat{p}}
\newcommand{\qHat}{\widehat{q}}
%\usepackage{xcolor}
\usepackage{xcolor}
\usepackage{xparse}
\definecolor{lightGray}{gray}{0.95}
\definecolor{lightGrayOne}{gray}{0.9}
\definecolor{lightBlueOne}{RGB}{204, 255, 255}
\definecolor{lightBlueTwo}{RGB}{204, 238, 255}
\definecolor{lightBlueThree}{RGB}{204, 204, 255}
\definecolor{AltBlue}{RGB}{119,14,161}


\definecolor{BGBlue}{RGB}{220,221,252}
\definecolor{BGBlueOne}{RGB}{204,229,255}



\definecolor{BGGreen}{RGB}{240,243,245}
\definecolor{lightGreenOne}{RGB}{179, 255, 179}
\definecolor{lightGreenTwo}{RGB}{198, 255, 179}
\definecolor{lightGreenThree}{RGB}{243, 255, 230}
\definecolor{AltGreen}{RGB}{193, 240, 240}

\definecolor{BOGreen}{RGB}{180,0,0}
\definecolor{BGGreenOne}{RGB}{220,250,220}

\definecolor{lightBrownOne}{RGB}{255, 221, 204}
\definecolor{lightBrownTwo}{RGB}{255, 229, 204}
\definecolor{lightBrownThree}{RGB}{242, 217, 230}


\definecolor{HLTGreen}{RGB}{230,244,215}
\definecolor{ExcBrown}{RGB}{153,0,0}
\definecolor{ExcBGBrown}{RGB}{255,204,204}
\definecolor{BGYellowOne}{RGB}{255,235,208}
\definecolor{BGPink}{RGB}{255,215,240}



\NewDocumentCommand{\HLT}{ O{HLTGreen} m }{\colorbox{#1}{#2}}
\NewDocumentCommand{\HLTEQ}{ O{HLTGreen} m }{\colorbox{#1}{$#2$}}

%\newcommand{\HLT}[1]{\colorbox{HLTGreen}{#1}}
\newcommand{\DEHLT}[1]{\colorbox{lightGrayOne}{\color{white} #1}}

\newcommand{\TextInBoxOne}[2]{  {\fcolorbox{lightGrayOne}{white}{\begin{minipage}{#1}  #2 \end{minipage}}}}

\newcommand{\TextInBoxOneQ}[2]{  {\fcolorbox{white}{lightGrayOne}{\begin{minipage}{#1}  #2 \end{minipage}}}}

\newcommand{\TextInBoxOneEQ}[2]{  {\fcolorbox{white}{lightBlueTwo}{\begin{minipage}{#1}  #2 \end{minipage}}}}

\newcommand{\QuizQuestion}[3]{  {\fcolorbox{black}{white}{\begin{minipage}{5.6 in}
\TextInBoxOneEQ{5.5in}{ #1 }\\
{\large \HLTEQ{\hspace{4.61in}\frac{\text{Score: \;\;\;\;}}{\text{#3}}}}\\
\vspace{.01in}#2 \end{minipage}}}}

\newcommand{\QuizQ}[3]{  {\fcolorbox{black}{lightGrayOne}{\begin{minipage}{5.6 in}
\TextInBoxOne{5.5in}{ #1 }\\
\vspace{.01in}#2 \end{minipage}}}}



\newcommand{\ExamQuestion}[3]{  {\fcolorbox{lightBlueTwo}{lightBlueTwo}{\begin{minipage}{5.85 in}
\TextInBoxOne{5.8in}{ #1 }\\
{\large \HLTEQ[lightBlueTwo]{\hspace{5.01in}\frac{\text{Score: \;\;\;\;}}{\text{#3}}}}\\
\end{minipage} }
#2 }}


\NewDocumentCommand{\MCOption}{O{1.75 in}m}{
\TextInBoxTwo[BGPink]{ #1 } {\TextInBoxTwo[white]{.1 in }{ \quad}\HLT{#2}}
}




\NewDocumentCommand{\MCOptionSelected}{m}{
\TextInBoxTwo[BGPink]{ 1.75 in } {\TextInBoxTwo[white]{.1 in }{{\huge $\bullet$}}\HLT{#1}}
}


%
%\NewDocumentCommand{\MCOption}{m}{
%\TextInBoxTwo[white]{.1 in }{ \quad}\HLT{#1}}







\NewDocumentCommand{\TextInBoxTwo}{ O{lightGrayOne} m m } {{\fcolorbox{white}{#1}{\begin{minipage}{#2} { #3} \end{minipage}}}}


\newcommand{\TextInBox}[2]{  {\fcolorbox{BGGreen}{BGGreen}{\begin{minipage}{#1}  #2 \end{minipage}}}}
\newcommand{\TextInBoxCol}[2]{
\fcolorbox{BGBlue}{BGBlue}{%
\begin{minipage}{#1}
 {\color{AltBlue} #2}
\end{minipage}}%
}




\newcommand{\BandInTopBox}[2]{
\fcolorbox{AltBlue}{AltBlue}{%
\begin{minipage}{#1}{ {\color{white}  #2 \hspace{.1in}} }
\end{minipage}}%
}


\newcommand{\TextInBoxThm}[2]{
\fcolorbox{AltBlue}{lightGray}{%
\begin{minipage}{#1}
 {\color{black} #2}
\end{minipage}}%
}

\newcommand{\TextInBoxThmOne}[2]{
\fcolorbox{BGBlue}{BGBlueOne}{%
\begin{minipage}{#1}
 {\color{AltBlue} #2}
\end{minipage}}%
}

\newcommand{\TextInBoxLem}[2]{
\fcolorbox{BGBlue}{lightGray}{%
\begin{minipage}{#1}
 {\color{black} #2}
\end{minipage}}%
}



\newcommand{\TextInBoxLemOne}[2]{
\vspace{.02 in}
\noindent
\fcolorbox{BGBlue}{BGBlue}{%
\begin{minipage}{#1}
 {\color{AltBlue} #2}
\end{minipage}}%
}


\newcommand{\CmntBox}[1]{
\noindent
\TextInBoxLem{5.3 in }{
\TextInBoxLemOne{5.2 in }{
#1
}}

}

\newcommand{\DefBox}[1]{
%\vspace{.1 in}
\noindent
\TextInBoxLem{6 in }{
\BandInTopBox{5.9 in }{}
\TextInBoxLemOne{5.9 in }{
#1
}}}


\newcommand{\DefBoxL}[1]{
%\vspace{.1 in}
\noindent
\TextInBoxLem{8 in }{
\BandInTopBox{7.9 in }{}
\TextInBoxLemOne{7.9 in }{
#1
}}}




%Old measurements
%\newcommand{\DefBoxOne}[2]{
%%\vspace{.1 in}
%\noindent
%\TextInBoxLem{6 in }{
%\BandInTopBox{5.9 in }{#1}
%\TextInBoxLemOne{5.9 in }{
%#2
%}}}
%

\newcommand{\DefBoxOne}[2]{
%\vspace{.1 in}
\noindent
\TextInBoxLem{6.8 in }{
\BandInTopBox{6.7 in }{#1}
\TextInBoxLemOne{6.7 in }{
#2
}}}


\newcommand{\ThmBox}[2]{
\noindent
\TextInBoxThm{6.8 in }{
\TextInBoxThmOne{6.7 in }{
#1}
#2}
}

\newcommand{\LemBox}[2]{
\noindent
\TextInBoxLem{6.8 in }{
\TextInBoxLemOne{6.7 in }{
#1}
#2}
}

\newcommand{\PropBox}[2]{
\vspace{.1 in}
\noindent
\TextInBoxLem{6.8 in }{
\TextInBoxLemOne{6.7 in }{
#1}
#2}
}




\newcommand{\TextInBoxExc}[2]{
\noindent
\fcolorbox{white}{BGGreenOne}{%
\begin{minipage}{#1}
 {\color{black} #2}
\end{minipage}}%
}


\newcommand{\TextInBoxExample}[2]{
\noindent
\fcolorbox{white}{BGPink}{%
\begin{minipage}{#1}
 {\color{black} #2}
\end{minipage}}%
}


\newcommand{\ExerciseBox}[1]{
\noindent
%\TextInBoxLem{6 in }{
\TextInBoxExc{6 in }{
#1}
%#2}
}


\newcommand{\ExampleBox}[1]{
\noindent
%\TextInBoxLem{6 in }{
\TextInBoxExample{6 in }{
#1}
%#2}
}


\newcommand{\IndicatorA}[2]{\mathbb{I}_{#2}({#1 })}


 


\newcommand \rbind[1]{%
    \saveexpandmode\expandarg
    \StrSubstitute{\noexpand#1}{,}{&}[\fooo]%
    %\StrSubstitute{\fooo}{,}{&}[\fooo]%
    \StrSubstitute{\fooo}{;}{\noexpand\\}[\fooo]%
    \StrSubstitute{\fooo}{:}{\noexpand\\}[\fooo]%
    \restoreexpandmode
   \left[ \begin{matrix}\fooo\end{matrix}\right]
    }
    
    
    
   \newcommand \ColVec[1]{%
    \saveexpandmode\expandarg
    \StrSubstitute{\noexpand#1}{,}{\noexpand\\}[\fooo]%
    %\StrSubstitute{\fooo}{,}{&}[\fooo]%
    \StrSubstitute{\fooo}{;}{\noexpand\\}[\fooo]%
    \StrSubstitute{\fooo}{:}{\noexpand\\}[\fooo]%
    \restoreexpandmode
   \left[ \begin{matrix}\fooo\end{matrix}\right]
    }
     \newcommand \RowVec[1]{%
    \saveexpandmode\expandarg
    \StrSubstitute{\noexpand#1}{,}{&}[\fooo]%
    %\StrSubstitute{\fooo}{,}{&}[\fooo]%
    \StrSubstitute{\fooo}{;}{&}[\fooo]%
    \StrSubstitute{\fooo}{:}{&}[\fooo]%
    \restoreexpandmode
   \left[ \begin{matrix}\fooo\end{matrix}\right]
    }



  \newcommand \Row[1]{%
    \saveexpandmode\expandarg
    \StrSubstitute{\noexpand#1}{,}{&}[\fooo]%
    %\StrSubstitute{\fooo}{,}{&}[\fooo]%
    \StrSubstitute{\fooo}{;}{&}[\fooo]%
    \StrSubstitute{\fooo}{:}{&}[\fooo]%
    \restoreexpandmode
    \begin{matrix}\fooo\end{matrix}
    }
        
    
    
    
    \newcommand \Col[1]{%
    \saveexpandmode\expandarg
    \StrSubstitute{\noexpand#1}{,}{\noexpand\\}[\fooo]%
    %\StrSubstitute{\fooo}{,}{&}[\fooo]%
    \StrSubstitute{\fooo}{;}{\noexpand\\}[\fooo]%
    \StrSubstitute{\fooo}{:}{\noexpand\\}[\fooo]%
    \restoreexpandmode
    \begin{matrix}\fooo\end{matrix}
    }

%%%%%%%%%%%%%%%%%%%%% Experimental %%%%%%%%%%%%%%%%%


\ExplSyntaxOn
\DeclareExpandableDocumentCommand{\replicate}{O{}mm}
 {
  \int_compare:nT { #2 > 0 }
   {
    {#3}\prg_replicate:nn {#2 - 1} { #1#3 }
   }
 }
\ExplSyntaxOff


\ExplSyntaxOn
\DeclareExpandableDocumentCommand{\repdiag}{O{}mm}
 {
  \int_compare:nT { #2 > 0 }
   {
    {\prg_replicate:nn {#2}{#3#1}}{#3}
   }
 }
\ExplSyntaxOff


\newcommand \StrRowDiag[1]{%
    \saveexpandmode\expandarg
    \StrSubstitute{\noexpand#1}{,}{&}[\fooo]%
    %\StrSubstitute{\fooo}{,}{&}[\fooo]%
    \StrSubstitute{\fooo}{;}{&}[\fooo]%
    \StrSubstitute{\fooo}{:}{&}[\fooo]%
    \StrCount{\fooo}{&}[\countfooo]
    \restoreexpandmode
    \repdiag[0]{\countfooo+1}{{,}}
   %\left[ \begin{matrix}\fooo\end{matrix}\right]
    }


\newcommand \DiagStrOne[2]{%
    \saveexpandmode\expandarg
    \StrSubstitute{\noexpand#1}{,}{\noexpand#2}[\fooo]%
    \restoreexpandmode
   %\left[ \begin{matrix}\fooo\end{matrix}\right]
   \fooo
    }
    
    \newcommand \DiagStr[1]{%
    \DiagStrOne{#1}{{\StrRowDiag{#1}}}
    }


%$\rbind{\replicate[,]{10}{\Col{\replicate[;]{7}{0}}}}$

%$\Col{1,2,3}$
%$\ColVec{\replicate[;]{5}{B}}$
%$ \StrRowDiag{1,2} $

%$\DiagStr{1,2,3}$

%\repdiag[-]{3}{A}
\ExplSyntaxOn
\NewDocumentCommand{\Split}{ m m o }
 {
  \tarass_split:nn { #1 } { #2 }
  \IfNoValueTF { #3 } { \tl_use:N } { \tl_set_eq:NN #3 } \l_tarass_string_tl
 }

\tl_new:N \l_tarass_string_tl

\cs_new_protected:Npn \tarass_split:nn #1 #2
 {
  \tl_set:Nn \l_tarass_string_tl { #2 }
  % we need to start from the end, so we reverse the string
  \tl_reverse:N \l_tarass_string_tl
  % add a comma after any group of #1 tokens
  \regex_replace_all:nnN { (.{#1}) } { \1\, } \l_tarass_string_tl
  % if the length of the string is a multiple of #1 a trailing comma is added
  % so we remove it
  \regex_replace_once:nnN { \,\Z } { } \l_tarass_string_tl
  % reverse back
  \tl_reverse:N \l_tarass_string_tl
 }
\ExplSyntaxOff

%%%%%%%%%%%%%%%%%%%%%%%%%%%%%%%%

\newcommand{\ShowRowMatrix}[3]{ \left[ {\begin{array}{ccc}
  \line(1,0){22} &{#1} &  \line(1,0){22} \\
     & \vdots& \\
  \line(1,0){22}  &{#2}& \line(1,0){22} \\
   &  \vdots & \\
    \line(1,0){22} &{#3}& \line(1,0){22}  \\
    \end{array}
   } \right]}
 


\newcommand{\ShowColMatrix}[3]{ \left[ {\begin{array}{ccccc}
  \line(0,1){25} & &\line(0,1){25} &  &  \line(0,1){25} \\
  {#1}  & \ldots & {#2} &\ldots   &{#3} \\
 \line(0,1){25} &  & \line(0,1){25}  &  &  \line(0,1){25} \\
    \end{array}
   } \right]}
   
   
   
   
\newcommand{\ShowRowVector}[1]{ \left[ {\begin{array}{ccc}
  \line(1,0){25} &{#1} &  \line(1,0){25} 
    \end{array}
   } \right]}   
   
   
\newcommand{\ShowColVector}[1]{ \left[ {\begin{array}{c}
  \line(0,1){25} \\    {#1} \\   \line(0,1){25}     \end{array}  } \right]}
  
\newcommand{\ColVector}[3]{ \left[ {\begin{array}{c}
  {#1}\\ \vdots \\    {#2}\\ \vdots\\{#3}  \end{array}  } \right]}
  
  
  
  
  
\newcommand{\EqSetThree}[3]{ \left\{ {\begin{array}{c}
  {#1}\\ \vdots \\    {#2}\\ \vdots\\{#3}  \end{array}  } \right.}  
  



\newcommand{\MatrixTypeA}[3]{ \left[ {\begin{array}{ccc}
 {#1}_{1,1} & \cdots & {#1}_{1,{#3}}   \\
  {#1}_{2,1} & \cdots & {#1}_{2,{#3}}   \\
    \vdots  & \ddots& \vdots  \\
     {#1}_{{#2},1} & \cdots & {#1}_{{#2},{#3}}   \\
    \end{array}
   } \right]}
 
\newcommand{\MatrixTypeAKronecker}[4]{ \left[ {\begin{array}{ccc}
 {#1}_{11}{#4} & \cdots & {#1}_{1{#3}}{#4}   \\
  {#1}_{21} {#4} & \cdots & {#1}_{2{#3}} {#4}   \\
    \vdots  & \ddots& \vdots  \\
     {#1}_{{#2}1} {#4} & \cdots & {#1}_{{#2}{#3}} {#4}   \\
    \end{array}
   } \right]}
 



\newcommand{\ShowIMat}{ {\begin{array}{cccc}
 1&  &  &    \\
  & 1 &  &  \\
    &  & \ddots &    \\
     & & & 1   \\
    \end{array}
   } }
 
\newcommand{\ShowVecOne}{
\begin{array}{c}
 1\\ 1 \\    1  
\end{array}
}

 
\newcommand{\ShowUnitVecOne}{
\begin{array}{c}
 1\\ 0 \\   0  
\end{array}
}


\newcommand{\ShowUnitVecTwo}{
\begin{array}{c}
 0\\ 1 \\   0  
\end{array}
}


\newcommand{\ShowUnitVecThree}{
\begin{array}{c}
 0\\ 0\\   1  
\end{array}
}

\newcommand{\ShowZeroThree}{
\begin{array}{c}
 0\\ 0\\   0 
\end{array}
}


\newcommand{\TwoBlockMatrix}[2]{\left[  {\begin{array}{c;{2pt/2pt}c}
   {#1} &  {#2}
   \end{array} }\right]}
   
   \newcommand{\TwoBlockMatrixH}[2]{\left[  {\begin{array}{c}
   {#1} \\
   \hdashline[2pt/2pt]
    {#2}
   \end{array} }\right]}
   
   \newcommand{\TwoBlockH}[2]{ {\begin{array}{c}
   {#1} \\
   \hdashline[2pt/2pt]
    {#2}
   \end{array} }}
   
   
\newcommand{\TwoBlock}[2]{ {\begin{array}{c;{2pt/2pt}c}
   {#1} &  {#2}
   \end{array} }}
   

      
   
   
   
 \newcommand{\ThreeBlockColVec}[3]{
   \left[ {\begin{array}{c}
  #1\\
  \hdashline[2pt/2pt]\\
   \vdots\\
  \hdashline[2pt/2pt]\\
  #2\\
  \hdashline[2pt/2pt]\\
   \vdots\\
  \hdashline[2pt/2pt]\\
   #3\\
    \end{array}
   } \right]
   }



\NewDocumentCommand{\ColDyn}{>{\SplitList{;}}m}
   {
     \left[\begin{array}{c}
       \ProcessList{#1}{ \inserColtitem }
     \end{array}\right]
   }
   \newcommand \inserColtitem[1]{ #1 \\}


\makeatletter
\newcommand{\ColDynAlt}[2][r]{%
  \gdef\@VORNE{1}
  \left[\hskip-\arraycolsep%
    \begin{array}{#1}\vekSp@lten{#2}\end{array}%
  \hskip-\arraycolsep\right]}

\def\vekSp@lten#1{\xvekSp@lten#1;vekL@stLine;}
\def\vekL@stLine{vekL@stLine}
\def\xvekSp@lten#1;{\def\temp{#1}%
  \ifx\temp\vekL@stLine
  \else
    \ifnum\@VORNE=1\gdef\@VORNE{0}
    \else\@arraycr\fi%
    #1%
    \expandafter\xvekSp@lten
  \fi}
\makeatother


\NewDocumentCommand{\eVec}{m O{}}{\MakeVec{e}_{#1, (#2)}}

\NewDocumentCommand{\Ones}{O{3}}{\Col{\replicate[,]{#1}{1}}}
\NewDocumentCommand{\Zeros}{O{3}}{\Col{\replicate[,]{#1}{0}}}











\title{  STAT 320: Principles of Probability\\ {\color{black}  Unit 4: Conditional Probability \& Statistical Independence}}

\author[UAEU]
{United Arab Emirates University}
\institute[UAEU] % (optional, but mostly needed)
{
  \inst{Department of Statistics}%
  %Indian Institute of Management,  Udaipur\\
  \vspace{0.1in}

  
}

\date{}


\newcommand{\Xnew}{ \HLTEQ[orange]{X_{_{\text{i}}}} }
\newcommand{\Ynew}{ \HLTEQ[orange]{Y_{_{\text{i}}}} }

%\date{\today}

\AtBeginSection[]
{
  \begin{frame}{Inhalt}
 % \begin{multicols}{1}
	\frametitle{Outline}
    \tableofcontents[currentsection]
  %  \end{multicols}
  \end{frame}
}

\begin{document}
\maketitle

%\begin{frame}{Outline}
%%\begin{multicols}{}
%  \tableofcontents
%%\end{multicols}
%\end{frame}

%\section{Introduction to DSBA 2023}
%
%
%\begin{frame}
%\qBoxCol{blue!30}{
%\begin{center} Course  Website \end{center}
%\qbx[4.2in]{teal!40}{\sqBullet{teal} \color{blue} $ \href{https://sites.google.com/iimu.ac.in/dsba2023e/home}{https://sites.google.com/iimu.ac.in/dsba2023e/home}$
%}\\
%\qbx[3.0in]{green!40}{ \sqBullet{green} Regular Announcements.
%}\\
%\qbx[3.0in]{olive!40}{\sqBullet{olive}  Slides and other materials.
%}
%}
%
%\pause
%\qBoxCol{blue!30}{
%\sqBullet{blue}
%You can contact the instructor at {\it subhadip.pal@iimu.ac.in} and schedule for office hours.  
%}
%\pause
%\qBoxCol{olive!30}{
%\sqBullet{olive}
%Mr. Praveen Kumar has been assigned as Teaching Assistant (TA) for this course.  His email I'd is:  {\it praveen.kumar@iimu.ac. }
%}
%
%
%\end{frame}
%


%
%\begin{frame}{Course Outline}
%\hspace{-.1in}\qBoxCol{blue!35}{
%% Please add the following required packages to your document preamble:
%% \usepackage{booktabs}
%\begin{table}[]
%\begin{tabular}{@{}lll@{}}
%\toprule
%         & Topics                                                & Dataset or Case                                    \\ \midrule \midrule
%\rowcolor{blue!20}     \multicolumn{1}{|l|}{1-2}   & \multicolumn{1}{l|}{Overview of Data Science}        & \multicolumn{1}{l|}{Household Data}                \\ \midrule
%\rowcolor{purple!20} 
%\multicolumn{1}{|l|}{3-5}   & \multicolumn{1}{l|}{Data Visualization}              & \multicolumn{1}{l|}{Global Super Store }       \\ \midrule
%\rowcolor{blue!20} 
%\multicolumn{1}{|l|}{6}     & \multicolumn{1}{l|}{Introduction to R/ JMP}          & \multicolumn{1}{l|}{}                              \\ \midrule
%\rowcolor{purple!20} 
%\multicolumn{1}{|l|}{7}     & \multicolumn{1}{l|}{Regression Analysis}             & \multicolumn{1}{l|}{Display \& Liquor Sales} \\ \midrule
%\rowcolor{blue!20} 
%\multicolumn{1}{|l|}{8}     & \multicolumn{1}{l|}{Multiple Regression}             & \multicolumn{1}{l|}{}                              \\ \midrule
%\rowcolor{purple!20} 
%\multicolumn{1}{|l|}{9}     & \multicolumn{1}{l|}{Dealing with Nominal Covariates} & \multicolumn{1}{l|}{Gender Divide}                 \\ \midrule
%\rowcolor{blue!20} 
%\multicolumn{1}{|l|}{10}    & \multicolumn{1}{l|}{Regression Diagonistics}         & \multicolumn{1}{l|}{}                              \\ \midrule
%\rowcolor{purple!20} 
%\multicolumn{1}{|l|}{11-12} & \multicolumn{1}{l|}{Project Presentations}            &\multicolumn{1}{l|}{}          \\\midrule \bottomrule
%\end{tabular}
%\end{table}
%}
%\end{frame}


%\begin{frame}{Case Study }
%\qBoxCol{teal!40}{\vspace{1in}\begin{center}\sqBullet{teal} \Large Case: Liquor sales and display space \end{center}
%\vspace{1in}
%}\\
%\end{frame}





\section{Conditional Probability}
\TransitionFrame[amethyst]{\Large Conditional Probability  }



\begin{frame}\frametitle{Example0}
\qbx[4.5in]{amethyst!40}{
\Exmpl{amethyst}{} Suppose that we toss 2 dice, and suppose that
each of the 36 possible outcomes is equally likely to occur and
hence has probability $\frac{1}{36}$ . Suppose further that we observe that the  first die is a 3. Then, given this information, what is
the probability that the sum of the 2 dice equals 8?
}\\
%\pause
\vspace{2in}
{\tiny Solution:  }
\vspace{2in}

\end{frame}




\begin{frame}
\define{Conditional Probability}	
{
Let $E$,  and  $F$ are two events such that $P(F)>0$, then the conditional probability of {\bf $E$ given $F$} is defined to be, 
$$P(E\mid F):= \frac{P(E\cap F)}{P(F)}.$$
}
\end{frame}



\begin{frame}{Example}
\qBox{\Qn: Suppose that a balanced die is tossed once.  Find the conditional probability that 1 appears, given that an odd number was obtained.
 }\\

\vspace{1.5in}
A:= \{ Observe a 1.\}, 
B:=\{Observe an odd number.\}\\
 $P(A\mid B)=\frac{P(A\cap B)}{P(B)}=\frac{\frac{1}{6}}{\frac{3}{6}}=\frac{1}{3}.$
\end{frame}




\begin{frame}\frametitle{Example}
\qbx[4.5in]{apricot!70}{
\Exmpl{apricot}{} A student is taking a one-hour-time-limit makeup examination.  Suppose the probability that the student will
finish the exam in less than $x$ hours is $ \frac{x}{2}$, for all $0\leq x\leq 1.$ Then, given that the student is still working after $0.75$ hour, what is the conditional probability that the full hour is used?
}\\
%\pause
\vspace{1in}
{\tiny Solution: Let $L_x$ denote the event that the student finishes
the exam in less than $x$ hours, $0\leq  x\leq 1$, and let $F$ be the
event that the student uses the full hour.  Because $F$ is the
event that the student is not finished in less than 1 hour, 
$P(F)=P(\Not{L_1})=1-P(L_1)=0.5$.\\
Now, the event that the student is still working at time $0.75$ is the complement of the event $L_{0.75}$, so the desired probability is obtained from
$$P(F\mid \Not{L_{0.75}})= \frac{F\cap \Not{L_{0.75}}}{  \Not{L_{0.75}}  }= \frac{P(F)}{1- P(F_{0.75})}= \frac{0.5}{0.625}=0.8$$


}
\vspace{2in}

\end{frame}





\begin{frame}\frametitle{Example}
\qbx[4.5in]{apricot!70}{
\Exmpl{apricot}{} A coin is flipped twice.  Assuming that all four
points in the sample space $\HLTW{\SampleS = \HLTY{ \{HH, HT, TH, TT\}}}$ are
equally likely, what is the conditional probability that both flips land on heads, given that (a) the first  flip lands on heads? 
(b) at least one flip lands on heads?
}\\
%\pause
%\vspace{1in}
{\tiny Solution: 

}
\vspace{2in}
\end{frame}







\begin{frame}\frametitle{Example}
\qbx[4.5in]{olive!40}{
\Exmpl{olive}{} A total of $n$ balls are sequentially and randomly
chosen, without replacement, from an urn containing $r$ red
and b blue balls $(n \leq r + b)$. Given that $k$ of the $n$ balls are
blue, what is the conditional probability that the first ball chosen is blue?
}\\
%\pause
%\vspace{1in}
{\tiny {\bf Solution:}   Show that this probability is $ \frac{k}{n}$.

}
\vspace{2in}
\end{frame}








\begin{frame}
\define{Conditional Probability}	
{
$$P(E\cap F):=P(E\mid F)P(F).$$
}
\end{frame}




\begin{frame}\frametitle{Example}
\qbx[4.5in]{teal!40}{
\Exmpl{teal}{} Celine is undecided as to whether to take a
French course or a chemistry course. She estimates that her
probability of receiving an A grade would be $\frac{1}{2}$ in a French
course and $\frac{2}{3}$ in a chemistry course. If Celine decides to
base her decision on the  flip of a fair coin, what is the probability that she gets an A in chemistry?
}\\
%\pause
%\vspace{1in}
{\tiny {\bf Solution:}   .

}
\vspace{2in}

\end{frame}




\begin{frame}
\define{ The Multiplication Rule:}	
{
$$P(E\cap F):=P(E\mid F)P(F).$$
}
\end{frame}


\begin{frame}\frametitle{Example}
\qbx[4.5in]{amber!40}{
\Exmpl{amber}{} Suppose that an urn contains 8 red balls and 4
white balls. We draw 2 balls from the urn without
replacement. If we assume that at each draw each ball in the
urn is equally likely to be chosen, what is the probability that
both balls drawn are red?
}\\
\pause
\vspace{1in}
{\tiny {\bf Solution:}  
Let $R_1$ and $R_2$ denote, respectively, the events that
the first and second balls drawn are red. Now, given that the
first ball selected is red, there are 7 remaining red balls and 4
white balls, so $P(R_2\mid R_1) =\frac{7}{11}$. . As $P(R_1) = \frac{8}{12}$ , the desired probability is
$$P(R_1\mid R_2) =P(R_2\mid R_1)\times P(R1)=\frac{7}{11}\frac{8
}{12}=\frac{14}{33}.$$

}
\vspace{2in}

\end{frame}






\begin{frame}
\define{ The Generalized Multiplication Rule:}	
{
$P(E_1\cap E_2\cap \cdots E_k):=P(E_1) \times P(E_2\mid E_1)P(E_3\mid E_2\cap E_1 )\times \ldots \times P(E_k\mid E_1\cap E_2\cap \ldots\cap  E_{k-1}).$
}
\end{frame}


\begin{frame}\frametitle{Example}
\qbx[4.5in]{brightpink!40}{
\Exmpl{brightpink}{} An ordinary deck of 52 playing cards is randomly
divided into 4 piles of 13 cards each. Compute the probability
that each pile has exactly 1 ace.
}\\
\pause
\vspace{.9in}
{\tiny {\bf Solution:}  
Define events $E_i$, $i =1, 2, 3, 4$, as follows:
$E_1 = \{\text{the ace of spades is in any one of the piles}\},$\\
$E_2 = \{\text{the ace of spades and the ace of hearts are in different piles}\},$\\
$E_3 = \{\text{ the aces of spades, hearts, and diamonds are all in different piles}\},$\\
$E_4 = \{\text{all 4 aces are in different piles}\},$\\

The desired probability is $P(E_1\cap E_2 \cap E_3 \cap E_4)$, and by the multiplication rule,
$P(E_1\cap E_2 \cap E_3 \cap E_4)=P(E_1) P( E_2\mid  E_1)P(E_3 \mid E_1 \cap E_2) P(   E_4 \mid E_1\cap E_2 \cap E_3 ) $\\
Note that 
$P(E_11) = 1$ ,  because $E_1$ is equal to the sample space.
$P(E_2\mid E_1) = \frac{39}{51}$, $P(E3\mid E_1 \cap  E_2) =-\frac{26}{50}$,  $P(E_4 \mid E_1\cap E_2 \cap E_3) = \frac{13}{49}.$\\ Therefore, 
$P(E_1\cap E_2 \cap E_3 \cap E_4)=P(E_1) P( E_2\mid  E_1)P(E_3 \mid E_1 \cap E_2) P(   E_4 \mid E_1\cap E_2 \cap E_3 =1 \times   \frac{39}{51}\times \frac{26}{50} \times  \frac{13}{49}  = 0.105$
}
\vspace{2in}

\end{frame}





\section{Law of Total Probability}
\TransitionFrame[amethyst]{\Large Law of Total Probability}


\begin{frame}\frametitle{Law of  Total Probability}
\begin{center}
\includegraphics[scale=.35]{figs/Law_of_Total_Probability_Simple.png} 
\vspace{.2in}\\
\qBrd[2in]{applegreen!40}{$ E= (E\cap F) \cup (E\cap \Not{F}) $}\\
\qBrd[3in]{green!40}{$P(E)= P(E\cap F) + P(E\cap \Not{F}) $  }
\end{center}
\vspace{.5in}
\end{frame}

\begin{frame}\frametitle{Law of  Total Probability}
\qbx[4.5in]{teal!40}{
Let E and F be two events, then 
\begin{center}
\qBrd[3in]{green!40}{$ P(E)= \HLTW{P(E\mid F)P(F) + P(E\mid  \Not{F}))(\Not{F})} $}
\end{center}
\vspace{.2in}
}
\end{frame}

\begin{frame}\frametitle{Law of  Total Probability (General)}
\begin{center}
\includegraphics[scale=.35]{figs/Law_of_Total_Probability.png} 
\end{center}
\vspace{.4in}
\end{frame}

\begin{frame}\frametitle{Law of  Total Probability (General)}
\vspace{-.1in}

\qbx[4.5in]{turquoisegreen!90}{
\qBrd[2.5in]{yellow-green!50}{Law of  Total Probability (General):} Let E be an event.  Assuming that the collection of sets   $\{F_1, F_2,\ldots,  F_k\}$  forms a partition of $\SampleS$,  we have
\
\begin{center}
%\qBrd[2in]{applegreen!40}{$ E= (E\cap F) \cup (E\cap \Not{F}) $}\\
\qBrd[3in]{green!40}{$\displaystyle P(E)=\HLTW{\sum_{j=1}^{k}  P(E\mid F_j)P(F_j)  }$}
\end{center}
}

\end{frame}






\begin{frame}\frametitle{Example}
\vspace{-.1in}
\qbx[4.5in]{applegreen!40}{
\Exmpl{applegreen}{} An insurance company believes that people can be divided into two classes: those who are accident prone and those who are not. 
The companys statistics show that an accident-prone person will have an accident at some time within a fixed 1-year period with probability $0.4$, whereas this probability is $0.2$ for a person who is not accident prone.  If we assume that $30\%$
 of the population is accident prone, what is the probability
that a new policyholder will have an accident within a year of
purchasing a policy?
}\\
\pause
\vspace{.6in}
{\tiny {\bf Solution:}  
 Let $A_1$ denote the event that the policyholder will have an accident within a year of purchasing
the policy, and let $A$ denote the event that the policyholder is accident prone. Hence, the desired probability is given by
$P(A_1) = P(A_1\mid A)P(A) + P(A1\mid \Not{A})P(\Not{A}) = (0.4)(0.3)+(0.2)(0.7	)=0.26$
}
\end{frame}










\section{Bayes' Theorem}
\TransitionFrame[amethyst]{\Large Bayes' Theorem  }


\begin{frame}

\begin{theorem}
Let $\HLTW{F_1, F_2, \ldots,  F_{_K}}$ be a set of mutually exclusive and
exhaustive events (meaning that exactly one of these events
must occur). Suppose now that $E$ has occurred and we are
interested in determining which one of the $F_j$ also occurred.
Then, we have the following theorem
$$\DBX{ \displaystyle P(F_i \mid E)=\HLTW{ \frac{ \displaystyle P(E\mid F_i) P(F_i )}{\displaystyle \sum_{j=1}^{K}P(E\mid F_j) P(F_j) }}}$$
\end{theorem}



\end{frame}


\begin{frame}
\begin{itemize}
\item This theorem is well known as Bayes's theorem, after the
English philosopher {\bf Thomas Bayes}.
\vspace{.25in}


\item We can use the following Applet to find the conditional
probability using Bayes's theorem:
\vspace{.25in}




\item \url{http://www.thomsonedu.com/statistics/book_
content/0495110817_wackerly/applets/seeingstats/
Chpt2/bayesTree.html}
\end{itemize}



\end{frame}






\begin{frame}\frametitle{Example}
\vspace{-.1in}
\qbx[4.5in]{amethyst!40}{
\Exmpl{amethyst}{}  An insurance company believes that people can be divided into two classes: those who are accident prone and those who are not. 
The companys statistics show that an accident-prone person will have an accident at some time within a fixed 1-year period with probability $0.4$, whereas this probability is $0.2$ for a person who is not accident prone.  If we assume that $30\%$
 of the population is accident prone. \\
 Suppose that a new policyholder has an accident within a year of purchasing a policy. What is the probability that he or she is accident prone?
}\\
\pause
\vspace{.6in}
{\tiny {\bf Solution:}  
 The desired probability is 
 $$P(A\mid A_1)=  \frac{P(A\cap A_1)}{P(A_1)}= \frac{(0.4)(0.3)}{0.26}= \frac{6}{13} $$
}
\end{frame}



%__________________________________________
\begin{frame}{ Bayes' Theorem: Example   }
\vspace{-.1in}
\qBox{\Qn:
A diagnostic test for a disease is such that it (correctly) detects the disease in 90\% of the
individuals who actually have the disease. Also, if a person does not have the disease, the test
will report that he or she does not have it with probability 0.9. Only 1\%  of the population has the
disease in question. If a person is chosen at random from the population and the diagnostic test
indicates that she has the disease, what is the conditional probability that she does, in fact, have
the disease? Are you surprised by the answer? Would you call this diagnostic test reliable?}

\pause 
\vspace{.1in}

{\tiny 
$A=\{ \text{The individial has the disease} \},  \Not{A}=\{ \text{ does not have the disease} \}$, 
$ B=\{ \text{The test shows a POSITIVE result}\}$
$\Not{B}=\{ \text{The test shows a NEGATIVE result}\}$

$\HLTEQ[yellow]{ P(B\mid A)=0.9, P(\Not{B}\mid A^c)=0.9 \text{ and  } P(A)=0.01}$
 $$P(A\mid B)= \frac{P(B\mid A)P(A)}{  P(B\mid A)P(A)+ P(B\mid \Not{A})P(\Not{A}) }= \frac{(0.9)(0.01)}{(0.9)(0.01)+ (0.1)(0.99)}=91\%$$
}

\end{frame}


\begin{frame}\frametitle{Example}
\vspace{-.1in}
\qbx[4.5in]{atomictangerine!40}{
\Exmpl{atomictangerine}{}     Suppose that we have 3 cards that are identical in form, except that both sides of the first card are colored red, both sides of the second card are colored black, and one side of the third card is colored red and the other side black. The 3 cards are mixed up in a hat, and 1 card is randomly selected and put down on the ground. If the upper side of the chosen card is colored red, what is the probability that the other side is colored black?
}\\
\pause
\vspace{.6in}
{\tiny {\bf Solution: Let RR, BB, and RB denote, respectively, the events that the chosen card is all red, all black, or the redblack card. Also, let R be the event that the upturned side of the chosen card is red. Then
the desired probability is obtained
$$    P(RB\mid R)=  \frac{P(R\mid RB)P(RB)}{P(R\mid RR)P(RR)+ P(R\mid RB)P(RB) + P(R\mid BB)P(BB)}= \frac{\frac{1}{2}\times \frac{1}{3} }{   (1)\times \frac{1}{3}  + \frac{1}{2}\times \frac{1}{3}  +  (0) \times  \frac{1}{3}      }=  \frac{1}{3}   .$$   
}  

}
\end{frame}






\begin{frame}\frametitle{Example}
\vspace{-.1in}
\qbx[4.5in]{olive!40}{
\Exmpl{olive}{}     A new couple, known to have two children, has just moved into town. Suppose that the mother is encountered walking with one of her children. If this child is a girl, what is the probability that both children are girls?
}\\
\pause

{\tiny {\bf Solution:   

}  
\vspace{1.6in}
}
\end{frame}





%
%\begin{frame}\frametitle{Example}
%\vspace{-.1in}
%\qbx[4.5in]{teal!40}{
%\Exmpl{teal}{}     A bin contains 3 different types of disposable
%ashlights. The probability that a type 1 
%ashlight will give over 100 hours of use is .7, with the corresponding probabilities for type 2 and type 3  ashlights being .4 and .3, respectively. Suppose that 20 percent of the  ashlights in the bin are type 1, 30 percent are type 2, and 50 percent are type 3.
%}\\
%\pause
%
%{\tiny {\bf Solution:   
%
%}  
%\vspace{1.6in}
%}
%\end{frame}
%





\begin{frame}\frametitle{Example}
\vspace{-.1in}
\qbx[4.5in]{teal!40}{
\Exmpl{teal}{}     A bin contains 3 different types of disposable
ashlights. The probability that a type 1 
ashlight will give over 100 hours of use is 0.7, with the corresponding probabilities for type 2 and type 3  ashlights being 0.4 and 0.3, respectively. Suppose that 20 percent of the  ashlights in the bin are type 1,  30 percent are type 2, and 50 percent are type 3.
\begin{enumerate}
\item What is the probability that a randomly chosen 
ashlight will give more than 100 hours of use?
\item Given that a  ashlight lasted over 100 hours, what is the conditional probability that it was a type j  ashlight, j = 1, 2, 3?
\end{enumerate}
}\\
%\pause

{\tiny {\bf Solution:   

}  
\vspace{1.6in}
}
\end{frame}




\section{The Notion of Statistical Independence }
\TransitionFrame[amethyst]{\Large The Notion of Statistical Independence }




\begin{frame}\frametitle{Statistical Independence }
\define{Statistically Independent Event}{
Two events $E$ and $F$ are said to be  statistically independent if
$$\DBX{\displaystyle P(E\cap F ) = P(E)\times P(F)    }$$
}



\pause 

\begin{center}
\qBrd[4.1in]{amethyst!60}{Corollary: Two events E and F are independent if and only
if $P(E\mid F) = P(E)$ and $P(F\mid E) = P(F).$
}
\end{center}
\end{frame}


\begin{frame}\frametitle{Example}
\vspace{-.1in}
\qbx[4.5in]{amber!40}{
\Exmpl{amber}{}    A card is selected at random from an ordinary deck of 52 playing cards. If E is the event that the selected card is an ace and F is the event that it is a spade, then E and F are independent.
}\\
\pause

{\tiny {\bf Solution:   
This follows because $P(E\cap F) = \frac{1}{52}$ whereas
$P(E) = \frac{4}{52}$ and $P(F) = \frac{13}{52}.$
}  
\vspace{1.6in}
}
\end{frame}



\begin{frame}\frametitle{Example}
\vspace{-.1in}
\qbx[4.5in]{amber!40}{
\Exmpl{amber}{}   Two coins are 
flipped, and all 4 outcomes are assumed to be equally likely.  If E is the event that the first coin lands on heads and F the event that the second lands on tails, then E and F are independent, since
$P(E \cap F)= P(\{HT\})= \frac{1}{4}$. Where as $P(E)= P(\{HH, HT\})=\frac{1}{2}$ and $P(F)= P(\{HT, TT\})=\frac{1}{2}$ 
}\\
%\pause

%{\tiny {\bf Solution:   
%This follows because $P(E\cap F) = \frac{1}{52}$ whereas
%$P(E) = \frac{4}{52}$ and $P(F) = \frac{13}{52}.$
%}  
\vspace{1.6in}
%}
\end{frame}




\begin{frame}\frametitle{Example}
\vspace{-.1in}
\qbx[4.5in]{babyblue!40}{
\Exmpl{babyblue}{}   Suppose that we toss 2 fair dice. Let $E_1$ denote the
event that the sum of the dice is 6 and F denote the event that the
first die equals 4.  Is $E_1$ statistically independent of $F$?\\
\HLTW{\text{Answer:} }
   $P(E_1 \cap F)= P(\{(4,2)\})= \frac{1}{36}$ where as $P(E_1)=\frac{5}{36}$ and $P(F)= \frac{1}{6}$.  Therefore,  $E_1$ and $F$ are not statistically independet because, 
$P(E_1 \cap F)\neq P(E_1) \times P( F) $.
}\\
\vspace{.1in}
\qbx[4.5in]{asparagus!40}{
\Exmpl{asparagus}{}   Now, suppose that we let $E_2$ be the event that the sum of the dice equals $7$.   Is $E_2$ statistically independent of $F$?\\
\HLTW{\text{Answer:} }
$P(E_2\cap F)= P(\{(4,3)\})= \frac{1}{36}$ where as $P(E_2)=\frac{1}{6}$ and $P(F)= \frac{1}{6}$.  Therefore,  $E_2$ and $F$ are  statistically independet because, 
$P(E_2 \cap F)= P(E_2) \times P( F) $.
}\\
%\pause

%{\tiny {\bf Solution:   
%This follows because $P(E\cap F) = \frac{1}{52}$ whereas
%$P(E) = \frac{4}{52}$ and $P(F) = \frac{13}{52}.$
%}  
%\vspace{1.6in}
%}
\end{frame}



\begin{frame}
\qBrd[4.5in]{bazaar!40}{Proposition: If $E$ and $F$ are independent, then so are $E$ and $\Not{F}$.}
{\tiny Solution:}
\vspace{2in}
\end{frame}








\begin{frame}\frametitle{Example}
\vspace{-.1in}
\qbx[4.5in]{amber!40}{
\Exmpl{amber}{}    A card is selected at random from an ordinary deck of 52 playing cards. If E is the event that the selected card is an ace and F is the event that it is a spade, then E and F are independent.
}\\
\pause

{\tiny {\bf Solution:   
This follows because $P(E\cap F) = \frac{1}{52}$ whereas
$P(E) = \frac{4}{52}$ and $P(F) = \frac{13}{52}.$
}  
\vspace{1.6in}
}
\end{frame}

 

\begin{frame}\frametitle{Generalized Definition of Statistical Independence }
\define{Statistically Independent Events}{
Three events $E$, $F$, and $G$ are said to be  statistically independent if
\begin{center}
\qBrd[3.5in]{applegreen!40}{ $P(E\cap F \cap G ) = P(E)\times P(F)  \times P(G)  $}\\
\qBrd[2.5in]{purple!30}{ $P(E\cap F ) = P(E)\times P(F)  $}\
\qBrd[2.5in]{amethyst!30}{ $P(E\cap G ) = P(E)\times P(G)  $}\\
\qBrd[2.5in]{bittersweet!40}{ $P(F\cap G ) = P(F)\times P(G)  $}\\

\end{center}
}
\vspace{1.5in}

\end{frame}

\TransitionFrame[antiquefuchsia]{\Large Questions?  }
 
 
 
\end{document}
